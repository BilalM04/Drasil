\chapter{Lesson Plans} \label{chap:casestudy}
With the addition of a JSON printer capable of generating Jupyter Notebooks, we 
are now looking to expand Drasil's application by generating educational 
documents. As discussed in Chapter~\ref{chap:intro}, Jupyter Notebooks are 
commonly used in teaching engineering courses due to their characteristics and 
advantages. One of the educational practices to enhance education is conducting 
lesson plans \cite{cicek2013effective, wong2018first}, which provide a 
guide for structuring daily activities in each class period. A lesson plan 
outlines the learning objectives, methods and procedures for achieving them, 
and the measurement of how student progress. Lesson plans are an ideal starting 
point for generating educational documents in Drasil because they are more 
accessible than academic papers. In addition, we are able to work with real 
examples in a lesson plan. This chapter will cover the structure of a lesson 
plan, how we define the language of lesson plans in Drasil, and a new case 
study on Projectile Lesson.

\section{Language of Lesson Plans} \label{chap:lessonLang}
To generate a new type of document, lesson plans, in Drasil, we must define its 
language first. Drasil's document language has SRS, and we are creating a 
language for lesson plans. As discussed in Chapter~\ref{chap:nbprinter}, a 
Drasil document has a title, authors, and sections, which hold the contents 
of the document. The definition of a document is defined in 
\textbf{drasil-lang} \footnote{\textbf{drasil-lang} holds the higher level 
language for Drasil.} as shown in Code~\ref{code:drasil-lang-document} 
\footnote{ShowToC is ShowTableOfContents in the source code, which is to 
determine whether to show the table of contents in the document.}, where 
\macblue{Document} is the type for SRS document and \macblue{Notebook} is for 
Jupyter Notebook, specifically lesson plans at this moment. The reason why we 
define them separately is because we print the SRS and lesson plans 
differently. We are able to pattern match the way we print the document in the 
printer.

\begin{listing}[h]
	\caption{Pseudocode for Definition of Document}
	\label{code:drasil-lang-document}
	\begin{lstlisting}[language=haskell1]
	data Document = Document Title Author ShowToC [Section]
								| Notebook Title Author [Section]
	\end{lstlisting}
\end{listing}

Before defining the language for lesson plans, we need to understand the 
components and categorize the knowledge to create a universal structure within 
Drasil. We analyzed the similarities and differences of elements in textbook 
chapters in 
\href{https://github.com/smiths/caseStudies/blob/master/CaseStudies/projectile/projectileLesson/AboutProjectileLesson.pdf}{Discussion
 of Projectile Lesson: What and	Why} using online resources. Based on our 
analysis, we narrowed down the elements and defined a structure that fits our 
lesson plans the most. It's worth noting that this structure may be subject to 
future modifications to better suit our needs. Following is the structure of
our lesson plans:
\begin{itemize}
	\item Introduction: an introduction of the lesson plan or the topic.
	\item Learning Objectives: what students can do or will learn after the 
	lesson.  
	\item Review: a recap of what has been covered previously.
	\item A Case Problem: a case problem that link the topic to a real world 
	problem.
	\item Example: an example of the case problem.
	\item Summary: a summary of the lesson plan.
	\item Bibliography: references that support the lesson plan.
	\item Appendix: additional resources or information of the lesson.
\end{itemize}

With the lesson plan structure in place, we can now define helper types and 
functions to create the document language for generating lesson plans. Our 
first step is to define the types and data for the lesson and its chapters in 
Drasil's document language, \textbf{drasil-docLang}. Code~\ref{code:core} is 
the core declaration of the lesson plan. A \macblue{LsnDesc} type represents a 
lesson description (line 3), which consists of lesson chapters (line 5), 
including an introduction, learning objectives, review, case problem, example, 
summary, bibliography, and appendix. The detail structure of each chapter is 
defined in line 14-33. At present, \macblue{Contents} is the only defined 
elements as the chapter structure has not yet been fully understood. We intend 
to further develop the chapter structure in the future. \todo{link to future 
work} 

\begin{listing}[h!]
	\caption{Source Code for Notebook Core Language}
	\label{code:core}
	\begin{lstlisting}[language=haskell1]		
		type LsnDesc = [LsnChapter]

		data LsnChapter = Intro Intro
										| LearnObj LearnObj
										| Review Review
										| CaseProb CaseProb
										| Example Example
										| Smmry Smmry
										| BibSec
										| Apndx Apndx
		
		-- ** Introduction
		newtype Intro = IntrodProg [Contents]
		
		-- ** Learning Objectives
		newtype LearnObj = LrnObjProg [Contents]
		
		-- ** Review Chapter
		newtype Review = ReviewProg [Contents]
		
		-- ** A Case Problem
		newtype CaseProb = CaseProbProg [Contents]
		
		-- ** Examples of the lesson
		newtype Example = ExampleProg [Contents]
		
		-- ** Summary
		newtype Smmry = SmmryProg [Contents]
		
		-- ** Appendix
		newtype Apndx = ApndxProg [Contents]
	\end{lstlisting}
\end{listing}

The \macblue{LsnDecl} type, as shown in Code~\ref{code:LsnDecl}, is used to 
declare all the necessary chapters for a lesson plan. It is similar in 
definition to \macblue{LsnDesc}, but in a more usable form. It is meant to be a 
semantic rendition of a lesson plan document, while \macblue{LsnDesc} is 
intended to be a general description and more suitable for printing 
\cite{lsnDeclandlsnDesc}. They are identical at this point because the chapter 
structure is not well understood, but they might evolve differently as we gain 
more understanding of our lesson plans.

\begin{listing}[h]
	\caption{Source Code for LsnDecl}
	\label{code:LsnDecl}
	\begin{lstlisting}[language=haskell1]
		type LsnDecl  = [LsnChapter]
		
		data LsnChapter = Intro NB.Intro
										| LearnObj NB.LearnObj
										| Review NB.Review
										| CaseProb NB.CaseProb
										| Example NB.Example
										| Smmry NB.Smmry
										| BibSec
										| Apndx NB.Apndx
	\end{lstlisting}
\end{listing}

\begin{listing}[h!]
	\caption{Source Code for Section and the Constructor}
	\label{code:section}
	\begin{lstlisting}[language=haskell1]
		data Section = Section
		{ tle  :: Title
			, cons :: [SecCons]
			, _lab :: Reference
		}
		makeLenses ''Section
		
		-- | Constructor for creating 'Section's with a 
		-- title ('Sentence'), introductory contents, 
		-- a list of subsections, and a	shortname ('Reference').
		section :: Sentence -> [Contents] -> [Section] -> Reference -> Section
		section title intro secs = Section title (map Con intro ++ map Sub secs)
	\end{lstlisting}
\end{listing}

Next, we need functions to generate chapters. We can use the \macblue{Section} 
type, as shown in Code~\ref{code:section}, which consists of a title, a list of 
contents, and a short name that is used for creating SRS sections. We can also 
take advantage of the \macred{section} smart constructor to build our own 
chapter constructors, as illustrates in Code~\ref{code:chapterConstructor}. 
Once we have these constructors, we can use them to build each chapter 
(Code~\ref{code:mkChapters}).

\begin{listing}[h!]
	\caption{Source Code for Chapter Constructors} 
	\label{code:chapterConstructor}
	\begin{lstlisting}[language=haskell1]
		learnObj, review, caseProb, example :: [Contents] -> 
				[Section] -> Section
		learnObj cs ss = section (titleize' Doc.learnObj) cs ss learnObjLabel
		review   cs ss = section (titleize Doc.review)    cs ss reviewLabel
		caseProb cs ss = section (titleize Doc.caseProb)  cs ss caseProbLabel
		example  cs ss = section (titleize Doc.example)   cs ss exampleLabel
	\end{lstlisting}
\end{listing}

\begin{listing}[h!]
	\caption{Source Code for Making Chapters} 
	\label{code:mkChapters}
	\begin{lstlisting}[language=haskell1]				
		-- | Helper for making the 'Learning Objectives'.
		mkLearnObj :: LearnObj -> Section
		mkLearnObj (LrnObjProg cs) = Lsn.learnObj cs []
		
		-- | Helper for making the 'Review'.
		mkReview :: Review -> Section
		mkReview (ReviewProg r) = Lsn.review r [] 
		
		-- | Helper for making the 'Case Problem'.
		mkCaseProb :: CaseProb -> Section
		mkCaseProb (CaseProbProg cp) = Lsn.caseProb cp [] 
		
		-- | Helper for making the 'Example'.
		mkExample:: Example -> Section
		mkExample (ExampleProg cs) = Lsn.example cs []
	\end{lstlisting}
\end{listing}

When building lesson plans, the document and chapters are encoded in the 
\macblue{LsnDecl} type, which is then converted to \macblue{LsnDesc} for 
printing. In Code~\ref{code:mkNb}, the \macred{mkNb} function takes the 
user-encoded list of chapters (i.e., \macblue{LsnDecl}) and \textbf{System 
Information} \footnote{System Information is a data structure designed to 
contain all the necessary information about a system for the purpose of 
generating artifacts.} to form a lesson plan document.

\begin{listing}[h!]
	\caption{Source Code for mkNb}
	\label{code:mkNb}
	\begin{lstlisting}[language=haskell1]
	mkNb :: LsnDecl -> (IdeaDict -> IdeaDict -> Sentence) 
			 -> SystemInformation -> Document
	mkNb dd comb si@SI {_sys = s, _kind = k, _authors = a} =
		Notebook (nw k `comb` nw s) (foldlList Comma List $ 
		map (S . name) a) $	mkSections si l where
			l = mkLsnDesc si dd
	
	-- | Helper for creating the lesson plan sections.
	mkSections :: SystemInformation -> LsnDesc -> [Section]
	mkSections si = map doit where
		doit :: LsnChapter -> Section
		doit (Intro i)     = mkIntro i
		doit (LearnObj lo) = mkLearnObj lo
		doit (Review r)    = mkReview r
		doit (CaseProb cp) = mkCaseProb cp
		doit (Example e)   = mkExample e
		doit (Smmry s)     = mkSmmry s
		doit BibSec        = mkBib (citeDB si)
		doit (Apndx a)     = mkAppndx a
	
	mkLsnDesc :: SystemInformation -> LsnDecl -> NB.LsnDesc
	mkLsnDesc _ = map sec where
	sec :: LsnChapter -> NB.LsnChapter
	sec (Intro i)     = NB.Intro i
	sec (LearnObj l)  = NB.LearnObj l
	sec (Review r)    = NB.Review r  
	sec (CaseProb c)  = NB.CaseProb c
	sec (Example e)   = NB.Example e  
	sec (Smmry s)     = NB.Smmry s
	sec BibSec        = NB.BibSec
	sec (Apndx a)     = NB.Apndx a
	\end{lstlisting}
\end{listing}

All types and functions mentioned in this chapter are declared in 
\textbf{drasil-docLang}, where the languages of SRS and lesson plans are 
located. Table~\ref{tab:notebookLang} summaries the responsibility for each 
module related to lesson plans.

\begin{longtable}[c]{|>{\raggedright}p{0.27\linewidth}|>{\raggedright\arraybackslash}p{0.69\linewidth}|}
	\caption{Summary of Notebook Modules} 
	\label{tab:notebookLang}                                              
	\\ \hline
	
	\rowcolor{McMasterMediumGrey}
	\textbf{Module} & \textbf{Responsibility}
	\\ \hline
	\multicolumn{2}{|l|}{\textbf{Drasil.DocLang}} 
	\\ \hline
	Notebook.hs & Contains constructors for building chapters.
	\\ \hline
	\multicolumn{2}{|l|}{\textbf{Drasil.DocumentLanguage.Notebook}} 
	\\ \hline
	Core.hs & Contains general description functions for lesson plans.
	\\ \hline
	DocumentLanguage.hs & Holds functions to create chapters and form a lesson 
	plan.
	\\ \hline
	LsnDecl.hs & Contains declaration functions for generating lesson plans. 
	\\ \hline
\end{longtable}


\section{A Case Study: Projectile Motion}
In Chapter~\ref{chap:lessonLang}, we discussed the language of lesson plans to 
introduce a new case study on projectile motion. We chose projectile motion as 
a starting point for our lesson plans for several reasons: i) it is often one 
of the initial concepts taught when students are introduced to the study of 
dynamics; ii) the developed model is considered relatively straightforward as 
it solely incorporates kinematic, which pertains to the geometric 
characteristics of motion \cite{smith2022projectile}; iii) Drasil already 
captures the knowledge of projectile, allowing us to showcase the reuse of 
knowledge. In this chapter, we are going to discuss how we reproduce the 
\href{https://github.com/smiths/caseStudies/blob/master/CaseStudies/projectile/projectileLesson/orgModeVersion/projMotLesson.pdf}{Projectile
 Motion Lesson} created by Dr. Spencer Smith and let Drasil generate the 
notebook document.

what information is reused from projectile.
