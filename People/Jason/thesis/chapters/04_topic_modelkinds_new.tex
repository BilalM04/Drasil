\chapter*{Classifying Theories}
\label{chap:framing-theories}

\begin{writingdirectives}
    \item What are they?
    \item What are theories used for?
    \item How are they captured in Drasil?
    \item Current problems? Solutions?
\end{writingdirectives}

In this chapter, we will focus on improving how Drasil captures ``theory''
knowledge to improve inspection and specialized interpretation. Specifically,
with focus on interpreting them for generating software artifacts.

\section{Theories}
\label{chap:framing-theories:sec:theories}

As mentioned in \Cref{chap:drasil}, the \acs{srs} template
\cite{SmithAndLai2005} breaks up software requirements and problems into a
series of well-understood components, providing developers with concrete
solution requirements they must satisfy, and domain experts with justification
for problem solutions. Notably, the \acs{srs} relates a programs \textit{inputs}
to a set of \textit{outputs} using a set of \textit{theories}. The inputs and
outputs are sets of variables, with data that need to be somehow fed into the
program, or calculated and output by the program. The theories connect the input
variables to the output by forming a \textit{solution/calculation path}. There
are at least 3 kinds of theories: Theory Models, General Definitions, and
Instance Models. Theory Models and General Definitions provide justification for
the mathematics of the solution: the Instance Models. The Instance Models,
specifically, together form the calculation path. There's also a 4th kind: Data
Definitions, intended for explaining how input variables should be interpreted
by the solution program, thus also explaining how they should be formatted; they
are typically intended to be implementation-focused, rather than theoretical.

For example, Drasils \acs{projectile} case study describes how to estimate if a
launcher, aligned at a particular angle, will hit a target from a specific
distance. The \acs{srs} requires users to fill in the:
\begin{enumerate}
    \item input variables, such as \(a_x\) and \(a_y\) (x and y components of
          acceleration), \(\mathbf{g}\) (gravitational acceleration constant),
          \(p^{i}\) (initial position)
    \item output variables, such as
    \item and theories connecting them, such as
\end{enumerate}

\section{Classifying by Structure}
\label{chap:framing-theories:sec:classifying-and-structuring}
