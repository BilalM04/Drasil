\chapter{More Theory Kinds}
\label{chap:more-theory-kinds}

\begin{writingdirectives}
	\item What other theory kinds can we find in the existing Drasil examples?
	\item How is ModelKinds affected by the language division in
	\Cref{chap:lang-division}?
	\item What theories currently are used in Drasil?
\end{writingdirectives}

In this chapter, we return to examining the existing theories in Drasil, hoping
to discover more \textit{kinds} of theories in them to create further
opportunities for domain-specific interpretation.

\section{\textquotedblleft{}Classify All The Theories\textquotedblright{}}
\label{chap:more-theory-kinds:sec:classify-all-the-theories}

With \Cref{chap:lang-division}, we divided the \Expr{} language into 3 variants
and use them to indicate usability of theories in code generation. Now, we may
return to examining the existing encoded theories in Drasil, while also begin
careful to ensure that all created theory kinds are expressible in the ``general
mathematical language'' (i.e., \ModelExpr{}). We may do this by ensuring they
all instantiate the \Express{} (\refCurrentExpressHaskell{}) typeclass. Starting
at the end and working our way back, we end with
\refCurrentModelKindsHaskell{}\footnote{In the \refCurrentModelKindsHaskell{}
	definition, there are two (2) TO-DO notes that you may disregard. The first one
	is merely a note for analyzing ``well-understood'' copies of our existing
	\acsp{ode}, and the second one refers to models that haven't yet been fully
	analyzed for how they will be used (other than for display).}.

\currentModelKindsHaskell{}

\subsection{Equational Constraints}
\label{chap:more-theory-kinds:sec:classify-all-the-theories:subsec:equational-constraints}

``Equational constraints'' are theories that assert certain properties over
other theories. They use \ConstraintSet{}s under the hood
(\refCurrentConstraintSetHaskell{}) to hold a list of relations for assertion.

\currentExampleEquationalConstraintsHaskell{}

\currentConstraintSetHaskell{}

\subsection{Equational Realms}
\label{chap:more-theory-kinds:sec:classify-all-the-theories:subsec:equational-realms}

Equational realms represent ``realms'' \cite{Carette2014realms}\qtodo{@JC: Is
	this the preferred reference?}, sets of unique axioms that are equivalently
interpretable, focused on different ways to define a particular variable. They
may be specialized to become equational models. \EquationalRealm{}s represent
``equational realm'' theories in Drasil and are effectively \MultiDefn{}s. For
example, we may define a theory with multiple ways to define the horizontal
force on an object: \refCurrentExampleEquationalRealmHaskell{}.

\currentExampleEquationalRealmHaskell{}

\currentMultiDefnHaskell{}

\currentDefiningExprHaskell{}

\subsection{Differential Equations}
\label{chap:more-theory-kinds:sec:classify-all-the-theories:subsec:differential-equations}

The capture of differential equations in Drasil is an active area of research.
Dong Chen continued work here, creating \NewDEModel{} \cite{Chen2022MEng} to
start capturing information about linear \acs{ode} systems. \DEModel{} is left
as a temporary carriage for the remaining theories to be similarly analyzed and
re-built with a deeper depth of knowledge capture, so that we can make better
use of the information in them. Thanks to Dongs work, Drasil is now able to
generate software for the \acs{dblpendulum} case study in Java, Python, C/C$++$,
and C\# \cite{Chen2022MEng}. As such, this research already has some success in
enabling more theories to be encoded in Drasil and appropriately used for
various purposes.

\subsection{Theories Left Undiscussed}
\label{chap:more-theory-kinds:sec:classify-all-the-theories:subsec:theories-left-undiscussed}

While we have analyzed a few theories and how they're used, there are still many
theories left undiscussed\footnote{\ModelKinds{} is an incomplete enumeration.}.
Following Drasils methodology, they will only be analyzed and captured as
necessary. Notably, \ModelKinds{} still contains \OthModel{}, meaning that there
still exist theories in Drasil, which are used in justifications, of which we
haven't yet decided how we want to use yet\footnote{This is ``future work'' for
now.}.

Finally, as a result of implementing \ModelKinds{} (\Cref{chap:modelkinds}) and
the expression language division (\Cref{chap:lang-division}), we are now able
to, in at least one way, restrict the terms we use in different kinds of
theories across different contexts. Ultimately, this adds some assurance that
all generated artifacts only contain relevant language in them, because we have
filtered out terms by their context. However, we have yet to discuss
\textit{coherence}\footnote{Here, meaning well-formedness.} of expressions in
their contexts.
