\begin{writingquestions}
      \item What is the context of this research? What is it about?
      \item What problem does this research tackle?
      \item Why is the research problem important/significant?
      \item What previous research exists?
      \item What is the purpose of this research? What are the goals?
      \item What did the author contribute? \wqanswer{\Cref{sec:intro:contributions}}
      \item What is in this work? \wqanswer{\Cref{sec:intro:outline}}
\end{writingquestions}


\section{Problem Statement}
\label{sec:intro:problemStatement}

In Drasil, we are focused on understanding families of scientific software, and
creating systematic rules to generate families of software solutions (for any
instance of a scientific problem that requires a scientific software solution).
Specifically, Drasil is focused on mathematics and physics-based models. In both
areas, we are concerned with what \textit{kinds of theories} are
well-understood, and ensuring that all created mathematical expressions are
``valid''.

\intodo{Re-write the above.}

\section{Contributions of the Author}
\label{sec:intro:contributions}

\begin{itemize}
      \item The problem of obtaining more information from our theories so that we
            may make better use from their meta-level information was understood
            before this work.
      \item A partial solution to this problem was also constructed before this
            work.
      \item The expression language used to build theories and, ultimately, inputs
            to outputs was constructed before this work.
\end{itemize}

\section{Thesis Outline}
\label{sec:intro:outline}

In \Cref{chap:ideology}, we discuss the focal ideology underpinning this work,
and Drasil. \Cref{chap:drasil} describes Drasil, the host project carrying the
fruits of this work. \Cref{chap:modelkinds} discusses how theories are encoded
in Drasil, the issues associated with using a single universal mathematical
language to describe theories, and how we can resolve these problems.
\Cref{chap:typedExpr} describes residual issues associated with validating and
transcribing mathematical expressions. \Cref{chap:knowledgeMgmt} continues to
discuss the ways in which general knowledge and theories are encoded in Drasil,
and methods for altering their encoding, leading into \cref{chap:futureWork},
where we discuss future work.
