\chapter{Introduction}
\label{chap:introduction}

\begin{writingdirectives}

      \item Move 1: Establishing a research territory by:
      \begin{itemize}

            \item showing research area is important, interesting, and
                  incomplete

            \item reviewing previous research

      \end{itemize}

      \item Move 2: Establishing a niche by noting gaps in previous research.

      \item Move 3: Occupying the niche by:
      \begin{itemize}

            \item outlining purpose

            \item listing research questions

            \item announcing principal findings

            \item stating the value of the previous research

      \end{itemize}

      \item General Structure:
      \begin{itemize}

            \item Introduction:
                  \begin{itemize}

                        \item Jazzy information to get reader hooked

                        \item States purpose of chapter

                        \item Roadmap of what will be discussed in chapter

                  \end{itemize}

            \item Background: context of research problem, sets up the need for
                  research and relevance

            \item PPSQ: should be within first 3 pages of thesis, after intro +
                  background information.

            \item Research design and context: description of where the research
                  takes place (Drasil), introducing methodology briefly

            \item Assumptions, limitations, scope of research, and expected
                  outcomes: what do we need from this work

            \item Overview of chapters

      \end{itemize}

      \item Last Paragraph: summarize key points of chapter, link to next
      chapter

      \item What is the context of this research? What is it about?

      \item What problem does this research tackle?

      \item Why is the research problem important/significant?

      \item What previous research exists?

      \item What is the purpose of this research? What are the goals?

      \item What did the author contribute?
      \wqanswer{\Cref{sec:intro:contributions}}

      \item What does this thesis contain? \wqanswer{\Cref{sec:intro:outline}}

\end{writingdirectives}

As the world increasingly relies on software and non-executable software
artifacts, the software becomes increasingly scrutinized in multiple facets.
Does a piece of software do what I want/need? Is it stable? Is it reliable? What
methods does it use? What do its components mean? Are its outputs accurate and
precise? Is it performant? Is it efficient? Further, to each answer, one might
ask: How do we know? Often, we look to the source code and its development to
justify responses to these questions. Where a requirements specification and a
software design is built, we can spend time investigating that the source code
conforms to the manuals, and that the manuals form satisfactory answers to our
questions. Where a manual is not built, we must spend time studying source code
to respond to these questions, perhaps even comparing them to a control group
program. However, this is terribly arduous process. Realistically, the answers
to these questions should have been already well-understood before the final
software product was formed, even if the answer was to ignore the question until
later. Drasil \cite{Drasil2021} looks to form answers to these same questions
through \textit{definition}. Using generative techniques, Drasil builds software
that conforms to specifications \textit{perfectly} by having all relevant
information of the specifications and how they translate into some software
artifacts well-understood to Drasil. This thesis aims to explore these ideas
through exploring capturing mathematical knowledge in Drasil.

\section{Problem Statement}
\label{sec:intro:problemStatement}

In the land of scientific software, mathematical knowledge and it's conversion
into software is of utmost importance to ensure the software is:

\intodo{I think I saw some better worded goals previously? TODO: Check Dan's work and Brook's work}
\begin{enumerate}

      \item \textbf{Ready \& Safely Usable} The software should be able to
            easily reach the runtime phase, and should never run into unexpected
            issues at runtime.

      \item \textbf{Reliable} The user of the software should be able to trust
            and feel comfortable that the software does as expected.

      \item \textbf{Accurate} The software should be a correct depiction of all
            aspects of the needs of the software.

      \item \textbf{Traceable} Partially related to being accurate, the
            ``accuracy'' should be traceable to the knowledge used to form the
            software.

      \item \textbf{Performant and efficient} The software should be slim, use
            only as much system resources as needed, and perform tasks within a
            reasonably usable and actionable time.

\end{enumerate}

As part of Drasil's focus, this mathematical knowledge must become sufficiently
codified such that users may be able to automatically build a program that
satisfies an \acs{srs} document. In other words, building programs exactly
according to the precise problem it solves by understanding the relevant
definitions and descriptions (using a specific \acs{srs}). By doing this, Drasil
will be able to reliably and flexibly generate a host of \acs{stem}-related
software artifacts.

\section{Contributions of the Author}
\label{sec:intro:contributions}

Dr. Jacques Carette had already built a prototype solution \todo{cite appendix
entry of code? I'm not sure, yet, how I should cite what Dr. Carette had already
written.} for the issue of transcribed theories in Drasil not exposing enough
information for reliable code generation (as discussed in
\Cref{chap:modelkinds}). Relying on Dr. Jacques Carettes prototype, A current
implementation and other re-designs in the works have been developed by the
author of this work. Both issues discussed in \Cref{chap:modelkinds} and
\Cref{chap:typedExpr} were discussed before the author had joined the Drasil
research group. Drasil, and the expression language in particular, had been
constructed before this work.

\section{Thesis Outline}
\label{sec:intro:outline}

In \Cref{chap:ideology}, we discuss the focal ideology underpinning this work
and Drasil. \Cref{chap:drasil} describes Drasil, the host project carrying the
fruits of this work. \Cref{chap:modelkinds} discusses how theories are encoded
in Drasil, the issues associated with using a single universal mathematical
language to describe theories, and how we can resolve these problems.
\Cref{chap:typedExpr} describes residual issues associated with validating and
transcribing mathematical expressions. \Cref{chap:storingChunks} continues to
discuss the ways in which general knowledge and theories are encoded in Drasil,
and methods for altering their encoding, leading into \cref{chap:futureWork},
where we discuss future work.
