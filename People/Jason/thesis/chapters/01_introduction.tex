\chapter{Introduction}
\label{chap:introduction}

\begin{writingdirectives}

      \item Move 1: Establishing a research territory by:
      \begin{itemize}

            \item showing research area is important, interesting, and
                  incomplete

            \item reviewing previous research

      \end{itemize}

      \item Move 2: Establishing a niche by noting gaps in previous research.

      \item Move 3: Occupying the niche by:
      \begin{itemize}

            \item outlining purpose

            \item listing research questions

            \item announcing principal findings

            \item stating the value of the previous research

      \end{itemize}

      \item General Structure:
      \begin{itemize}

            \item Introduction:
                  \begin{itemize}

                        \item Jazzy information to get reader hooked

                        \item States purpose of chapter

                        \item Roadmap of what will be discussed in chapter

                  \end{itemize}

            \item Background: context of research problem, sets up the need for
                  research and relevance

            \item PPSQ: should be within first 3 pages of thesis, after intro +
                  background information.

            \item Research design and context: description of where the research
                  takes place (Drasil), introducing methodology briefly

            \item Assumptions, limitations, scope of research, and expected
                  outcomes: what do we need from this work

            \item Overview of chapters

      \end{itemize}

      \item Last Paragraph: summarize key points of chapter, link to next
      chapter

      \item What is the context of this research? What is it about?

      \item What problem does this research tackle?

      \item Why is the research problem important/significant?

      \item What previous research exists?

      \item What is the purpose of this research? What are the goals?

      \item What did the author contribute?
      \wqanswer{\Cref{sec:intro:contributions}}

      \item What does this thesis contain? \wqanswer{\Cref{sec:intro:outline}}

\end{writingdirectives}

\intodo{Add links to where these above writing directives were derived from.}

The usual means of building software involves multiple artifacts (such as
specifications and code) that contain duplicate information that is also
supposed to be linked (\textit{traceability}). Drasil aims to use a generative
approach to de-duplicate this information and make traceability more immediate.

For any ``software'', its artifacts (such as specifications and code) are linked
together by a common thread of knowledge. This knowledge typically ends up used
in the production of the software artifacts, and, often, appears in different
forms and is duplicated, in the produced artifacts. Drasil\cite{Drasil2021} aims
to use a generative approach to de-duplicate this information and make the
\textit{traceability} of knowledge more immediate. Drasil currently studies
generating \ACF{scs} conforming to a \ACF{srs}. This work aims to

As the world increasingly relies on software and non-executable software
artifacts, the software becomes increasingly scrutinized in multiple facets.
Does a piece of software do what I want/need? Is it stable? Is it reliable? What
methods does it use? What do its components mean? Are its outputs accurate and
precise? Is it performant? Is it efficient? Further, to each answer, one might
ask: How do we know? Often, we look to the source code and its development to
justify responses to these questions. Where a requirements specification and a
software design is built, we can spend time investigating that the source code
conforms to the manuals, and that the manuals form satisfactory answers to our
questions. Where a manual is not built, we must spend time studying source code
to respond to these questions, perhaps even comparing them to a control group
program. However, this is terribly arduous process. Realistically, the answers
to these questions should have been already well-understood before the final
software product was formed, even if the answer was to ignore the question until
later. Drasil \cite{Drasil2021} looks to form answers to these same questions
through \textit{definition}. Using generative techniques, Drasil builds software
that conforms to specifications by having all relevant information of the
specifications and how they translate into some software artifacts
well-understood to Drasil. This thesis aims to explore these ideas through
exploring capturing mathematical knowledge in Drasil.

\section{Problem Statement}
\label{sec:intro:problemStatement}

Drasil has de-duplicated knowledge across \acs{scs} artifacts relevant to
specifications and code. Through codifying knowledge and collecting a coherent
set of information in a knowledge database, we are able to generate a wide
variety of software artifacts (e.g., \acs{oo} programs [Java, C$++$, Python,
            Swift, and C\#] with guided usage via Makefiles, and requirements specifications
      [HTML and TeX]). This codified knowledge was de-duplicated from an originating
set of artifacts via bottom-up gathering, however, we should be able to use the
same knowledge to generate more artifacts in different languages, flavours, and
with more options. However, each desired artifact language has its own way of
encoding information (such as mathematical expressions). This leaves us needing
to teach Drasil more about the targeted languages (and, at times, about the
existing codified knowledge) in order to reliably generate usable artifacts.
Mathematical expression and theory encoding becomes a key point of interest for
us because they are used in across the board (e.g., derivations, code,
constraints, definitions, etc.). Drasil relies on a single universal untyped
mathematical language to describe general mathematical knowledge (theories).
Unfortunately, this results in unreliable and brittle conversions of
mathematical knowledge into other forms because, we, lack information about
their structure (leading to inflexible conversions to other forms), don't
statically know when expressions are admissible in different contexts (e.g., in
code generation, derivations, etc.), and we don't know when are well-formed
(well-typed). As more theories are codified and typed, Drasils knowledge
database faces difficulties in scaling since it relies on a single unique map
for each type of knowledge, resulting in an ever-growing list of maps and a
tediously precise means of knowledge collection and reference.

\section{Research Questions}

\begin{enumerate}

      \item[\namedlabel{rq:one}{RQ1}] Drasil has a language of simple
            mathematical expressions that are used in multiple contexts. But not
            all expressions are valid in all contexts. How do we fix that?

      \item[\namedlabel{rq:two}{RQ2}] Drasil's current encoding of "theories"
            are essentially black boxes. We would like to be able to use some
            structural information present in the short list of the ``kinds'' of
            theories that show up in scientific computing. How do we codify
            that?

      \item[\namedlabel{rq:three}{RQ3}] How can we ensure that our language(s)
            of simple mathematical expressions admits only valid expressions?

      \item[\namedlabel{rq:four}{RQ4}] Our current ``typed'' approach to
            collecting different kinds of data is hard to extend. How can we
            make it easier to extend?

\end{enumerate}

\section{Contributions of the Author}
\label{sec:intro:contributions}

\intodo{\ref{rq:four} Link each sentence in the contributions section to the
      rqs.}

Dr. Jacques Carette built a prototype solution\todo{Cite appendix entry of
      code?. How should I cite what Dr. Carette had already written?} for the
issue of transcribed theories in Drasil not exposing enough information
for reliable code generation (as discussed in \Cref{chap:modelkinds}).
Relying on Dr. Jacques Carettes prototype, A current implementation and
other re-designs in the works have been developed by the author of this
work. Both issues discussed in \Cref{chap:modelkinds} and
\Cref{chap:typedExpr} were discussed before the author had joined the
Drasil research group. Drasil, and the expression language in particular,
had been constructed before this work.

\section{Thesis Outline}
\label{sec:intro:outline}

In \Cref{chap:ideology}, we discuss the focal ideology underpinning this work
and Drasil. \Cref{chap:drasil} describes Drasil, the host project carrying the
fruits of this work. \Cref{chap:modelkinds} discusses how theories are encoded
in Drasil, the issues associated with using a single universal mathematical
language to describe theories, and how we can resolve these problems.
\Cref{chap:typedExpr} describes residual issues associated with validating and
transcribing mathematical expressions. \Cref{chap:storingChunks} continues to
discuss the ways in which general knowledge and theories are encoded in Drasil,
and methods for altering their encoding, leading into \cref{chap:futureWork},
where we discuss future work.
