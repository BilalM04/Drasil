\chapter{Introduction}
\label{chap:introduction}

\begin{writingdirectives}

      \item \textit{Based on the content of a video by Dr. Cecile Badenhorst
            (\url{https://www.youtube.com/watch?v=c2oGY1c51jc}) and a post about
            writing introductions by UNSW:
            \url{https://www.student.unsw.edu.au/introductions}}

      \item Move 1: Establishing a research territory by:
      \begin{itemize}

            \item showing research area is important, interesting, and
                  incomplete

            \item reviewing previous research

      \end{itemize}

      \item Move 2: Establishing a niche by noting gaps in previous research.

      \item Move 3: Occupying the niche by:
      \begin{itemize}

            \item outlining purpose

            \item listing research questions

            \item announcing principal findings

            \item stating the value of the previous research

      \end{itemize}

      \item General Structure:
      \begin{itemize}

            \item Introduction:
                  \begin{itemize}

                        \item Jazzy information to get reader hooked

                        \item States purpose of chapter

                        \item Roadmap of what will be discussed in chapter

                  \end{itemize}

            \item Background: context of research problem, sets up the need for
                  research and relevance

            \item PPSQ: should be within first 3 pages of thesis, after intro +
                  background information.

            \item Research design and context: description of where the research
                  takes place (Drasil), introducing methodology briefly

            \item Assumptions, limitations, scope of research, and expected
                  outcomes: what do we need from this work

            \item Overview of chapters

      \end{itemize}

      \item Last Paragraph: summarize key points of chapter, link to next
      chapter

      \item What is the context of this research? What is it about?

      \item What problem does this research tackle?

      \item Why is the research problem important/significant?

      \item What previous research exists?

      \item What is the purpose of this research? What are the goals?

      \item What did the author contribute?
      \wqanswer{\Cref{sec:intro:contributions}}

      \item What does this thesis contain? \wqanswer{\Cref{sec:intro:outline}}

\end{writingdirectives}

When developing software, developers pull information from a common pool of
domain-specific knowledge, including the requirements and related fields (such
as physics and financial mathematics). The pool is common, but not necessarily
formally shared, leading to issues in synchronization, communication, and
understanding, all of which harm the final software produced in some way. As
knowledge changes, software artifacts are slow-moving to adjust, requiring
continued development. For well-understood domains, such as \ACF{scs}, where
there exists commonly agreed upon ``knowledge'' and software is somehow usable,
Drasil aims to capture domain-specific knowledge and use a generative approach
to build software as \textit{view} of the captured knowledge. Drasil is a suite
for \textit{generative software development}.

Currently focusing on \acs{scs}, Drasil uses a stable scientific knowledge-base
to generate families of \acs{scs} conforming to a \ACF{srs}
\cite{SmithAndLai2005}. By formalizing the language associated with an \acf{srs}
document (such as symbols, units, mathematical expressions, theories, etc.) and
how the language terms translate into other languages, Drasil is capable of
generating \acs{scs} conforming to a \acf{srs}, improving consistency,
traceability, understandability, maintainability, and reusability of the
\acf{scs} (but, more importantly, of the domain-specific knowledge as a whole,
as well).

Specifically, in this thesis, we will explore capturing mathematical theories
frequently pulled from in building \acs{scs}, typing mathematical expressions,
restricting mathematical expression usage to appropriate contexts, and scaling
Drasils knowledge database.

\intodo{\textit{From Dr. Carette}: The usual means of building software involves
      multiple artifacts (such as specifications and code) that contain
      duplicate information that is also supposed to be linked
      (\textit{traceability}). Drasil aims to use a generative approach to
      de-duplicate this information and make traceability more immediate. Drasil
      currently uses a stable scientific knowledge-base to generate families of
      \ACF{scs} conforming to a \ACF{srs}\cite{SmithAndLai2005}.}

\iffalse
      % Potential replacement for the above paragraph
      The usual means of building software involves multiple artifacts (such as
      specifications and code) that contain duplicate information that is
      supposed to be linked (\textit{traceability}). Drasil aims to capture the
      information and knowledge necessary to create these artifacts, and
      re-generate the artifacts, making traceability more immediate.

      % A less-specific introduction
      A common thread of knowledge links together all software artifacts.
      Software developers pull on this thread to produce software artifacts, but
      the thread is lost on the authors, isn't effectively shared, and artifacts
      quickly becomes out of date as more is understood or requirements changed.
      Typically, this knowledge is duplicated across many artifacts (such as
      specifications and code). Through codifying domain knowledge,
      Drasil\cite{Drasil2021} aims to use a generative approach to
      \textit{de-duplicate} this information, and generate software artifacts
      that are \textit{traceable}, \textit{reusable}, \textit{maintainable}, and
      \textit{consistent}, against a stable body of knowledge. Drasil currently
      focuses on \ACF{scs} conforming to a specific \ACF{srs}
      template\cite{SmithAndLai2005}.
\fi

\section{Background}
\label{sec:intro:background}

\footnotetext{``Generate All the Things'' is Drasils tagline.}

Intended to ``generate all the things''\footnotemark{}, Drasil
\footnote{\url{https://jacquescarette.github.io/Drasil/}} is a Haskell-based
\cite{Haskell2010} software suite studying how knowledge capture may improve
modern software development. ``Knowledge'' is considered ``captured'' in Drasil
by codifying it and its relations to other things using \ACFP{dsl} encoded as
Haskells data types. For example, ``quantities,'' as we've currently needed from
a physics-based point-of-view, are encoded as:

\intodo{Code snippet: QuantityDicts}

And an instance of a ``quantity'' might appear as:

\intodo{Code snippet: instance of a QuantityDict.}

The ways that we can translate ``quantities'' into other ``things'' are encoded
as Haskell-level functions and instances of typeclasses, such as:

\intodo{Code snippet: example of how QuantityDicts are converted into at least 1
      or 2 other things (e.g., SRS rows, symbols, etc.).}

Drasil is developed through a ``bottom-up'' methodology against several
\acs{scs} case studies (\refCaseStudiesTable{}), capturing and de-duplicating
knowledge as needed to re-generate the original artifacts, and, more in a wide
variety of similarly applicable languages (such as \acs{html}, \LaTeX{}, Java,
C/C\+\+, etc.).

For example, the \ACF{glassbr} case study (examining predicting whether a glass
slab can withstand an explosive blast) had software artifacts manually
written\footnote{\url{https://github.com/smiths/caseStudies/tree/master/CaseStudies/glass}}.
A coherent net of knowledge/discussion (a ``story'') is then formed by
dissecting the artifacts, understanding why each piece existed, what it relates
to and how, and how it can be translated into other things. The original
artifacts are then re-generated in a wider
variety\footnote{\url{https://github.com/JacquesCarette/Drasil/tree/master/code/stable/glassbr}}
of other languages by codifying how the net of knowledge can be translated into
other languages.

However, not all the case studies are capable of generating software artifacts
yet (\refCaseStudiesCodeTable{}). Each for their own reason, but we will focus
on a critical common denominator between them all: capturing mathematical
knowledge for reliable \acs{scs} artifact generation (and more).

\section{Problem Statement}
\label{sec:intro:problemStatement}

\iffalse
      Drasil has de-duplicated knowledge across \acs{scs} artifacts relevant to
      specifications and code. Through codifying knowledge and creating coherent
      ``stories'', Drasil is able to generate a wide variety of software
      artifacts (e.g., \acs{oo} programs via \acs{gool} with guided usage via
      Makefiles, and requirements specifications [HTML and TeX]). This codified
      knowledge was de-duplicated from an originating set of artifacts via
      bottom-up gathering, however, we should be able to use the same knowledge
      to generate more artifacts in different languages, flavours, and with more
      options. However, each desired artifact language has its own way of
      encoding information (such as mathematical expressions). This leaves us
      needing to teach Drasil more about the targeted languages (and, at times,
      about the existing codified knowledge) in order to reliably generate
      usable artifacts.
\fi

\begin{itemize}

      \item As Drasil focuses on generative software development of \acs{scs},
            scientific knowledge capture is at the forefront of its priority
            list because the preciseness and specificity of the \acsp{dsl} used
            is directly associated with the capabilities of the instances.

            Mathematical expression and theory encoding becomes a key point of
            interest for us because they are used in across the board (e.g.,
            derivations, code, constraints, definitions, etc.).

      \item As general-purpose programming languages are to domain-specific
            programming languages, a single universal mathematical language is
            to theory-specific language.

      \item Drasil relies on a single universal untyped mathematical language to
            describe general mathematical knowledge (theories).

            As a result, transforming encoded theories into other forms (such as
            code) is a complex task (similar to the complexity associated with
            transpiling a general-purpose program into another). The task is
            inflexible and arduous in nature because of a clear lack of
            information about the meaningful structure of the theories and how
            and when they are usable or valid.

      \item Additionally, as we encode more kinds of knowledge in Drasil, we
            face difficulties in scaling its knowledge database.

\end{itemize}

Continuing, as we encode more \textit{types} of information in Drasil, we face
difficulties in placing parameterized types into Drasils active knowledge
database and need it to scale.

\section{Research Questions}
\label{sec:intro:researchquestions}

\begin{enumerate}

      \item[\namedlabel{rq:one}{RQ1}] Drasil has a language of simple
            mathematical expressions that are used in multiple contexts. But not
            all expressions are valid in all contexts. How do we fix that?

      \item[\namedlabel{rq:two}{RQ2}] Drasil's current encoding of ``theories''
            are essentially black boxes. We would like to be able to use some
            structural information present in the short list of the ``kinds'' of
            theories that show up in scientific computing. How do we codify
            that?

      \item[\namedlabel{rq:three}{RQ3}] How can we ensure that our language(s)
            of simple mathematical expressions admits only valid expressions?

      \item[\namedlabel{rq:four}{RQ4}] Our current ``typed'' approach to
            collecting different kinds of data is hard to extend. How can we
            make it easier to extend?

\end{enumerate}

\section{Contributions of the Author}
\label{sec:intro:contributions}

In listed code snippets, I will refer to at least two major points of time in
relation to Drasils development (time measured by their git blob hash):
``current''\footnote{Blob hash: dc3674274edb00b1ae0d63e04ba03729e1dbc6f9} and
``original''\footnote{Blob hash: 9c26b43d3e30c3f618e534a3f176a5152729af74}. The
``original'' code refers to a code snippet as it was written before I was
onboarded to Drasil. The ``current'' code includes my work, at least, but might
also include the work of others who were also contributing to the project while
I was actively contributing. The work of others might include, but not limited
to, code formatting, code commenting, and extensions.

Drasil has existed since 2014, and has already seen success in its case studies,
which are used to guide the development of Drasil. Drasils focus on \acs{scs}
relies on knowledge of mathematical theories and language, for which Drasil has
a working understanding of before this work. However, some case studies were
unable to participate in code generation due to a lack of flexible theory
information (\ref{rq:two}), or just being inapplicable. This work contributes to
structuring theory information and allowing for future developers to encode more
kinds of theories and their relationships with other things (discussed in
\Cref{chap:modelkinds}). The solution builds on a prototype by Dr. Jacques
Carette\todo{Cite Dr. Carettes ModelKinds prototype.} that facilitates
structured theories to define relationships between ``code'' and ``theories.''

Theories rely on mathematical expressions as well. We commonly differ the usable
set of language in different contexts (you are free to write a lot more on your
pencil and paper derivations than on your typical calculator). To obtain
information about the expressibility in different contexts, we divide the
expression language using a \acs{ttf} \cite{Carette2009} encoding, with a
\acsp{gadt} backend for structural edits (\Cref{chap:modelkinds}). However,
``expressibility'' also relies on the expressions adhering to a precise
syntactic set of rules. As such, we build a system of typing rules for the
expression language (\Cref{chap:typedExpr}).

Finally, to enable capturing data with type parameters and generally scale
Drasils knowledge database (\ref{rq:four}), this work merges the typed database
collections into a single untyped, yet type-preserving, database (discussed in
\Cref{chap:storingChunks}).

\section{Thesis Outline}
\label{sec:intro:outline}

In \Cref{chap:ideology}, we discuss the focal ideology underpinning this work
and Drasil. \Cref{chap:drasil} describes Drasil, the host project carrying the
fruits of this work. \Cref{chap:modelkinds} discusses how theories are encoded
in Drasil (\ref{rq:two}), the issues associated with using a single universal
mathematical language to describe theories (\ref{rq:one}), and how we can
resolve these problems. \Cref{chap:typedExpr} describes issues associated with
the formation of mathematical expressions (\ref{rq:three}).
\Cref{chap:storingChunks} focuses on how Drasil stores information, and how it
can be scaled.
