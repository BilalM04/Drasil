\begin{writingdirectives}

      \item Move 1: Establishing a research territory by:
      \begin{itemize}

            \item showing research area is important, interesting, and
                  incomplete

            \item reviewing previous research

      \end{itemize}

      \item Move 2: Establishing a niche by noting gaps in previous research.

      \item Move 3: Occupying the niche by:
      \begin{itemize}

            \item outlining purpose

            \item listing research questions

            \item announcing principal findings

            \item stating the value of the previous research

      \end{itemize}

      \item General Movements:
      \begin{itemize}

            \item Introduction:
                  \begin{itemize}

                        \item Jazzy information to get reader hooked

                        \item States purpose of chapter

                        \item Roadmap of what will be discussed in chapter

                  \end{itemize}

            \item Background: context of research problem, sets up the need for
                  research and relevance

            \item PPSQ: should be within first 3 pages of thesis, after intro +
                  background information.

            \item Research design and context: description of where the research takes
                  place (Drasil), introducing methodology briefly

            \item Assumptions, limitations, scope of research, and expected outcomes:
                  what do we need from this work

            \item Overview of chapters

      \end{itemize}

      \item Last Paragraph: summarize key points of chapter, link to next chapter

      \item What is the context of this research? What is it about?

      \item What problem does this research tackle?

      \item Why is the research problem important/significant?

      \item What previous research exists?

      \item What is the purpose of this research? What are the goals?

      \item What did the author contribute?
      \wqanswer{\Cref{sec:intro:contributions}}

      \item What does this thesis contain? \wqanswer{\Cref{sec:intro:outline}}

\end{writingdirectives}


\section{Problem Statement}
\label{sec:intro:problemStatement}

In Drasil, we are focused on understanding families of scientific software, and
creating systematic rules to generate families of software solutions (for any
instance of a scientific problem that requires a scientific software solution).
Specifically, Drasil is focused on mathematics and physics-based models. In both
areas, we are concerned with what \textit{kinds of theories} are
well-understood, and ensuring that all created mathematical expressions are
``valid''.

\intodo{Re-write the above.}

\section{Contributions of the Author}
\label{sec:intro:contributions}

\begin{itemize}
      \item The problem of obtaining more information from our theories so that
            we may make better use from their meta-level information was
            understood before this work.
      \item A partial solution to this problem was also constructed before this
            work.
      \item The expression language used to build theories and, ultimately,
            inputs to outputs was constructed before this work.
\end{itemize}

\section{Thesis Outline}
\label{sec:intro:outline}

In \Cref{chap:ideology}, we discuss the focal ideology underpinning this work,
and Drasil. \Cref{chap:drasil} describes Drasil, the host project carrying the
fruits of this work. \Cref{chap:modelkinds} discusses how theories are encoded
in Drasil, the issues associated with using a single universal mathematical
language to describe theories, and how we can resolve these problems.
\Cref{chap:typedExpr} describes residual issues associated with validating and
transcribing mathematical expressions. \Cref{chap:knowledgeMgmt} continues to
discuss the ways in which general knowledge and theories are encoded in Drasil,
and methods for altering their encoding, leading into \cref{chap:futureWork},
where we discuss future work.
