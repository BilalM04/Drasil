\section{Problem Statement}

In Drasil, we are focused on understanding families of scientific software, and
creating systematic rules to generate families of software solutions (for any
instance of a scientific problem that requires a scientific software solution).
Specifically, Drasil is focused on mathematics and physics-based models. In both
areas, we are concerned with what \textit{kinds of theories} are
well-understood, and ensuring that all created mathematical expressions are
``valid''.

\intodo{Move above paragraph to the top of the ``Drasil'' section.}

\section{Contributions of the Author}

\begin{itemize}
      \item The problem of obtaining more information from our theories so that we
            may make better use from their meta-level information was understood
            before this work.
      \item A partial solution to this problem was also constructed before this
            work.
      \item The expression language used to build theories and, ultimately, inputs
            to outputs was constructed before this work.
\end{itemize}

\section{Thesis Outline}

In \autoref{chap:ideology}, we discuss the focal ideology underpinning this
work, and Drasil. \autoref{chap:drasil} describes Drasil, the host project
carrying the fruits of this work. \autoref{chap:modelkinds} discusses how
theories are encoded in Drasil, the issues associated with using a single
universal mathematical language to describe theories, and how we can resolve
these problems. \autoref{chap:typedExpr} describes residual issues associated
with validating and transcribing mathematical expressions.
\autoref{chap:knowledgeMgmt} continues to discuss the ways in which general
knowledge and theories are encoded in Drasil, and methods for altering their
encoding, leading into \autoref{chap:futureWork}, where we discuss future work.
