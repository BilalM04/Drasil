\chapter{Introduction}
\label{chap:introduction}

\begin{writingdirectives}

      \item \textit{Based on the content of a video by Dr.\ Cecile Badenhorst
            (\url{https://www.youtube.com/watch?v=c2oGY1c51jc}) and a post about
            writing introductions by UNSW:
            \url{https://www.student.unsw.edu.au/introductions}}

      \item Move 1: Establishing a research territory by:
      \begin{itemize}

            \item showing research area is important, interesting, and
                  incomplete

            \item reviewing previous research

      \end{itemize}

      \item Move 2: Establishing a niche by noting gaps in previous research.

      \item Move 3: Occupying the niche by:
      \begin{itemize}

            \item outlining purpose

            \item listing research questions

            \item announcing principal findings

            \item stating the value of the previous research

      \end{itemize}

      \item General Structure:
      \begin{itemize}

            \item Introduction:
                  \begin{itemize}

                        \item Jazzy information to get reader hooked

                        \item States purpose of chapter

                        \item Roadmap of what will be discussed in chapter

                  \end{itemize}

            \item Background: context of research problem, sets up the need for
                  research and relevance

            \item PPSQ: should be within first 3 pages of thesis, after intro +
                  background information.

            \item Research design and context: description of where the research
                  takes place (Drasil), introducing methodology briefly

            \item Assumptions, limitations, scope of research, and expected
                  outcomes: what do we need from this work

            \item Overview of chapters

      \end{itemize}

      \item Last Paragraph: summarize key points of chapter, link to next
      chapter

      \item What is the context of this research? What is it about?

      \item What problem does this research tackle?

      \item Why is the research problem important/significant?

      \item What previous research exists?

      \item What is the purpose of this research? What are the goals?

      \item What did the author contribute?
      \wqanswer{\Cref{chap:intro:sec:contributions}}

      \item What does this thesis contain? \wqanswer{\Cref{chap:intro:sec:thesis-outline}}

\end{writingdirectives}

Software developers pull on their understanding of problems to build software
solutions, and working together, developers similarly pull their understanding
from a shared pool of knowledge. Often, developers share their knowledge through
documentation, keeping every product owner\footnote{For our purposes, defining a
``product owner'' as all people that have any responsibility in the development
of a software artifact.} ``in the loop.'' As knowledge and requirements change,
software implementations trail behind the current needs and understandings,
until developers manually update the code. By applying generative techniques to
software development, we may have some, or \textit{all}, of the manual updates
performed for us.

Drasil \cite{Drasil2021} is an exploration of applying generative techniques to
well-understood domains to create an alternative way of developing software.
Focused on building \ACF{scs}, Drasil allows users to describe scientific
problems using an abstracted \ACF{srs} template to automatically build a
software solution conforming to the specifications outlined. However, Drasil is
only capable of generating software for the problems that it has sufficiently
``understood'' (i.e., ones that have been sufficiently dissected and encoded in
Drasil).

With a focus on scientific theories and mathematical expressions, in this
thesis, we aim to make some improvements to how Drasil captures certain kinds of
knowledge. However, we will also make improvements to how Drasil stores
knowledge in general.

\section{Background: Drasil}
\label{chap:introduction:sec:background}

\porthref{Drasil}{https://jacquescarette.github.io/Drasil/} is a software suite
for generating software from well-formed, principled ``stories''\footnote{For
our purposes, ``stories'' being an abstraction of the requirements of the
generated software.}. Focused on \ACF{scs}, Drasil allows users to ``fill in the
blanks'' to describe their scientific problems using a precise \ACF{srs} format
\cite{SmithAndLai2005}. By providing sufficient information in the ``blanks,''
Drasil is able to use the information to generate various software artifacts,
including whole programs (in various supported languages, such as Java and C\#),
build tools, and documentation. The ``blanks'' are holes for
\textit{domain-specific knowledge} and are filled in using one of many
\ACFP{dsl}. Each individual fragment of knowledge in Drasil is known as a
``chunk,'' and we encode each useful idea necessary to discuss/build our desired
software artifacts as chunks. For example, if we wanted to encode a variable,
\(\theta{}_{c}\), representing the firing angle of a cannon\footnote{Running
example based on Drasil's \acs{projectile} case study.}, we might write:

\exampleAngleVariableEncoding{}

\refExampleAngleVariableEncoding{} encodes the variable with a \acs{uid}
(``desiredFiringAngle''), a short name (``desired firing angle''), long name
(``needed firing angle for a cannon to hit a target''), a symbol
(\(\theta{}_{c}\)), and a unit (radians). The many used chunks and \acsp{dsl}
make up a \textit{network of domains} \cite{Czarnecki2005}, which allow Drasil
to make domain-specific transformations, such as the one most desired in Drasil:
generating computation software conforming to a precise \acs{srs}.

\roughNetworkOfDomainsIntro{}

Roughly, \refRoughNetworkOfDomainsIntro{} shows how Drasil's \textit{Smith et
      al.} knowledge transformer works, with each node representing a domain of
knowledge, and each arrow representing a mapping between them. Drasil users
mostly enter in their scientific knowledge near the ``top'' of the diagram to
form a coherent \acs{srs} abstraction using relevant \acsp{dsl}. Drasil takes
their \acs{srs} abstraction, audits it, and allows the user to pick from a
series of options to generate software that conforming to the scientific problem
encoded.

The ``scientific knowledge'' at the top of the diagram is not necessarily a
complete capture of all scientific knowledge, of course. Rather, it is a
``bubble''/collection of the scientific knowledge. Drasil's development follows
the needs of a series of manually built case studies, using them as ``seedling''
data to navigate development. As Drasil is ``taught'' more
information\footnote{By providing it with a means of discussing relevant ideas
      through \acsp{dsl}.}, Drasil's span of producible artifacts widens.
Unfortunately, at the moment, Drasil is not able to generate code for all of its
case studies.

Some case studies are currently in-progress (i.e., incomplete) and won't be a
focus of this work, but left for future work by others. Instead, we will focus
on those which \textit{we} understand how to manually produce code for.
Specifically, we will largely focus on a common denominator: strengthening
mathematical knowledge capture, but we will also spend time learning how to
scale Drasil's knowledge (chunk) database against expansion in further chunk
type creation.

\subsection{Theories and Expressions}
\label{chap:intro:sec:drasil:subsec:theories-and-expressions}

One of the most important aspects of understanding and describing scientific
problems is the underlying \textit{theories}. Drasil relies on users describing
theory knowledge using \textit{relations} through a \textit{single universal
      untyped mathematical expression language}. To be clear, this means that Drasil
relies on on-demand interpretation and comprehension of \textit{terms} from this
expression language. Equations, relations, derivations, constraints, and
definitions are all described using this single language. However, the language
does not contain sufficient ``depth'' to adequately make use of its encoded
information. In other words, this expression language is a ``lower-level'' view
of the information we really need to make domain-specific transformations.

Drasil attempts to use the \textit{theories} gathered from a user-filled
\acs{srs} to understand what problem they intended to describe (i.e.,
calculating the inputs and outputs, and a sequence of calculation steps that go
from the former to the latter). Each theory needs to be sufficiently understood
to produce calculation steps that we can use. For example, if we wanted to
define our firing angle variable (\refExampleAngleVariableEncoding{}) in terms
of the distance of the target, gravity, and the initial
velocity\footnote{Assuming no air drag or resistance.}, we might write:
\(\arcsin{}(\mathit{targetDistanceFromCannon} \cdot{} g / v^2) / 2\). In Drasil,
we might encode this as \refExampleAngleEquationEncoding{}.

\exampleAngleEquationEncoding{}

With this theory encoding, we have enough information to generate a visually
pleasing description box for the \acs{srs} \LaTeX{} and \acs{html} documents
(\refExampleAngleEquationEncodingToTeX{}). Converting to typesetting languages
is reasonably unproblematic because they only use the theory knowledge in a
shallow light (displaying) and we can create any glyph we might need. More
importantly, we can also convert this theory to Python code as well:

\exampleAngleEquationEncodingToPython{}

However, this example is reasonably simple. If we were handed an equation of
this \(\nu{} = f(x,y,z)\) form, it's somewhat reasonable, but perhaps dubious
to assume that one way to define \(\nu{}\) is \(f(x,y,z)\). However, if we were
handed an equivalent, implicit version of it, then we can assume less
information about it. For example, if we re-wrote the equation for
\refExampleAngleEquationEncoding{} as \(\mathit{targetDistanceFromCannon}
\cdot{} g = v^2\sin{}(2\theta{}_c)\), then we've lost the critical
\textit{definition} assumption that we used to write the python code.
Automatically isolating for \(\theta_c\) is possible, but we're interested in
the definitional information, and we want it as readily-available as possible.
Now, with this scrambled variant as well, we similarly can't typeset it in the
expected definitional form we desired, even though the relation is equivalent.
With more complex theories, more issues arise. For example, defining theories
with multiple definitions lacks information about which definition to ``pick''
for software implementations, \acsp{ode} lack information about solving methods,
and some theories may be purely abstract and entirely unusable for code
generation. In other words, the current encoding of theories lacks information
about the structure of the theories. Additionally, we also lack information
about when expression terms are usable in the context of code (e.g., expressions
involving derivatives are not always directly usable in code), where we expect
them to be directly translatable to programming languages. As a result, knowing
\textit{how to} and \textit{when you can} transform captured theories into other
forms (such as executable code) is a complex task (similar to the complexity
associated with transpiling a general-purpose program into another \textemdash{}
an exercise in futility!).

With the theory encoding above, we also never discussed what
\(\mathit{targetDistanceFromCannon}\), \(g\), nor \(v\) were. On paper and
pencil, it's fairly reasonable to assume that they are all some numeric type.
However, when we generate software, we expect it to be \textit{executed}, and
hence undergo a type-checking phase by compilers/interpreters. As the expression
language is currently untyped, Drasil currently allows software generation with
``ill-typed'' (according to the respective compilers and interpreters)
expressions. In particular, for example, this poses problem for Swift code
generation because Swift doesn't allow numeric addition if both operants are not
of the same type, while Java uses implicit type coercion to mask these issues.
Ultimately, this means that more time is spent developing and debugging
generated artifacts than we would like. We would prefer that generated software
come with an assurance of correctness.

\subsection{Capturing (\& Remembering) Everything}
\label{chap:intro:sec:drasil:subsec:capturing-and-remembering-everything}

Knowledge capture is at the heart of Drasil. It is what allows Drasil to make
domain-specific interpretations that we're interested in, to ultimately generate
software artifacts. \textit{Chunks} are unique instances of knowledge, tagged
with a \acs{uid} for referencing them, and a \textit{type} for understanding
their structure. The chunk is unique with respect to a global ``chunk database''
that Drasil registers all chunks into. At the moment, Drasil's chunk database
uses a series of typed maps (\refOriginalChunkDBHaskell{}) with \acs{uid} keys
and typed chunk values. This means that as we create new types of chunks in
Drasil, the list of maps in the database will grow. Additionally, the list of
types will have to be known before using Drasil, which means that extending the
list of chunks is difficult for users. In other words, the current chunk
database is not \textit{extensible} because of its typed nature.

\section{Problem Statement}
\label{chap:intro:sec:problem-statement}

\begin{writingdirectives}
      \item Key problems:
      \begin{itemize}
            \item theory structural knowledge
            \item scope of the expression language and information about theory usage
            \item expression language is untyped
            \item scaling against new types
      \end{itemize}
\end{writingdirectives}

Drasil isn't able to adequately make domain-specific interpretations of encoded
theories because it lacks structural information about them. They also don't
provide static information about the contexts in which they are usable,
partially because the underlying expression language exposes little information
about this too. Additionally, Drasil's lack of type information stops us from
reliably generating usable software (in particular, there is no assurance of
type checking). Together, these issues affect our ability to reliably and
flexibly generate software. Finally, Drasil's chunk database is not extensible,
limiting admissible chunk types to only those that exist in core Drasil.\\

\noindent\textbf{Research Questions}:

\begin{enumerate}

      \item[\namedlabel{rq:modelkinds}{RQ1}] Drasil's current encoding of
            ``theories'' are essentially black boxes. We would like to use
            structural information present in the short list of the ``kinds'' of
            theories that show up in scientific computing. How do we codify
            that?

      \item[\namedlabel{rq:lang_division}{RQ2}] Drasil's theory encodings rely
            on a single mathematical expression language, which does not expose
            information about applicability to different contexts. In each
            context (e.g., code, theories, and common arithmetic), certain terms
            of the expression language should be treated differently or are
            simply inapplicable. How can we restrict term usage by context?

      \item[\namedlabel{rq:typing}{RQ3}] How can we ensure that our mathematical
            expression language admits only valid expressions?

      \item[\namedlabel{rq:chunkdb}{RQ4}] Our current ``typed'' approach to
            collecting different kinds of data is difficult to extend. How can
            we make it easier to extend?

\end{enumerate}

\section{Contributions of the Author}
\label{chap:intro:sec:contributions}

In listed code snippets with ``Original'' in their titles, I'm referring to code
that existed in full before I started contributing to Drasil. As such, I claim
no authorship in those snippets. Unless otherwise stated, all other snippets,
excluding excerpts from generated artifact snippets, includes some of my work,
but might also include the work of others who were also contributing to the
project while I was also actively contributing. The work of others might
include, but is not limited to, code formatting, code commenting, and
extensions.

My notable contributions include:

\begin{itemize}

      \item Implementing \ModelKinds{} to expose theory structure knowledge,
            creating opportunity for new domain-specific interpretation of
            theory knowledge (resolving \ref{rq:modelkinds}). The solution
            builds on a prototype by Dr.\ Jacques Carette\footnote{Unfortunately,
                  the code associated with the prototype had been deleted. However,
                  \refOriginalNewModelKindsHaskell{} is, if memory serves me
                  correctly, nearly identical.}.

      \item Splitting the mathematical expression language into 3, each with
            their own intended area of usage, but using a \acs{ttf}
            \cite{Carette2009} to keep same user-friendliness in expression
            transcription (resolving \ref{rq:lang_division}).

      \item Adding type checking and inference to mathematical expression
            languages (resolving \ref{rq:typing}).

      \item Creating an extensible chunk database prototype that can register
            chunks with types external to core Drasil and chunks with type
            parameters (resolving \ref{rq:chunkdb}).

\end{itemize}

\section{Thesis Outline}
\label{chap:intro:sec:thesis-outline}

In \Cref{chap:ideology}, we discuss what we can learn from the source code of
general software, and specifically, how we can abstract over them to form
similar software, varying over certain aspects of the software to make the
background knowledge reusable. \Cref{chap:drasil} discusses Drasil, a software
suite for generating software from coherent descriptions of scientific problems.
\Cref{chap:modelkinds} discusses how Drasil captures and uses theories to
generate code, and how we improved theory capture to create more opportunity for
domain-specific interpretation of the theories (\ref{rq:modelkinds})
\Cref{chap:lang-division} discusses how Drasil encodes mathematical expressions
and associated issues (\ref{rq:lang_division}). \Cref{chap:more-theory-kinds}
revisits the existing captured theories in Drasil, rebuilding them with
specialized encodings for others to use as needed. \Cref{chap:typed-expr}
discusses issues associated with the formation of mathematical expressions and
what it means for expressions to be ``well-typed'' (\ref{rq:typing}).
\Cref{chap:storingChunks} focuses on how Drasil stores information, and how it
can scale against future development of Drasil and libraries (\ref{rq:chunkdb}).
\Cref{chap:future-work} discusses future work. Finally, \Cref{chap:conclusion}
sums up the achievements of this work.
