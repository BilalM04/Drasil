\chapter{Typing the Expression Language}
\label{chap:typedExpr}

\section{Background: Problem}

\begin{itemize}

      \item Writing invalid expressions is possible.
            \begin{itemize}

                  \item On paper, writing invalid expressions is as easy as
                        making a typo, but complete gibberish can also be
                        written. We rely on manually checking expressions to
                        ensure that they are ``correct''. As the number of
                        expressions grows, the cost of manually checking grows
                        rapidly, and changes result in costly setbacks. Imagine
                        systems with 10, 100, and 1000+ expressions, the cost
                        grows rapidly.

                  \item With computers, we can systematically check the validity
                        of expressions by imposing various kinds of
                        restrictions.

            \end{itemize}

      \item Mentally tracking expression creations to ensure they follow the
            implicit rules of the expression language is too difficult, and
            leads to mental strain.

      \item Compiling to ``lower languages'' requires special type checking
            before compiling to them. For example, the Swift code generator has
            to ensure that there are no ambiguously typed numerals as the types
            of numerics are not overloaded in Swift.

      \item Dynamically checking for invalid expression states is possible, but
            difficult and would result in increasingly difficult term tracking
            as terms in the expression language grow/are added.

      \item In general, being able to express invalid expressions causes large
            burden and mental overhead.

\end{itemize}


\section{Requirements \& properties of a good solution}

\begin{itemize}

      \item Invalid expressions should not be representable in the various
            expression languages (i.e., the expression types should strictly
            indicate valid expression constructions), without loss of
            generality.

      \item Invalid expression formation attempts should be statically found and
            reported by the compiler, at compile-time. This will move the
            previously runtime errors to compile-time.

      \item Invalid expression cases should not need to be considered when
            working (e.g., case-ing) with expressions.

      \item ``Safety = Preservation + Progress'' (\cite{Harper2016}, Ch.6)

\end{itemize}

\section{Solution}

\begin{itemize}

      \item Use TTF encodings of the smart constructors to lessen the cognitive
            load of handling at least 3 different expression languages.

      \item Statically type all 3 variants of Expr through GADTs.

\end{itemize}

\subsection{Syntax}

\subsubsection{Current}

An idealized version of the current syntax.

\startSyntaxTable
    \newsyntaxRow{Type}{\tau}{Integer}{\bb{Z}}{Integer numbers}
    \syntaxRow{Real}{\bb{R}}{Real numbers}
    \syntaxRow{String}{String}{Text}
    \syntaxRow{Bool}{\bb{B}}{Truth values (true/false)*}
    \syntaxRow{Vector($\tau$, $n$)}{[\tau]_n}
        {Vectors (single element type, fixed length)*}
    \syntaxRow{Tuple($\tau_1...\tau_n$)}
        {\tau_1 \times \tau_2 \times ... \times \tau_n}
        {Alternative vectors/tuples (fixed length+differently typed elements)*}
     \\

    \newsyntaxRow{Literal}{l}{Integer[$n$]}{n}{Integer number}
    \syntaxRow{Real[$r$]}{r}{Real number}
    \syntaxRow{String[$s$]}{``s''}{Text}
    \syntaxRow{Bool[$b$]}{b}{Boolean value}
    \syntaxRow{Vector($l_1...l_n$)}{<l_1, ..., l_n>}{Vectors*}
    \syntaxRow{Tuple($l_1...l_n$)}{(l_1, ..., l_n)}{Tuples*}
     \\

    \newsyntaxRow{UnaryOp}{\ominus}{Not}{\lnot \_}{Logical negation}
    \syntaxRow{Neg}{- \_}{Numeric negation}
    \syntaxRow{...}{}{\tiny{omitted for brevity}}
     \\

    \newsyntaxRow{BinaryOp}{\oplus}{Sub}{\_ - \_}{Subtraction}
    \syntaxRow{Pow}{\_^\_}{Powers}
    \syntaxRow{...}{}{\tiny{omitted for brevity}}
     \\
    
    \newsyntaxRow{AssocBinOp}{\otimes}{Add}{\_ + \_}{Addition}
    \syntaxRow{Mul}{\_ \times \_}{Multiplication}
    \syntaxRow{...}{}{\tiny{omitted for brevity}}
     \\

    \newsyntaxRow{UID}{u}{UID(s)}{\texttt{UID ``s''}}{UIDs}
     \\

    \newsyntaxRow{Expr}{e}{Literal($l$)}{l}{Literal values}
    \syntaxRow{Vector($e_1...e_n$)}{<e_1, ..., e_n>}{Vectors}
    \syntaxRow{Var($u$)}{u}{Variable (QuantityDict Chunk)}
    \syntaxRow{FuncCall($f, e_1...e_n$)}{f(e_1...e_n)}
        {``Complete'' function application}
    \syntaxRow{UnaryOp($\ominus$,$e$)}{\ominus\ e}{Unary operations}
    \syntaxRow{BinaryOp($\oplus$,$e_1$,$e_2$)}
        {e_1 \oplus e_2}{Binary operations}
    \syntaxRow{AssocOp($\otimes$, $e_1...e_n$)}{e_1 \otimes ... \otimes e_n}
        {Associative binary operations}
    \syntaxRow{Case($e_{1c}e_{1e}...e_{nc}e_{ne}$)}
        {if\ e_{1c}\ then\ e_{2e}\ elif\ e_{2c}\ ...}
        {If-then-else-if-then-else (Switch-like statements)}
    \syntaxRow{BigAsBinOp($\otimes, e_1, e_2$)}{\bigotimes_{i=e_1}^{e_2} i}
        {Apply a ``big'' op to a discrete range defined as a range}
    \syntaxRow{IsInRlItrvl($u, e_1, e_2$)}{u \in [e_1, e_2]}{Variable in range}

\closeSyntaxTable

    $*$: does not currently appear in the code at the moment, but would be
    needed/desired


\subsection{Typing Rules}

\subsubsection{Literal}

\begin{enumerate}

    \item \[ \infer{\ofTy{Integer[i]}{Literal Integer}}{\ofTy{i}{Integer}} \]
    \item \[ \infer{\ofTy{Str[s]}{Literal String}}{\ofTy{s}{String}} \]
    \item \[ \infer{\ofTy{Dbl[d]}{Literal Real}}{\ofTy{d}{Double}} \]
    \item \[ \infer{\ofTy{ExactDbl[d]}{Literal Real}}{\ofTy{d}{Integer}} \]
    \item \[ \infer{\ofTy{Perc[n,d]}{Literal Real}}{\ofTy{n}{Integer} & \ofTy{d}{Integer}} \]

\end{enumerate}


\subsubsection{Miscellaneous}


\begin{enumerate}

    \item Completeness:
        \newrule{}
            {\ofTy{Complete[]}{Completeness}}
        
        \newrule{}
            {\ofTy{Incomplete[]}{Completeness}}

    \item AssocOp:
        \begin{enumerate}
            \item Numerics:
                \newrule{\ofTy{x}{Numerics($\tau$)}}
                    {\ofTy{Add[]}{AssocOp x}}
        
                \newrule{\ofTy{x}{Numerics($\tau$)}}
                    {\ofTy{Mul[]}{AssocOp x}}
    
            \item Bool:
                \newrule{}
                    {\ofTy{And[]}{AssocOp Bool}}
        
                \newrule{}
                    {\ofTy{Or[]}{AssocOp Bool}}
        \end{enumerate}

    \item UnaryOp:
        \begin{enumerate}
            \item Numerics:
                \newrule{\ofTy{x}{NumericsWithNegation(x)}}
                    {\ofTy{Neg[]}{UnaryOp x x}}

                \newrule{\ofTy{x}{NumericsWithNegation(x)}}
                    {\ofTy{Abs[]}{UnaryOp x x}}
                
                % TODO: Exp
                
                For Log, Ln, Sin, Cos, Tan, Sec, Csc, Cot, Arcsin, Arccos, Arctan, and Sqrt, please use the following template, replacing ``$\$TRG$'' with the desired operator:
                \newrule{}
                    {\ofTy{\$TRG[]}{UnaryOp Real Real}}

            \item Vectors:
                \newrule{\ofTy{x}{NumericsWithNegation(x)}}
                    {\ofTy{NegV[]}{UnaryOp [x] [x]}}

                \newrule{\ofTy{x}{Numerics(x)}}
                    {\ofTy{Norm[]}{UnaryOp [x] Real}}

                \newrule{\ty{x}}
                    {\ofTy{Dim[]}{UnaryOp [x] Integer}}
                
            \item Booleans:
                \newrule{}
                    {\ofTy{Not[]}{UnaryOp Bool Bool}}

        \end{enumerate}

    \item BinaryOp:
        \begin{enumerate}
            \item Arithmetic: % TODO: 
            \item Bool: % TODO: 
            \item Equality: % TODO: 
            \item Ordering: % TODO: 
            \item Indexing: % TODO: 
            \item Vectors: % TODO: 
        \end{enumerate}
    
    \item RTopology:
        \newrule{}
            {\ofTy{Discrete[]}{RTopology}}

        \newrule{}
            {\ofTy{Continuous[]}{RTopology}}
    
    \item DomainDesc: % TODO: Why does the topology appear as a type constructor argument, and type signature argument?
        \newrule{\ofTy{top}{$\tau_1$} & \ofTy{bot}{$\tau_2$} & \ofTy{s}{Symbol} & \ofTy{rtop}{RTopology}}
            {\ofTy{BoundedDD[s, rtop, top, bot]}{DomainDesc Discrete $\tau_1$ $\tau_2$}}

        \newrule{\ty{topT} & \ty{botT} & \ofTy{s}{Symbol} & \ofTy{rtop}{RTopology}}
            {\ofTy{AllDD[s, rtop]}{DomainDesc Continuous topT botT}}

    \item Inclusive:
        \newrule{}
            {\ofTy{Inc[]}{Inclusive}}

        \newrule{}
            {\ofTy{Exc[]}{Inclusive}}

    \item RealInterval:
        \newrule{\ty{a} & \ty{b} & \ofTy{top}{(Inclusive, a)} & \ofTy{bot}{(Inclusive, b)}}
            {\ofTy{Bounded[top, bot]}{RealInterval a b}}

        \newrule{\ty{a} & \ty{b} & \ofTy{top}{(Inclusive, a)}}
            {\ofTy{UpTo[top]}{RealInterval a b}}

        \newrule{\ty{a} & \ty{b} & \ofTy{bot}{(Inclusive, b)}}
            {\ofTy{UpFrom[bot]}{RealInterval a b}}

\end{enumerate}


\subsubsection{Expr}

\begin{haskell}{Current Expression Language}{curExpr}{https://github.com/JacquesCarette/Drasil/blob/dc3674274edb00b1ae0d63e04ba03729e1db\newline{}c6f9/code/drasil-lang/lib/Language/Drasil/Expr/Lang.hs\#L81-L135}{https://github.com/JacquesCarette/Drasil/blob/dc3674274edb00b1ae0d63e04ba03729e1dbc6f9/code/drasil-lang/lib/Language/Drasil/Expr/Lang.hs\#L81-L135}
-- | Expression language where all terms are supposed to be 'well understood'
--   (i.e., have a definite meaning). Right now, this coincides with
--   "having a definite value", but should not be restricted to that.
data Expr where
  -- | Brings a literal into the expression language.
  Lit :: Literal -> Expr
  -- | Takes an associative arithmetic operator with a list of expressions.
  AssocA   :: AssocArithOper -> [Expr] -> Expr
  -- | Takes an associative boolean operator with a list of expressions.
  AssocB   :: AssocBoolOper  -> [Expr] -> Expr
  -- | C stands for "Chunk", for referring to a chunk in an expression.
  --   Implicitly assumes that the chunk has a symbol.
  C        :: UID -> Expr
  -- | A function call accepts a list of parameters and a list of named parameters.
  --   For example
  --
  --   * F(x) is (FCall F [x] []).
  --   * F(x,y) would be (FCall F [x,y]).
  --   * F(x,n=y) would be (FCall F [x] [(n,y)]).
  FCall    :: UID -> [Expr] -> [(UID, Expr)] -> Expr
  -- | For multi-case expressions, each pair represents one case.
  Case     :: Completeness -> [(Expr, Relation)] -> Expr
  -- | Represents a matrix of expressions.
  Matrix   :: [[Expr]] -> Expr
  -- | Unary operation for most functions (eg. sin, cos, log, etc.).
  UnaryOp       :: UFunc -> Expr -> Expr
  -- | Unary operation for @Bool -> Bool@ operations.
  UnaryOpB      :: UFuncB -> Expr -> Expr
  -- | Unary operation for @Vector -> Vector@ operations.
  UnaryOpVV     :: UFuncVV -> Expr -> Expr
  -- | Unary operation for @Vector -> Number@ operations.
  UnaryOpVN     :: UFuncVN -> Expr -> Expr
  -- | Binary operator for arithmetic between expressions (fractional, power, and subtraction).
  ArithBinaryOp :: ArithBinOp -> Expr -> Expr -> Expr
  -- | Binary operator for boolean operators (implies, iff).
  BoolBinaryOp  :: BoolBinOp -> Expr -> Expr -> Expr
  -- | Binary operator for equality between expressions.
  EqBinaryOp    :: EqBinOp -> Expr -> Expr -> Expr
  -- | Binary operator for indexing two expressions.
  LABinaryOp    :: LABinOp -> Expr -> Expr -> Expr
  -- | Binary operator for ordering expressions (less than, greater than, etc.).
  OrdBinaryOp   :: OrdBinOp -> Expr -> Expr -> Expr
  -- | Binary operator for @Vector x Vector -> Vector@ operations (cross product).
  VVVBinaryOp   :: VVVBinOp -> Expr -> Expr -> Expr
  -- | Binary operator for @Vector x Vector -> Number@ operations (dot product).
  VVNBinaryOp   :: VVNBinOp -> Expr -> Expr -> Expr
  -- | Operators are generalized arithmetic operators over a 'DomainDesc'
  --   of an 'Expr'.  Could be called BigOp.
  --   ex: Summation is represented via 'Add' over a discrete domain.
  Operator :: AssocArithOper -> DiscreteDomainDesc Expr Expr -> Expr -> Expr
  -- | A different kind of 'IsIn'. A 'UID' is an element of an interval.
  RealI    :: UID -> RealInterval Expr Expr -> Expr
\end{haskell}


\subsubsection{ModelExpr}

\begin{haskell}{Current ModelExpr Language}{curModelExpr}{https://github.com/JacquesCarette/Drasil/blob/ab9e091dabd81685ddef86b0d218582c9f75cb20/code/drasil-lang/lib/Language/Drasil/ModelExpr/Lang.hs\#L82-L151}
-- | Expression language where all terms are supposed to have a meaning, but
--   that meaning may not be that of a definite value. For example,
--   specification expressions, especially with quantifiers, belong here.
data ModelExpr where
  -- | Brings a literal into the expression language.
  Lit       :: Literal -> ModelExpr
  
  -- | Introduce Space values into the expression language.
  Spc       :: Space -> ModelExpr
  
  -- | Takes an associative arithmetic operator with a list of expressions.
  AssocA    :: AssocArithOper -> [ModelExpr] -> ModelExpr
  -- | Takes an associative boolean operator with a list of expressions.
  AssocB    :: AssocBoolOper  -> [ModelExpr] -> ModelExpr
  -- | Derivative syntax is:
  --   Type ('Part'ial or 'Total') -> principal part of change -> with respect to
  --   For example: Deriv Part y x1 would be (dy/dx1).
  Deriv     :: Integer -> DerivType -> ModelExpr -> UID -> ModelExpr
  -- | C stands for "Chunk", for referring to a chunk in an expression.
  --   Implicitly assumes that the chunk has a symbol.
  C         :: UID -> ModelExpr
  -- | A function call accepts a list of parameters and a list of named parameters.
  --   For example
  --
  --   * F(x) is (FCall F [x] []).
  --   * F(x,y) would be (FCall F [x,y]).
  --   * F(x,n=y) would be (FCall F [x] [(n,y)]).
  FCall     :: UID -> [ModelExpr] -> [(UID, ModelExpr)] -> ModelExpr
  -- | For multi-case expressions, each pair represents one case.
  Case      :: Completeness -> [(ModelExpr, ModelExpr)] -> ModelExpr
  -- | Represents a matrix of expressions.
  Matrix    :: [[ModelExpr]] -> ModelExpr
  
  -- | Unary operation for most functions (eg. sin, cos, log, etc.).
  UnaryOp       :: UFunc -> ModelExpr -> ModelExpr
  -- | Unary operation for @Bool -> Bool@ operations.
  UnaryOpB      :: UFuncB -> ModelExpr -> ModelExpr
  -- | Unary operation for @Vector -> Vector@ operations.
  UnaryOpVV     :: UFuncVV -> ModelExpr -> ModelExpr
  -- | Unary operation for @Vector -> Number@ operations.
  UnaryOpVN     :: UFuncVN -> ModelExpr -> ModelExpr
  
  
  -- | Binary operator for arithmetic between expressions (fractional, power, and subtraction).
  ArithBinaryOp :: ArithBinOp -> ModelExpr -> ModelExpr -> ModelExpr
  -- | Binary operator for boolean operators (implies, iff).
  BoolBinaryOp  :: BoolBinOp -> ModelExpr -> ModelExpr -> ModelExpr
  -- | Binary operator for equality between expressions.
  EqBinaryOp    :: EqBinOp -> ModelExpr -> ModelExpr -> ModelExpr
  -- | Binary operator for indexing two expressions.
  LABinaryOp    :: LABinOp -> ModelExpr -> ModelExpr -> ModelExpr
  -- | Binary operator for ordering expressions (less than, greater than, etc.).
  OrdBinaryOp   :: OrdBinOp -> ModelExpr -> ModelExpr -> ModelExpr
  -- | Space-related binary operations.
  SpaceBinaryOp :: SpaceBinOp -> ModelExpr -> ModelExpr -> ModelExpr
  -- | Statement-related binary operations.
  StatBinaryOp  :: StatBinOp -> ModelExpr -> ModelExpr -> ModelExpr
  -- | Binary operator for @Vector x Vector -> Vector@ operations (cross product).
  VVVBinaryOp   :: VVVBinOp -> ModelExpr -> ModelExpr -> ModelExpr
  -- | Binary operator for @Vector x Vector -> Number@ operations (dot product).
  VVNBinaryOp   :: VVNBinOp -> ModelExpr -> ModelExpr -> ModelExpr
  
  
  -- | Operators are generalized arithmetic operators over a 'DomainDesc'
  --   of an 'Expr'.  Could be called BigOp.
  --   ex: Summation is represented via 'Add' over a discrete domain.
  Operator :: AssocArithOper -> DomainDesc t ModelExpr ModelExpr -> ModelExpr -> ModelExpr
  -- | A different kind of 'IsIn'. A 'UID' is an element of an interval.
  RealI    :: UID -> RealInterval ModelExpr ModelExpr -> ModelExpr
  
  -- | Universal quantification
  ForAll   :: UID -> Space -> ModelExpr -> ModelExpr
\end{haskell}


\subsubsection{CodeExpr}

\begin{enumerate}

    \item \[ \infer{A}{B & C} \]

\end{enumerate}

