\chapter{Typing the Expression Language}
\label{chap:typedExpr}

Mathematical expressions are one of the most prominent components of any
abstracted concept from scientific software artifacts. With a pencil and paper,
our mistakes might go unnoticed, because they are never ``hard'' validated by
any machine, but ``soft'' validated by us and other readers. In other words,
there is no clear \textit{validity assertion} when we traditionally write
expressions on paper. While \Cref{chap:modelkinds} focused on understanding how
mathematical expressions could be dissected and transformed into ``code,'' it
neglected to discuss which expressions it could even begin to dissect and
transform \textemdash{} that is to say, which expressions are ``valid?'' The
objective of this chapter is to create a system of type rules, and enforce
well-formedness/typedness\footnote{Please note that I will be using
      ``well-formed'' and ``well-typed'' interchangeably.} through them, for Drasils
mathematical expression languages.

\section{Recap of Drasils Math-related Expression Languages}

To recap, at this point, we have three (3) relevant and used ``mathematical
expression'' languages.

\subsection{One for \textquotedblleft{}Simple\textquotedblright{} Mathematics}

\Expr{} is a mathematical expression languages whose vocabulary is intended to
always have a definite value. In other words, with little to no extra work on
your end, you should be able to directly input these expressions in your
standard calculator (perhaps with a bit of work to handle vectors, functions,
etc.) to evaluate them.

\subsection{One For \textquotedblleft{}Code\textquotedblright{}}

\CodeExpr{} is a heavily mathematics-focused expression language with a few
extra features over \Expr{} for \acs{gool}/``code.'' The vocabulary should be
nearly directly usable in \acs{gool} for outputting to general-purpose
object-oriented programming languages. \CodeExpr{} is a superset of \Expr{}:
\(\Expr{} \subseteq{} \CodeExpr{}\).

\subsection{And One For General Mathematics}

\ModelExpr{} is the classical mathematics we know and love. It contains nearly
everything we know (up to what we've encoded thus far) and is intended to be a
descriptive language, with no particular restrictions on its terms (other than
that they should at least be describable on pencil and paper too). \ModelExpr{}
is also a superset of \Expr{}: \(\Expr{} \subseteq{} \ModelExpr{}\).

However, \ModelExpr{}s terms are unlikely to appear in \CodeExpr{} due to their
indescribable nature in computable \acs{oo} ``code.''

As of right now, these languages have proven themselves to be effective
encodings, weakly proven through Drasils case studies being able to produce
working software artifacts. However, they are not without issue. Notably, at the
moment, Drasil does not have any readily-available type information about their
constructions. This lack of type information hampers the ``reliability'' aspect
of the code generator because the generator is unable to restrict its output
artifacts to those which are directly usable. In order for the generated
artifacts to be directly usable, they must be \textit{well-typed} programs. In
other words, we need to make sure that the generated expressions and programs
conform to the \textit{type rules} of the respective interpreters and/or
compilers.

\section{Type Safety}

Before compilation/execution, programming language compilers and interpreters
check input programs against a logical \textit{type system} to ensure that the
steps never perform invalid instructions, where program evaluation may become
impossible\footnote{Some instructions/operations may be ``nonsense!''}. For
example, to avoid nonsensical instructions, such as \(1 + \texttt{true}\), where
parameter \textit{type} mismatches occur. In this example, a computer (like us)
have no reasonable way to understand how to add \(1\) and the truth value
\texttt{true}. Type systems provide \textit{types} (such as
\(\mathbb{Z}\)/\texttt{integer} and \(\mathbb{B}\)/\texttt{boolean}) that are
assigned to each \textit{term} (such as \(1\), \(+\), and \texttt{true}), and a
set of \textit{typing rules} to restrict how terms interact and form. Types are
information about the structure of terms. \textit{Types} are typically
meta-level information. Terms are the value/``primitive'' data of a programming
language, such as numerics, functions/methods, operators, and modules. A series
of \textit{inference rules} makes up the type rules of a system. Here,
\textit{type safety} is approximately an assurance of \textit{preservation} and
\textit{progress} formed through the typing rules \cite{Harper2016}.
\textit{Preservation} is an assurance that the steps of evaluation preserve
typing. \textit{Progress} is an assurance that ``well-typed'' expressions can be
evaluated to value, or they are already a value and no evaluation is necessary.
For an expression to be ``well-typed'' (or ``well-formed'') means that it is
guaranteed to evaluate fully to a value without unexpected/illegal operations.
For a language to be considered \textit{type-safe}, it means that it only admits
valid well-typed expressions and a definition for what it means to be well-typed
with respect to some evaluation function.

Invalid/ill-typed expressions aren't only ``bad'' because they aren't fully
evaluable. They're also bad because they dilute the pool of expressions we're
interested in, and that we may want to analyze. For ill-typed expressions,
automated analysis is only good for automatically searching for the ill-typed
areas, not using them to understand the intent of the writer (who may have just
made a type error). Manual analysis is okay, but it does not scale well against
large amounts of expressions (for which Drasil handles). Additionally, ill-typed
expressions add extra overhead on us when handling them, either to ``correct''
them or to ignore them.

\subsection{A Simple Language}

For example, if we had a small ``simple'' language, \(\bb{L}\), that contains
terms for integer and boolean values, and addition, ``less than'' comparison,
conjunction, and if-then-else (ternary operators)\footnote{Assume the
      definitions of the functions be total and understood/used under the conventional
      sense that mathematicians so often do.}, we might write the syntax inductively,
as follows:

\begin{longtable}{ r c c l}
      \(\bb{L}(l)\) & ::=       & \(n\)                                                        & Integers (where \(n\) is any integer) \\
                    & \(\vert\) & \texttt{true}                                                & True                                  \\
                    & \(\vert\) & \texttt{false}                                               & False                                 \\
                    & \(\vert\) & \(l_1\ \texttt{+}\ l_2\)                                     & Addition                              \\
                    & \(\vert\) & \(l_1\ \texttt{<}\ l_2\)                                     & ``Less than'' comparison              \\
                    & \(\vert\) & \(l_1\ \texttt{\land}\ l_2\)                                 & Conjunction                           \\
                    & \(\vert\) & \(\texttt{if}\ l_1\ \texttt{then}\ l_2\ \texttt{else}\ l_3\) & if-then-else (ternary ``if'')         \\
\end{longtable}

Great! Now, we can form expressions, such as:

\begin{equation}
      10
      \label{ex:sl:s:good1}
\end{equation}

\begin{equation}
      23 + (400\ \texttt{+}\ 4000)
      \label{ex:sl:s:good2}
\end{equation}

\begin{equation}
      \texttt{if}\ \texttt{true}\ \texttt{then}\ 95\ \texttt{else}\ 96
      \label{ex:sl:s:good3}
\end{equation}

\begin{equation}
      42\ \texttt{+}\ \texttt{false}
      \label{ex:sl:s:bad1}
\end{equation}

\begin{equation}
      22 + \texttt{if}\ (40\ \texttt{+}\ 400)\ <\ 96\ \texttt{then}\ \texttt{false}\ \texttt{else}\ 96\ \texttt{+}\ 400
      \label{ex:sl:s:bad2}
\end{equation}

\begin{equation}
      \texttt{if}\ 0\ \texttt{then}\ 1\ \texttt{else}\ \texttt{true}
      \label{ex:sl:s:bad3}
\end{equation}

% Forcibly rename the "equation" environment tags to "expression" for the sake
% of the below discussion.
\Crefname{equation}{Expression}{Expressions}

Now, let's evaluate these expressions.
\Cref{ex:sl:s:good1,ex:sl:s:good2,ex:sl:s:good3} can be calculated with a
conventional understanding of the operations, respectively, as \(10\), \(4423\),
and \(95\). However, \Cref{ex:sl:s:bad1,ex:sl:s:bad2,ex:sl:s:bad3} are
worrisome. Regarding, \Cref{ex:sl:s:bad1}, we just don't have any conventional
sense of addition on integers with booleans, so evaluation is unclear.
Continuing, to calculate the outermost addition expression of
\Cref{ex:sl:s:bad2}, we must first calculate the if-then-else condition would
evaluate to \texttt{true}, so we can short-circuit to the left expression
(\texttt{false}), but now we have no conventional sense of addition on integers
with booleans. So, we have an undefined expression, similar to
\Cref{ex:sl:s:bad1}. Finally, unless we were familiar with C-like languages, we
might not want to think of \(0\) as being equivalent to \texttt{false}, so we
immediately find issue in the condition of \Cref{ex:sl:s:bad3}. However,
ignoring the issue with the condition value, we might also find issue, similar
to \Cref{ex:sl:s:bad2}, with how different \textit{kinds} of values in the
branch operands.

\Crefname{equation}{Equation}{Equations}

Other than modifying the syntax into a convoluted mess to avoid issues like
these (which won't be easy, and might not be possible), we look to \textit{type
      systems} to rescue the \(\bb{L}\). The key is in understanding that there are
different ``kinds'' (\textit{types}) of values (\textit{terms}). Then, we can
either filter out the invalid expressions. First, we must analyze and capture
our universe of types, \(\tau\):

\begin{longtable}{ r c c l}
      \(\Tau(\tau)\) & ::=       & \(\bb{B}\) & Booleans \\
                     & \(\vert\) & \(\bb{Z}\) & Integers
\end{longtable}

Note that we are restricting the numeric-related operations to strictly
integers. The restriction is only there for simplification of numerics. \(\tau\)
is an enumeration of all permissible \textit{types} of \textit{terms} we can
have in \(\bb{L}\).

Next, we need to add the typing rules. They will restrict our syntax to only
those constructions which are semantically valid. We will do so using inference
judgments, as follows:

\begin{equation}
      \left.
      \infer{n : \bb{Z}}{}
      \right.
      \qquad
      \text{(where \(n\) is any integer.)}
      \label{eq:exTR:int}
\end{equation}

\begin{equation}
      \left.
      \infer{\texttt{true} : \bb{B}}{}
      \right.
      \qquad
      \text{True}
      \label{eq:exTR:true}
\end{equation}

\begin{equation}
      \left.
      \infer{\texttt{false} : \bb{B}}{}
      \right.
      \qquad
      \text{False}
      \label{eq:exTR:false}
\end{equation}

\begin{equation}
      \left.
      \infer{(a\ \texttt{+}\ b) : \bb{Z}}
      {a : \bb{Z}  &  b : \bb{Z}}
      \right.
      \qquad
      \text{Addition}
      \label{eq:exTR:addition}
\end{equation}

\begin{equation}
      \left.
      \infer{(a\ \texttt{<}\ b) : \bb{B}}
      {a : \bb{Z}  &  b : \bb{Z}}
      \right.
      \qquad
      \text{``Less than'' comparison}
      \label{eq:exTR:lessThan}
\end{equation}

\begin{equation}
      \left.
      \infer{(a\ \texttt{\wedge}\ b) : \bb{B}}
      {a : \bb{B}  &  b : \bb{B}}
      \right.
      \qquad
      \text{Conjunction}
      \label{eq:exTR:conjunction}
\end{equation}

\begin{equation}
      \left.
      \infer{(\texttt{if}\ b\ \texttt{then}\ x\ \texttt{else}\ y) : \tau}
      {b : \bb{B}  &  x : \tau  &  y : \tau}
      \right.
      \qquad
      \text{if-then-else (ternary ``if'')}
      \label{eq:exTR:ifThenElse}
\end{equation}

So as long as we follow these typing rules while we build our expressions, when
we try to evaluate any of these expressions, we should not arrive at invalid
expressions where evaluation cannot be completed.

Notably, on paper and pencil, or common typesetting, there is nothing tangibly
stopping us from continuing to write invalid expressions. However, with
computerized languages, we're able to tangibly (as far as ``tangible'' goes with
computers) stop users from writing invalid expressions. This is a huge gain over
conventional pencil and paper, because we're able to type check statements with
far less time and effort than if we were to do it manually.

\section{Back to Our Expression Languages}

Similar to the untyped \(\bb{L}\), in Drasil, our expression languages (\Expr{},
\CodeExpr{}, and \ModelExpr{}) are \textit{untyped}\footnote{i.e., they don't
      have type information, and are not type checked.}. We currently need to manually
ensure that our generated artifacts \textit{type check}. If we only had a few
expressions, then it would not be much of an issue to manually type check them.
However, with scale, it becomes problematic. In particular, just throwing
``generation'' into the mix makes it an area of concern, because there is no
assurance of type safety. In this chapter, we aim to bring type information to
the expression languages, and see how we can improve type safety and expression
usage.

\subsection{Decomposing the Expression Languages}

Before we begin to decompose the expression languages into their semantics,
syntax, and typing rules, we will start off with a common denominator language
to all the three (3): \Literal{}. The \Literal{} language is a simple
non-recursive sum type, where each possible value is a different kind of
``literal'' value\footnote{Note that this work is basing it's understanding on
      the existing syntax and usage, and does not intend to argue the existence of
      certain operations or kinds of values necessarily.}. The syntax is as follows:

\begin{longtable}{ r c c l}
      \(\mathit{Literal}(l)\) & ::=       & \(n\)              & Integers (where \(n \in \bb{Z}\))                \\
                              & \(\vert\) & \(s\)              & Strings (where \(s\) is any string of text)      \\
                              & \(\vert\) & \(r\)              & Real numbers (where \(r \in \bb{R}\)             \\
                              & \(\vert\) & \(r_w\)            & Whole-numbered
      reals (where \(r_w \in \bb{R} \land r_w \in \bb{Z}\)\
      \footnote{Intentionally redundant to indicate that \(r_w\) is a real
      number, but it should also be ``whole,'' and, as such, an integer.})                                        \\
                              & \(\vert\) & \(n \texttt{/} d\) & Fractions/percentages (where \(n,d \in \bb{Z}\)) \\
\end{longtable}

The types are as follows:

\begin{longtable}{ r c c l}
      \(\Tau_{\mathit{Literal}}(\tau)\) & ::=       & \(\bb{R}\) & Reals    \\
                                        & \(\vert\) & \(\bb{Z}\) & Integers \\
                                        & \(\vert\) & \(\bb{S}\) & Strings
      \footnote{For sake of brevity, we will think of \(\bb{S}\) as the type of
            all possible strings of text.}
\end{longtable}

Creating the typing rules is a fairly straightforward process:

\begin{equation}
      \left.
      \infer{n : \bb{Z}}{}
      \right.
      \qquad
      \text{(where \(n \in \bb{Z}\))}
      \label{tr:lit:int}
\end{equation}

\begin{equation}
      \left.
      \infer{n : \bb{S}}{}
      \right.
      \qquad
      \text{(where \(s\) is any string of text.)}
      \label{tr:lit:string}
\end{equation}

\begin{equation}
      \left.
      \infer{r : \bb{R}}{}
      \right.
      \qquad
      \text{(where \(r \in \bb{R}\).)}
      \label{tr:lit:real}
\end{equation}

\begin{equation}
      \left.
      \infer{r_w : \bb{R}}{}
      \right.
      \qquad
      \text{(where \(r_w \in \bb{R} \land \lfloor{} r_w \rfloor{} = r_w\).)}
      \label{tr:lit:wholeReal}
\end{equation}

\begin{equation}
      \left.
      \infer{(n\ \texttt{/}\ d) : \bb{R}}{}
      \right.
      \qquad
      \text{(where \(n, d \in \bb{Z}\).)}
      \label{tr:lit:fraction}
\end{equation}

Note that the ``Whole-numbered Reals'' and ``Reals'' may have overlap. The
difference is merely the information difference. This is something that can be
altered later, if desired.

\imptodo{Continue writing here!}

\tedioustodo{Draw out the syntax diagram of each language, but start with Expr
      and discuss how CodeExpr and ModelExpr are extensions of Expr.}

\tedioustodo{Re-write the typing rules without the Haskell code references (keep
      it mathematical/theoretical).}

\tedioustodo{Discuss how and where we can add type information to Drasils
      expression languages, and the pros/cons of the solutions. For example,
      should type enforcement done at the construction-level or post-facto
      processed?}

In Haskell, there are at least two ways that we can perform these type rules: at
creation, and after creation. If we choose to restrict at the construction, we
have a stronger sense of type safety for any and all expressions, unlike with
post-creation manual checking of expressions, where you can't necessarily share
proofs that an expression ``well-typed''\footnote{At least not in Haskell, but
      perhaps in Agda!}.

\imptodo{Either preserve some stuff from the below content of this chapter, or
      delete it all.}

\begin{itemize}

      \item Compiling to ``lower languages'' requires special type checking
            before compiling to them. For example, the Swift code generator has
            to ensure that there are no ambiguously typed numerals as the types
            of numerics are not overloaded in Swift.

      \item Dynamically checking for invalid expression states is possible, but
            difficult and would result in increasingly difficult term tracking
            as terms in the expression language grow/are added.

      \item In general, being able to express invalid expressions causes large
            burden and mental overhead.

\end{itemize}


\section{Requirements \& properties of a good solution}

\begin{itemize}

      \item Invalid expressions should not be representable in the various
            expression languages (i.e., the expression types should strictly
            indicate valid expression constructions), without loss of
            generality.

      \item Invalid expression formation attempts should be statically found and
            reported by the compiler, at compile-time. This will move the
            previously runtime errors to compile-time.

      \item Invalid expression cases should not need to be considered when
            working (e.g., case-ing) with expressions.

      \item ``Safety = Preservation + Progress'' (\cite{Harper2016}, Ch.6)

\end{itemize}

\section{Solution}

\begin{itemize}

      \item Use TTF encodings of the smart constructors to lessen the cognitive
            load of handling at least 3 different expression languages.

      \item Statically type all 3 variants of Expr through GADTs.

\end{itemize}

\subsection{Syntax}

\subsubsection{Current}

An idealized version of the current syntax.

\startSyntaxTable
    \newsyntaxRow{Type}{\tau}{Integer}{\bb{Z}}{Integer numbers}
    \syntaxRow{Real}{\bb{R}}{Real numbers}
    \syntaxRow{String}{String}{Text}
    \syntaxRow{Bool}{\bb{B}}{Truth values (true/false)*}
    \syntaxRow{Vector($\tau$, $n$)}{[\tau]_n}
        {Vectors (single element type, fixed length)*}
    \syntaxRow{Tuple($\tau_1...\tau_n$)}
        {\tau_1 \times \tau_2 \times ... \times \tau_n}
        {Alternative vectors/tuples (fixed length+differently typed elements)*}
     \\

    \newsyntaxRow{Literal}{l}{Integer[$n$]}{n}{Integer number}
    \syntaxRow{Real[$r$]}{r}{Real number}
    \syntaxRow{String[$s$]}{``s''}{Text}
    \syntaxRow{Bool[$b$]}{b}{Boolean value}
    \syntaxRow{Vector($l_1...l_n$)}{<l_1, ..., l_n>}{Vectors*}
    \syntaxRow{Tuple($l_1...l_n$)}{(l_1, ..., l_n)}{Tuples*}
     \\

    \newsyntaxRow{UnaryOp}{\ominus}{Not}{\lnot \_}{Logical negation}
    \syntaxRow{Neg}{- \_}{Numeric negation}
    \syntaxRow{...}{}{\tiny{omitted for brevity}}
     \\

    \newsyntaxRow{BinaryOp}{\oplus}{Sub}{\_ - \_}{Subtraction}
    \syntaxRow{Pow}{\_^\_}{Powers}
    \syntaxRow{...}{}{\tiny{omitted for brevity}}
     \\
    
    \newsyntaxRow{AssocBinOp}{\otimes}{Add}{\_ + \_}{Addition}
    \syntaxRow{Mul}{\_ \times \_}{Multiplication}
    \syntaxRow{...}{}{\tiny{omitted for brevity}}
     \\

    \newsyntaxRow{UID}{u}{UID(s)}{\texttt{UID ``s''}}{UIDs}
     \\

    \newsyntaxRow{Expr}{e}{Literal($l$)}{l}{Literal values}
    \syntaxRow{Vector($e_1...e_n$)}{<e_1, ..., e_n>}{Vectors}
    \syntaxRow{Var($u$)}{u}{Variable (QuantityDict Chunk)}
    \syntaxRow{FuncCall($f, e_1...e_n$)}{f(e_1...e_n)}
        {``Complete'' function application}
    \syntaxRow{UnaryOp($\ominus$,$e$)}{\ominus\ e}{Unary operations}
    \syntaxRow{BinaryOp($\oplus$,$e_1$,$e_2$)}
        {e_1 \oplus e_2}{Binary operations}
    \syntaxRow{AssocOp($\otimes$, $e_1...e_n$)}{e_1 \otimes ... \otimes e_n}
        {Associative binary operations}
    \syntaxRow{Case($e_{1c}e_{1e}...e_{nc}e_{ne}$)}
        {if\ e_{1c}\ then\ e_{2e}\ elif\ e_{2c}\ ...}
        {If-then-else-if-then-else (Switch-like statements)}
    \syntaxRow{BigAsBinOp($\otimes, e_1, e_2$)}{\bigotimes_{i=e_1}^{e_2} i}
        {Apply a ``big'' op to a discrete range defined as a range}
    \syntaxRow{IsInRlItrvl($u, e_1, e_2$)}{u \in [e_1, e_2]}{Variable in range}

\closeSyntaxTable

    $*$: does not currently appear in the code at the moment, but would be
    needed/desired


\subsection{Typing Rules}

\subsubsection{Miscellaneous}


\begin{enumerate}

    \item Completeness:
        \newrule{}
            {\ofTy{Complete[]}{Completeness}}
        
        \newrule{}
            {\ofTy{Incomplete[]}{Completeness}}

    \item AssocOp:
        \begin{enumerate}
            \item Numerics:
                \newrule{\ofTy{x}{Numerics($\tau$)}}
                    {\ofTy{Add[]}{AssocOp x}}
        
                \newrule{\ofTy{x}{Numerics($\tau$)}}
                    {\ofTy{Mul[]}{AssocOp x}}
    
            \item Bool:
                \newrule{}
                    {\ofTy{And[]}{AssocOp Bool}}
        
                \newrule{}
                    {\ofTy{Or[]}{AssocOp Bool}}
        \end{enumerate}

    \item UnaryOp:
        \begin{enumerate}
            \item Numerics:
                \newrule{\ofTy{x}{NumericsWithNegation(x)}}
                    {\ofTy{Neg[]}{UnaryOp x x}}

                \newrule{\ofTy{x}{NumericsWithNegation(x)}}
                    {\ofTy{Abs[]}{UnaryOp x x}}
                
                % TODO: Exp
                
                For Log, Ln, Sin, Cos, Tan, Sec, Csc, Cot, Arcsin, Arccos, Arctan, and Sqrt, please use the following template, replacing ``$\$TRG$'' with the desired operator:
                \newrule{}
                    {\ofTy{\$TRG[]}{UnaryOp Real Real}}

            \item Vectors:
                \newrule{\ofTy{x}{NumericsWithNegation(x)}}
                    {\ofTy{NegV[]}{UnaryOp [x] [x]}}

                \newrule{\ofTy{x}{Numerics(x)}}
                    {\ofTy{Norm[]}{UnaryOp [x] Real}}

                \newrule{\ty{x}}
                    {\ofTy{Dim[]}{UnaryOp [x] Integer}}
                
            \item Booleans:
                \newrule{}
                    {\ofTy{Not[]}{UnaryOp Bool Bool}}

        \end{enumerate}

    \item BinaryOp:
        \begin{enumerate}
            \item Arithmetic: % TODO: 
            \item Bool: % TODO: 
            \item Equality: % TODO: 
            \item Ordering: % TODO: 
            \item Indexing: % TODO: 
            \item Vectors: % TODO: 
        \end{enumerate}
    
    \item RTopology:
        \newrule{}
            {\ofTy{Discrete[]}{RTopology}}

        \newrule{}
            {\ofTy{Continuous[]}{RTopology}}
    
    \item DomainDesc: % TODO: Why does the topology appear as a type constructor argument, and type signature argument?
        \newrule{\ofTy{top}{$\tau_1$} & \ofTy{bot}{$\tau_2$} & \ofTy{s}{Symbol} & \ofTy{rtop}{RTopology}}
            {\ofTy{BoundedDD[s, rtop, top, bot]}{DomainDesc Discrete $\tau_1$ $\tau_2$}}

        \newrule{\ty{topT} & \ty{botT} & \ofTy{s}{Symbol} & \ofTy{rtop}{RTopology}}
            {\ofTy{AllDD[s, rtop]}{DomainDesc Continuous topT botT}}

    \item Inclusive:
        \newrule{}
            {\ofTy{Inc[]}{Inclusive}}

        \newrule{}
            {\ofTy{Exc[]}{Inclusive}}

    \item RealInterval:
        \newrule{\ty{a} & \ty{b} & \ofTy{top}{(Inclusive, a)} & \ofTy{bot}{(Inclusive, b)}}
            {\ofTy{Bounded[top, bot]}{RealInterval a b}}

        \newrule{\ty{a} & \ty{b} & \ofTy{top}{(Inclusive, a)}}
            {\ofTy{UpTo[top]}{RealInterval a b}}

        \newrule{\ty{a} & \ty{b} & \ofTy{bot}{(Inclusive, b)}}
            {\ofTy{UpFrom[bot]}{RealInterval a b}}

\end{enumerate}


\subsubsection{Expr}

\begin{haskell}{Current Expression Language}{curExpr}{https://github.com/JacquesCarette/Drasil/blob/dc3674274edb00b1ae0d63e04ba03729e1db\newline{}c6f9/code/drasil-lang/lib/Language/Drasil/Expr/Lang.hs\#L81-L135}{https://github.com/JacquesCarette/Drasil/blob/dc3674274edb00b1ae0d63e04ba03729e1dbc6f9/code/drasil-lang/lib/Language/Drasil/Expr/Lang.hs\#L81-L135}
-- | Expression language where all terms are supposed to be 'well understood'
--   (i.e., have a definite meaning). Right now, this coincides with
--   "having a definite value", but should not be restricted to that.
data Expr where
  -- | Brings a literal into the expression language.
  Lit :: Literal -> Expr
  -- | Takes an associative arithmetic operator with a list of expressions.
  AssocA   :: AssocArithOper -> [Expr] -> Expr
  -- | Takes an associative boolean operator with a list of expressions.
  AssocB   :: AssocBoolOper  -> [Expr] -> Expr
  -- | C stands for "Chunk", for referring to a chunk in an expression.
  --   Implicitly assumes that the chunk has a symbol.
  C        :: UID -> Expr
  -- | A function call accepts a list of parameters and a list of named parameters.
  --   For example
  --
  --   * F(x) is (FCall F [x] []).
  --   * F(x,y) would be (FCall F [x,y]).
  --   * F(x,n=y) would be (FCall F [x] [(n,y)]).
  FCall    :: UID -> [Expr] -> [(UID, Expr)] -> Expr
  -- | For multi-case expressions, each pair represents one case.
  Case     :: Completeness -> [(Expr, Relation)] -> Expr
  -- | Represents a matrix of expressions.
  Matrix   :: [[Expr]] -> Expr
  -- | Unary operation for most functions (eg. sin, cos, log, etc.).
  UnaryOp       :: UFunc -> Expr -> Expr
  -- | Unary operation for @Bool -> Bool@ operations.
  UnaryOpB      :: UFuncB -> Expr -> Expr
  -- | Unary operation for @Vector -> Vector@ operations.
  UnaryOpVV     :: UFuncVV -> Expr -> Expr
  -- | Unary operation for @Vector -> Number@ operations.
  UnaryOpVN     :: UFuncVN -> Expr -> Expr
  -- | Binary operator for arithmetic between expressions (fractional, power, and subtraction).
  ArithBinaryOp :: ArithBinOp -> Expr -> Expr -> Expr
  -- | Binary operator for boolean operators (implies, iff).
  BoolBinaryOp  :: BoolBinOp -> Expr -> Expr -> Expr
  -- | Binary operator for equality between expressions.
  EqBinaryOp    :: EqBinOp -> Expr -> Expr -> Expr
  -- | Binary operator for indexing two expressions.
  LABinaryOp    :: LABinOp -> Expr -> Expr -> Expr
  -- | Binary operator for ordering expressions (less than, greater than, etc.).
  OrdBinaryOp   :: OrdBinOp -> Expr -> Expr -> Expr
  -- | Binary operator for @Vector x Vector -> Vector@ operations (cross product).
  VVVBinaryOp   :: VVVBinOp -> Expr -> Expr -> Expr
  -- | Binary operator for @Vector x Vector -> Number@ operations (dot product).
  VVNBinaryOp   :: VVNBinOp -> Expr -> Expr -> Expr
  -- | Operators are generalized arithmetic operators over a 'DomainDesc'
  --   of an 'Expr'.  Could be called BigOp.
  --   ex: Summation is represented via 'Add' over a discrete domain.
  Operator :: AssocArithOper -> DiscreteDomainDesc Expr Expr -> Expr -> Expr
  -- | A different kind of 'IsIn'. A 'UID' is an element of an interval.
  RealI    :: UID -> RealInterval Expr Expr -> Expr
\end{haskell}

