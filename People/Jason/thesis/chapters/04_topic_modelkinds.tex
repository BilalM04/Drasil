% ModelKinds -- Theory types / discrimination -- ``Expressions in context''

In working to understand some phenomenon, we often look to our understanding of
the phenomenon. In mathematics (amongst other fields, such as philosophy,
physics, chemistry, biology, computer science, etc. [anything studied in a
university?]), we often think about various phenomenon in terms of theories and
axioms, where the theories and axioms are written using a global mathematical
language with precise meanings and sharing no similarity to English and human to
human communication languages.

A language is used to describe a field of knowledge (theories and axioms). The
language is typically tailored to the field of knowledge. One language of
interest to this research is that of the mathematical language. A mathematical
language is used in understanding theories and axioms. As mathematicians
transferring knowledge to one another, we often use a written form of knowledge,
though unlikely, potentially with a precise structure to our transcription of
the knowledge, but still, the structure is textual. We might break up the
transcription of the knowledge into sections. The sections of a theory might
perhaps have a name, a description, a derivation, some information on it's
origin (e.g., a reference), and some sort of information in a language we can
use to convey the important information we decided to share. Of course, for a
mathematical theory, a mathematical language would be used. Describing a
programming language artifact, we might prefer to call our theories by a series
of other names (e.g., functions/methods, constants, modules, statements,
variables, etc.), and prefer to use the parts of the AST of a programming
language to describe our theories.

In Drasil, both of these languages are of great importance. The mathematical
field of knowledge is directly used in at least two domains currently captured;
the scientific theories, and GOOL\todo{cite}. Meanwhile, the programming
language theory (artifacts) is only directly used in GOOL\todo{cite}, and it's
various renderers. The SmithEtAl\todo{cite} generator uses knowledge transcribed
as scientific requirements to automatically generate a series of
representational software programs. In the SmithEtAl generator doing this, it
requires both fields, extensively. The focus of the scientific requirements
documents are to transcribe the relevant knowledge of a scientific system as
developed by a domain/scientific expert. For example, \todo{SRS doc example} is
used to fully explain \todo{something} such that a software developer can
construct a program based on it. Of course, if the SRS document is to be truly
sufficient for a software developer to unambiguously create a representational
software artifact, then the software developer (who potentially knows nothing
about the domain discussed in the SRS document) will need to be able to credibly
transcribe everything into their program, matching the requirements as set by
the SRS document. As Drasil is an apparatus for testing our understanding of
phenomenon, the SmithEtAl template is being tested for it's confidence in this
regard (and is being shown to work). However, if the process is to be truly
``assembly-line''-able, then we should be able to mechanize it.

In Drasil, a proof of concept of this understanding is being built. In
particular, the SmithEtAl template is being attempted to be systematically used
to generate a family of software artifacts in a variety of \acl{oo} programming
languages through building a single \ACL{gool}.

In order for Drasil to generate software artifacts that solve problems, it must
have a way of discussing the space your problem lies in, and the ways it can be
transformed into the working calculation steps that a software artifact can
follow to solve your problem.

\section{A Unified Language}
\label{sec:modelkinds:language}

\originalExprHaskell

The original smart constructors were specific to Expr:

\originalFewExprSmartConstructorsHaskell

Theories were transcribed using ``RelationConcepts'':

\originalRelationConcept

Where ``Relations'' are merely an alias for Expr:

\originalRelation{}

\theoriesWithoutModelKinds

With ModelKinds, we, approximately, obtain the following:

\theoriesWithModelKinds

We split up Expr into 3 kinds of expression languages that better fit our needs.

\section{Problem}

\begin{itemize}

	\item In modelling any problem, one will, of course, model the work of their
	      pencil and paper. Thankfully, with scientific problems, the
	      well-understood ones will be codified\cite{well-understood}, and as
	      such, are a good fit for Drasil, as it is intended to automatically
	      generate software that performs calculations for a specific problem in
	      the universe of generic programs supported.

	\item In most, if not all, physics problems, a common language of mathematic
	      calculation. In Drasil, this language is written as an \ACF{adt} in
	      \refOriginalExprHaskell{}.

	\item Here, we explain that a mathematical expression is defined by the
	      above \acf{ast}. The language contains the commonly found operations
	      in a well-understood physics textbook (here, with a focus on
	      graduate-level scientific problems).

	\item It's important that each knowledge encoding in Drasil exposes as much
	      information as reasonably possible (and useful). We want to expose the
	      ``specifications'' of each piece of knowledge that we are encoding so
	      that transformers and generators may appropriately make use of
	      contained knowledge.

	\item With mathematical models, it's very easy to write ``difficult to
	      interpret'' expressions, and create expressions for which aren't
	      directly calculable (i.e., things that require an extra paper and
	      pencil/mental mathematics before performing), either without extra
	      surrounding information or simply too ``difficult'' (areas where we
	      might find choices are especially difficult).

	      \begin{itemize}

		      \item When writing a pencil \& paper, we usually write with extra
		            context (generally more information that needs to be read to
		            fully understand some expression). We might also infer
		            information about the model. Unfortunately, mechanizing
		            inference is difficult, artificial learning is a whole field
		            of study of its own. As such, we reverse the relationship of
		            the inference by having knowledge container expose
		            everything on its own.

		      \item In other words, we replace Expr as a knowledge container,
		            and restrict its usage to strictly ``expressions'', as
		            opposed to ``expressions'' and information about models.

	      \end{itemize}

	\item Since we want to generate code that represents calculations of all
	      sorts, it's important that the mathematical expression language we use
	      to write calculations expose sufficient information to the generator
	      in a concise and easy-to-digest manner.

	\item For example, assuming the following expressions are written using the
	      existing mathematical Expr language in Drasil...
	      \begin{itemize}

		      \item While \(y = x\) might conventionally be seen as ``y is equal
		            to x'', we might want, in our model, for it to be understood
		            as ``x is defined by y'' but displayed differently.

		      \item \(a = b = c = ... = d\) can be ambiguously read ...

		      \item Truth statements such as \(y < x\) might be usable in
		            asserting constraints at software runtime, but it's
		            difficult to find a place for it to belong without extra
		            context.

	      \end{itemize}

\end{itemize}

\section{Requirements \& properties of a good solution}

\begin{itemize}

	\item Expressions are great for viewing, but not for a computer to
	      systematically use to generate things.

	\item Theories should expose more information about themselves so that we
	      can directly interpret them without needing to traverse over an
	      expression.

	\item Specifically, more expressions should be usable in code generation,
	      amongst other things.

\end{itemize}

\section{Solution}

\begin{itemize}

	\item Mathematical expressions may have definitive meanings at different
	      levels of interpretation. Splitting our focal expression language into
	      3 variants is easy thanks to GADTs and TTF \cite{Carette2009};
	      CodeExpr, ModelExpr, and Expr. This will allow us to restrict terms to
	      the different levels of interpretation through a common type (either
	      using the GADT, or the TTF constraints).

	\item Replacing ``RelationConcepts'':

	      \begin{itemize}

		      \item Using raw expressions to transcribe whole mathematical
		            theories and data structures leads to many problems and
		            difficulties in interpretation alone.

		      \item Replacing raw expressions with ``encodings one step higher''
		            that can push out the same raw expressions as a property of
		            the higher-level encodings will allow us to quickly
		            identify, discriminate, and efficiently use the information
		            originally contained in the expressions better. This is done
		            through replacing ``RelationConcepts'' with ``ModelKinds'':

		      \item \currentModelKindsHaskell

		      \item \intodo{Discuss EquationalModels, EquationalRealms,
			            EquationalConstraints, DEModel, NewDEModel, etc.}

			  \item We can still push out ModelExprs and Exprs from things where
	         		necessary, if they can be explained using them.

			  \item \intodo{Justify type parameters everywhere}

	      \end{itemize}

\end{itemize}

\subsection{Continued}

\intodo{Remaining problems}
