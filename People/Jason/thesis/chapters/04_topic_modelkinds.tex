% ModelKinds -- Theory types / discrimination -- ``Expressions in context''

In working to understand some phenomenon, we often look to the boundaries of our
understandings of the phenomenon — the open-ended questions and gaps in our
knowledge. In any domain of knowledge (such as mathematics, philosophy, physics,
chemistry, biology, computer science, or any other studied subject), we often
think about various phenomenon in logical terms of theories and axioms, where
the theories and axioms are written using a global logical language with precise
meanings\todo{I almost make it sound like we're not using English/Any other
	language to communicate ideas, I should clarify this.}. The language is
typically tailored to the field of knowledge. As this work pertains to Drasil,
one language of interest to this research is that of the mathematical language.
A mathematical language is heavily used in science, conveying important
theories, axioms, corollaries, amongst other things. As mathematicians
transferring knowledge to one another, we often use a written form of knowledge,
though unlikely, potentially, with a precise structure to our transcription of
the knowledge, but still, the structure is textual. We might break up the
transcription of the knowledge into sections. The sections of a theory may have
a name, a natural language description, a derivation, some information regarding
its origin (e.g., a reference), and a formalization of the theories important
information in a precise language. Of course, for a mathematical theory, a
mathematical language would be used. Describing a programming language artifact,
we might prefer to call our theories by a series of other names (e.g.,
functions/methods, constants, modules, packages, libraries, \acsp{api},
\acsp{ast}.

In Drasil, both of these languages are of great importance. The mathematical
field of knowledge is directly used in at least two domains currently captured:
the scientific theories, and \acs{gool} \cite{Carette2019}. Meanwhile, the
programming language theory (artifacts) is only directly used in \acs{gool}, and
its various renderers. The SmithEtAl \acs{srs} \cite{SmithAndLai2005} template
generator uses knowledge transcribed as scientific requirements to automatically
generate a series of representational software programs. Thankfully, Smith and
Lai \cite{SmithAndLai2005} break down theories relevant to \acs{scs} into at
least four (4) kinds for their \acs{srs} template:

\todo{Define the 4 kinds of models here.}
\begin{enumerate}

	\item \textbf{Theory Models}:

	\item \textbf{General Definitions}:

	\item \textbf{Instance Models}:

	\item \textbf{Data Definitions}:

\end{enumerate}

\intodo{Example of a theory/instance model here.}

\intodo{Insider knowledge: the theories are actively being re-thought.}

Since the SmithEtAl generator focuses itself on knowledge captured in the form
of the \acs{srs} template, Drasil requires a comprehensive understanding of both
fields in order to generate representational software for the \acs{srs}. The
focus of the scientific requirements documents are to transcribe the relevant
knowledge of a \acs{scs} system as developed by a domain/scientific expert. For
example, \todo{SRS document example} is used to fully explain \todo{something}
such that a software developer can construct a program based on it. Of course,
if the \acs{srs} document is to be truly sufficient for a software developer to
unambiguously create a representational software artifact, then the software
developer (who potentially knows nothing about the domain discussed in the
\acs{srs} document) will need to be able to credibly transcribe everything into
their program, matching the requirements as set by the \acs{srs} document.
Drasil is an apparatus for testing our understanding of these scenarios, rather
than having a read-only ``view'' of the scientists/domain experts knowledge
available to the software developer, the knowledge itself is available, in its
most raw encoded form.

Drasil relies on searching for a calculation path that relates the inputs and
outputs designated in an \acs{srs} document. The relations are based on the
grounded theories (\textit{Instance Models}) designated in the \acs{srs}. The
relations themselves are currently required to be of the form \(x = f(a, b, c,
\ldots{})\), with some exceptions made for \acsp{ode}. \acsp{ode} are being
actively developed to remove the manually written exceptions made for them
(outside the scope of this work).

Instance Models, General Definitions, and Data Definitions each rely on a
\textit{Relation Concept} (\RelationConcept{}): a notable and named mathematical
relation with a description and abbreviated name. A \RelationConcept{} is
modelled as a \Relation{} coupled with a \ConceptChunk{} (\todo{Add ConceptChunk
	original definition to thesis appendix.}, a named \textit{thing} with a name,
abbreviation, and natural language description):

\originalRelationConcept{}

The \Relation{}s \todo{add a reference to an appendix entry for the ``type
	Relation = Expr'' definition} in the \RelationConcept{} are, fundamentally,
just an instance of a universal mathematical language which conveys their
information, but given a type synonym to indicate that the instance should
be a mathematical relation.

\section{A Universal Math Language}
\label{sec:modelkinds:language}

\originalExprHaskell

The mathematical language used in Drasil is universally used, covering a wide
range of knowledge, including facilities for creating commonly used primitive
data types, operations, and functions  in mathematics, physics, and computer
science and programming languages. For example, \todo{An expression.} is
transcribed in Drasil as follows:

\intodo{Transcription of the above expression.}

This transcription relies on smart constructors, such as \todo{Ref. a new
	appendix entry for new smart constructors.}. The smart constructors used are
all specialized to \Expr{} and perform simplifications along the way (such
as folding \(1 \cdot x\) into \(x\)).

\section{A Universal Theory Language}

As Drasils theories rely heavily on \Relation{}s (and \RelationConcept{}s)
(\todo{Ref. the definitions of DDs, IMs, GDs, and TMs}), we may observe that the
theories are an accurate reflection of writing out mathematics involved as one
might write them down on paper: exactly as they please. In modelling any
problem, one will, of course, model the work of their pencil and paper. This is
exactly what occurs here. Theories here are shallow representations of
knowledge, that mimic your pencil and paper.

\intodo{Rewrite the point form notes in the ModelKinds section.}

\theoriesWithoutModelKinds

With ModelKinds, we, approximately, obtain the following:

\theoriesWithModelKinds

We split up Expr into 3 kinds of expression languages that better fit our needs.

\section{Problem}

\begin{itemize}

	\item In modelling any problem, one will, of course, model the work of their
	      pencil and paper. Thankfully, with scientific problems, the
	      well-understood ones will be codified\cite{well-understood}, and as
	      such, are a good fit for Drasil, as it is intended to automatically
	      generate software that performs calculations for a specific problem in
	      the universe of generic programs supported.

	\item In most, if not all, physics problems, a common language of mathematic
	      calculation. In Drasil, this language is written as an \ACF{adt} in
	      \refOriginalExprHaskell{}.

	\item Here, we explain that a mathematical expression is defined by the
	      above \acf{ast}. The language contains the commonly found operations
	      in a well-understood physics textbook (here, with a focus on
	      graduate-level scientific problems).

	\item It is important that each knowledge encoding in Drasil exposes as much
	      information as reasonably possible (and useful). We want to expose the
	      ``specifications'' of each piece of knowledge that we are encoding so
	      that transformers and generators may appropriately make use of
	      contained knowledge.

	\item It is very easy to write ``difficult/impossible to interpret''
	      expressions. For example, it is possible to create expressions for
	      which aren't directly calculable (i.e., things that require an extra
	      paper and pencil/mental mathematics before performing), either without
	      extra surrounding information or simply impossible.

	      \begin{itemize}

		      \item When writing a pencil \& paper, we usually write with extra
		            context (generally more information that needs to be read to
		            fully understand some expression). We might also infer
		            information about the model. Unfortunately, mechanizing
		            inference is difficult, artificial learning is a whole field
		            of study of its own. As such, we reverse the relationship of
		            the inference by having knowledge container expose
		            everything on its own. We additionally expect knowledge as a
		            requisite to working with the model.

		      \item In other words, we replace Expr as a knowledge container,
		            and restrict its usage to strictly ``expressions'', as
		            opposed to ``expressions'' and information about models.

	      \end{itemize}

	\item Since we want to generate code that represents calculations of all
	      sorts, it's important that the mathematical expression language we use
	      to write calculations expose sufficient information to the generator
	      in a concise and easy-to-digest manner.

	\item For example, assuming the following expressions are written using the
	      existing mathematical Expr language in Drasil\ldots{}
	      \begin{itemize}

		      \item While \(y = x\) might conventionally be seen as ``y is equal
		            to x'', we might want, in our model, for it to be understood
		            as ``x is defined by y'' but displayed differently.

		      \item \(a = b = c = ... = d\) can be ambiguously read\ldots{}

		      \item Truth statements such as \(y < x\) might be usable in
		            asserting constraints at software runtime, but it's
		            difficult to find a place for it to belong without extra
		            context.

	      \end{itemize}

\end{itemize}

\section{Requirements \& properties of a good solution}

\begin{itemize}

	\item Expressions are great for viewing, but not for a computer to
	      systematically use to generate things.

	\item Theories should expose more information about themselves so that we
	      can directly interpret them without needing to traverse over an
	      expression.

	\item Specifically, more expressions should be usable in code generation,
	      amongst other things.

\end{itemize}

\section{Solution}

\begin{itemize}

	\item Mathematical expressions may have definitive meanings at different
	      levels of interpretation. Splitting our focal expression language into
	      3 variants is easy thanks to GADTs and TTF \cite{Carette2009};
	      CodeExpr, ModelExpr, and Expr. This will allow us to restrict terms to
	      the different levels of interpretation through a common type (either
	      using the GADT, or the TTF constraints).

	\item Splitting Expr into 3 variants; Expr, CodeExpr, and ModelExpr. This
	      allows us to both segregate terms to particular domains, and add weak
	      typing where necessary, as required (e.g., areas where we might
	      require a dependently typed host language).

	      \begin{itemize}

		      \item
		            \url{https://github.com/JacquesCarette/Drasil/issues/1220}

	      \end{itemize}


	\item Replacing ``RelationConcepts'':

	      \begin{itemize}

		      \item Using raw expressions to transcribe whole mathematical
		            theories and data structures leads to many problems and
		            difficulties in interpretation alone.

		      \item Replacing raw expressions with ``encodings one step higher''
		            that can push out the same raw expressions as a property of
		            the higher-level encodings will allow us to quickly
		            identify, discriminate, and efficiently use the information
		            originally contained in the expressions better. This is done
		            through replacing ``RelationConcepts'' with ``ModelKinds'':

		      \item \currentModelKindsHaskell

		      \item \intodo{Discuss EquationalModels, EquationalRealms,
			            EquationalConstraints, DEModel, NewDEModel, etc.}

		      \item We can still push out ModelExprs and Exprs from things where
		            necessary, if they can be explained using them.

		      \item \intodo{Justify type parameters everywhere}

	      \end{itemize}

\end{itemize}

\intodo{Remaining problems}
