\chapter{Classifying Theories}
\label{chap:framing-theories}

\begin{writingdirectives}
    \item What are they?
    \item What are theories used for?
    \item How are they captured in Drasil?
    \item Current problems? Solutions?
\end{writingdirectives}

In this chapter, we will focus on improving inspection and interpretation
capabilities of theories in Drasil. Specifically, with focus on interpreting
them for generating software artifacts.

\section{Theories}
\label{chap:framing-theories:sec:theories}

As mentioned in \Cref{chap:drasil}, the \acs{srs} template
\cite{SmithAndLai2005} breaks up software requirements and problems into a
series of well-understood components, providing developers with concrete
solution requirements they must satisfy, and domain experts with justification
for problem solutions. Notably, the \acs{srs} relates a programs \textit{inputs}
to a set of \textit{outputs} using a set of \textit{theories}. The inputs and
outputs are sets of variables, with data that need to be somehow fed into the
program, or calculated and output by the program. The theories connect the input
variables to the output by forming a \textit{solution/calculation path}. There
are at least 3 kinds of theories\footnote{There's also a 4th kind: Data
Definitions, intended for explaining how input variables should be interpreted
by the solution program, thus also explaining how they should be formatted; they
are typically intended to be implementation-focused, rather than theoretical.}:
Theory Models, General Definitions, and Instance Models. Theory Models and
General Definitions provide justification for the mathematics of the solution:
the Instance Models. The Instance Models, specifically, together form the
calculation path.

For example, Drasils \porthref{\acs{projectile} case
study}{https://jacquescarette.github.io/Drasil/examples/projectile/SRS/srs/Projectile_SRS.html}
describes how to estimate if a launcher, aligned at a particular angle, will hit
a target from a specific distance. The \acs{srs} requires users to fill in the:
\begin{enumerate}
    \item input variables:
          \begin{enumerate}
              \item \(p_\mathit{target}\), the targets distance from the
                    launcher,
              \item \(v_\mathit{launch}\), the projectile launch speed,
              \item and \(\theta\), the launch angle.
          \end{enumerate}
    \item output variables:
          \begin{enumerate}
              \item \(s\), a message, explaining if the projectile hit the
                    target, fell short, or went long,
              \item and \(d_\mathit{offset}\), the expected distance between the
                    target position and the landing position.
          \end{enumerate}
    \item and theories, connecting the inputs to the outputs:
          \begin{enumerate}
              \item \({t_{\text{flight}}}=\frac{2 {v_{\text{launch}}}
                        \sin\left(\theta{}\right)}{\mathbf{g}}\), estimating
                    flight time with \(v_{\mathit{launch}}\) and \(\theta\),
              \item \({p_{\text{land}}}=\frac{2 {v_{\text{launch}}}^{2}
                        \sin\left(\theta{}\right)
                        \cos\left(\theta{}\right)}{\mathbf{g}}\), a calculation
                    of the landing position,
              \item \({d_{\text{offset}}}={p_{\text{land}}}-{p_{\text{target}}}\),
                    calculation of distance between the targets position and the
                    expected landing position of the projectile,
              \item and \(s=\begin{cases} \text{``The target was hit.''}, &
              |\frac{{d_{\text{offset}}}}{{p_{\text{target}}}}| < \varepsilon{}
              \\
              \text{``The projectile fell short.''}, & {d_{\text{offset}}} < 0 \\
              \text{``The projectile went long.''},  & {d_{\text{offset}}} >
              0\end{cases}\), \newline{}calculating the output message.
          \end{enumerate}
\end{enumerate}

From these 3 key bodies of information along with some supporting background
knowledge (such as assumptions, constants, etc.), Drasil forms a calculation
path, deriving the output variables from the input variables using the instance
models\footnote{And, less importantly, data definitions.}. Here, Drasil
approximately recognizes that to calculate the outputs, it approximately needs
to follow a path along the lines of the order of presented theories. With it,
Drasil is able to generate representational code. For example, \acsp{projectile}
generates \refOriginalJavaProjectileMain{} for one of the Java-flavoured
artifacts. In \refOriginalJavaProjectileMain{}, it uses a
\inlineCode{java}{write_output} method to output the calculated output variables
after calculating them using the relevant theories\footnote{Note:
\(t_\text{flight}\) is seemingly unused in the generated code, but it is used in
the derivation of \(p_\text{land}\). However, it being ``unused'' is irrelevant
to this work.}.

\originalJavaProjectileMain{}

In order to build the quantities, one does it similar to
\refOriginalQuantityDictExampleHaskell{}. Similarly, in order to build the
theories, Drasil also has a \acs{dsl}, \RelationConcept{}s
(\refOriginalRelationConcept{}); they are used in Drasils representation of
Instance Models, and allow users to couple a mathematical relation (encoded
using \Relation{}s), a natural language description of the relation, and a name.

\originalRelationConcept{}

Notably, the \Relation{}s are any mathematical relation. To make use of them,
Drasil tries to analyze them (\refOriginalRelToQDHaskell{}) to understand their
capabilities, and, where designated, attempt to generate code fragments through
\acs{gool}. Unfortunately, this analysis is brittle, relying on Drasils users
encoding their theory knowledge in precise forms.

\originalRelToQDHaskell{}

As we can see in \refOriginalRelToQDHaskell{}, the current analysis only covers
relations of the form \(x = f(a, b, c, \ldots{})\). Here arises 2 notable
issues:
\begin{enumerate}
    \item[\namedlabel{mk:issue:1}{Issue 1}] it only handles one theory kind:
    variable definitions,
    \item[\namedlabel{mk:issue:2}{Issue 2}] and, for those definitions, it
    requires a specific form.
\end{enumerate}

As a result of \ref{mk:issue:1}, we aren't able to encode adequately all the
theories we're interested in using, and want to generate representational code
of. In particular, as Drasil is heavily guided by physics-focused case studies,
\acsp{ode} are desired! When we want to use \acsp{ode} in the solution of a
problem, extra information is required. For example, we might need to give
Drasil (and/or developers) information about a desired approximation formula
with particular ``settings.'' Drasil does circumvent this issue for \acsp{ode},
but we would like to reconcile the half-measures and push all necessary
information back in to the theory encodings.

Assuming we wanted to describe the theory of a line, there are many ways we can
describe the equation: polynomial (\(a \cdot{} x + b \cdot{} y + c = 0\)),
slope-intercept form (\(y = m \cdot{} x + b\)), point-slope form (\(y_1 - y_2 =
m(x_1 - x_2)\)), and so on. However, as a result of \ref{mk:issue:2}, we are
forced to use ``simple'' slope-intercept form, even though we are aware of other
forms and may prefer to describe it in other forms.

\imptodo{Continue writing here!}



















\section{Classifying by Structure}
\label{chap:framing-theories:sec:classifying-and-structuring}


\imptodo{Now, recall, the theories are supposed to expose sufficient information
    such that a developer can use them to immediately write code. Similarly, it
    should also expose enough information such that we can describe how it
    translates to code.}

\imptodo{Note something about increasing the ``depth'' and ``breadth'' of
    captured knowledge.}







An \Expr{} alone is a weak conveyor of the inner knowledge of theories, similar
to normal ``pencil and paper'' mathematical expressions, without extra
information, the expression alone may be ineffectual or nearly unusable in code
generation.

The theories are ``shallow'', or ``flat,'' representations of the theoretical
information because they do not directly provide information about the structure
of the information they carry.

When transmitting information on pencil and paper to your colleagues, this is
fine, because they will likely read extra contextual information from it, and
know how to use it accordingly.

However, our machines aren't able to infer as well as we're able to. Our
machines are clueless to the contextual information about the theories because
we have provided no means of discussion to them.

In other words, Drasil does not yet understand the language of the associated
context.

In general, it is important that each knowledge encoding in Drasil exposes as
much information as reasonably possible (and useful).

We want to expose the ``specifications'' of each piece of knowledge that we are
encoding so that transformers and generators may appropriately make use of
contained knowledge.

Any other forms of theories were unusable. This was known as the ``\relToQD{}
hack.''

\refOriginalRelToQDHaskell{} is a symptom of insufficient knowledge capture.

\relToQD{} relied on implicit (and unguaranteed) requirements of inputs. It had
no static checks for consistency that an input ``theory'' was indeed usable in
code generation.

More generally, it is very easy to write ``difficult/impossible to interpret''
expressions.

For example, it is possible to create expressions for which aren't directly
calculable (i.e., things that require an extra paper and pencil/mental
mathematics before performing), either without extra surrounding information or
simply impossible.












\section{Reconciling Mathematical Knowledge}

Fundamentally, the issues arise because the information is too ``shallow'' for
Drasil to adequately make use of them \textemdash{} it can't infer from
expressions.

The raw expressions are great for viewing and human-guided inference by experts,
but not an inexperienced person, or for a computer to systematically use to
generate things.

At the moment, the available tools' type signatures would appear as something
too amazing, so much so in fact, that one would (and should) question its
accuracy.

A hypothetical type signature, \inlineHs{RelationConcept -> Code}, is quite far
fetched.

Not all expressions that can possibly be contained in a \RelationConcept{} are
usable in code\footnote{When we discuss ``code'' generation, we target
generating usable software, but we currently primarily target general-purpose
\acs{oo} programs.} generation.















To resolve this, we need to reconcile \textit{theories} with
\textit{mathematical knowledge}, strengthening the \textit{depth of knowledge}
contained in a theory.

Should this occur, we should observe \refTheoriesWithoutModelKinds{} having the
\textit{Theory} and \textit{Mathematical Knowledge} nodes merged, and have the
expressions understood to only be one of many possible ``views'' of higher-level
usable knowledge.

In other words, the expressions would not be used to transfer knowledge any
longer, but they might remain as one component of it.

Through resolving these issues, we will have deeper knowledge available at
Drasils compile-time, and we will be able to better understand which theories
are usable in code generation, and which aren't.

Additionally, we will be able to better handle more \textit{kinds} of theories
without needing to create complex traversal and analysis algorithms to recognize
when certain kinds of theories were transcribed in the expressions.



















\section{\textquotedblleft{}Classify All The Theories\textquotedblright{}}

Issues occurring due to weak knowledge capture may be resolved through strong
knowledge capture.

Beginning with the existing case studies of Drasil, we will attempt to classify
our existing knowledge better.

We aim to make \RelationConcept{} a ``view'' of other more information-dense
encodings.

In other words, we replace \Expr{} as a knowledge container, and restrict its
usage to strictly ``mathematical expressions'', as opposed to ``expressions''
and information about models/theories.

One notable change is that we will require the new theory knowledge containers
to be able to fully re-create the original shallow/raw \Expr{}s as a property of
the new theory encodings.

The once meta-level knowledge of the theories, lost in the Haskell
implementation, becomes exposed and understood to Drasil.

Ultimately, this is done through replacing \RelationConcept{} usage with
\ModelKind{}\footnote{\ModelKind{} is built upon Dr. Jacques Carettes prototype
of an earlier version of \ModelKinds{}}, an aggregation of existing
Drasil-related knowledge of mathematical theories. \ModelKind{} is defined using
a \acs{gadt}, with one (1) type parameter. The type parameter is currently used
to determine whether the model is ``fully refined''/``grounded'' or not, and,
hence, usable in code generation.

\refCurrentModelKindsHaskell{} displays the creation of \ModelKind{} and
\ModelKinds{}.

Please note that this aggregation is based purely on the existing model examples
in the existing Drasil case studies, and the existing models are incomplete in
the larger scope.

Each \textit{kind of model} we find only has one requirement: that it should
carry enough information to, and provide a means of, recreating the original
expression from which they were abstracted out of.

This requirement is essentially that of ``viewing'' it in the expression
language, but it also tends to also add flexibility in how many ways that we can
``view'' the information in different forms.

In the Haskell code, this requirement is enforced through requiring them all to
instantiate the \Express{} typeclass (\refCurrentExpressHaskell{}).

\currentModelKindsHaskell{}

In the above \ModelKinds{} definition, there are two (2) TODO notes that you may
disregard.

The first one is merely a note for analyzing ``well-understood'' copies of our
existing \acsp{ode}, and the second one refers to models that haven't yet been
fully analyzed for how they will be used (other than for display).

\subsection{Quantity Definitions}

\currentQDefinitionHaskell{}

Assume \(y = x\) is transcribed as a \RelationConcept{}: while \(y = x\) might
conventionally be seen as ``y is equal to x'', we might want, in our model, for
it to be understood as ``x is defined by y'' but displayed differently.

Here, \(=\) is overloaded as ``definition'', instead of what \(=\) was defined
as in \Expr{}, as an ``equality'' operator.

To resolve this overloading and weak knowledge capture of definitions, we create
\EquationalModel{}s: theories that contain information about definitions of
symbols, built using a \QDefinition{} (\refCurrentQDefinitionHaskell{}).

If an \EquationalModel{} deals with theoretical symbols and is defined using
either a \ModelExpr{} or an \Expr{}, it may be used in Theory Models and General
Definitions.

If an \EquationalModel{} is defined using an \Expr{} and deals with only the
concrete (non-abstract) symbols, then the \EquationalModel{} is usable for code
generation.

At the moment, there is no information attached to symbols yet regarding whether
they are abstract or instanced, so that portion of the rule is not enforced.

\eztodo{Example of an EquationalModel/QDefinition in Haskell code, the SRS, and
	the generated code.}




