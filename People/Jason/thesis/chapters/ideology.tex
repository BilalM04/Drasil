The focus of this work is fundamentally based on the idea of ``generation.''
However, unlike GitHub and OpenAI's Copilot, the ideology does not delve into
artificial intelligence, and does not focus on ``autocompleting code'' by
natural language. Instead, the ideology focuses on creating principled stories
of knowledge that we may use to generate software artifacts (amongst other
knowledge).

\section{On Developing Software}\label{sec:idlgy:on_developing_software}

As a software developer working to build a new piece of software, one might
sense that they are writing ``a lot of the same code'' as their or other
pre-existing software projects. One might consider building a library, shared
between all projects for some common functionality/tooling, this is a large
improvement for their program --- being able to reuse their code is great for
debugging and removing the possibility of bugs when stable and tested code is
used. The library will surely save them time in the future once they've
stabilized it to a reasonable degree, after which they will not need to worry
about repeating the same errors they made when they were originally writing it.
However, the library might not be portable across machines, the library might
not make sense to those more or less familiar with certain ideas touched upon by
the library, others might question the validity of the knowledge it pulls from,
and the library might not be readily accessible to those not using the same
programming language the library was written in (of course, they can write an
FFI\todo{FFI acronym}, but this is a fairly complex task
that many are unfamiliar with).

Commonly, one looks to use mature libraries and frameworks to underpin their
projects, occasionally without guarantee that connecting these libraries is
safe. As general purpose programming languages are often also used,
misunderstandings of tacit project knowledge may also cause errors.
Unfortunately, this programming methodology takes significant time until a
working product is formed with minimal bugs, even when building on the shoulders
of giants.

To resolve this, we believe we need to revisit the original sensation felt when
code re-creation/duplication was recognized -- why does this sensation occur?

The reason is obvious (ignoring the more obvious fact that they are aware of
similarities between their code and that previously written)! They feel this
sensation because they already understood it and had written it previously.
Implicitly, when writing the software, the developer already had a tie between
some algorithm and the code they had written (at times, specific to the
programming language they had chosen to use).

Unfortunately, a non-trivial amount of time is spent manually recreating the
same piece of code from the same implicit knowledge-base, and the developer is
left with less than preferred time to work on the ``components that are
interesting'' that they might never have built before or that they might not
have connected together in the past.

\section{Thoughts of Generation}\label{sec:idlgy:thoughts_of_generation}

The ideology foundational to this work is predicated on this simple idea that if
we understand how a \textit{thing} works, and we can create a working model
describing it, then we should be able to encode it, describing everything about
this model. Furthermore, if we understand how this \textit{thing} can be
transformed into another \textit{thing}, then we should be able to describe the
process of transformation as well. Finally, if we are able to do this en masse
to a pool of \textit{various things}, we should be able to generate whole pools
of \textit{other various things} from seedling pools.

Applying this idea to the world of software, where we can create and model
\textit{things} (henceforth referred to as ``knowledge''), we can take pools of
knowledge and generate other pools of knowledge (including whole software
artifacts). With a weak breadth and low depth of captured knowledge, the
originating pool of knowledge may approximately appear as a description of the
software artifacts generated. However, with a strong breadth and high depth of
captured knowledge, the original pool of knowledge may appear far removed from
the software artifacts generated (it may contain little to no discussion of the
desired software artifacts at all). In this case, the stress of software
development is alleviated entirely, and a ``perfect'' software artifact (or
series of ``perfect'' software artifacts) is (are) constructed.

\section{The Goal}\label{sec:idlgy:the_goal}

If one originally sets out to build a program that does something they
understand very well and each component of every step of the grand
scheme/algorithm of the program was understood, the development of their related
software should be a clear matter of principled engineering. However, with the
existing programming methodology, it is not yet simple enough to reliably
produce error-free programs with no trade-offs.

Optimally, they would use their natural language to perfectly describe their
problem to their computer and have it magically give them a program that does
exactly what they wanted. Unfortunately, it is difficult to have computers
systematically understand and act on natural language as well as us humans can.

\section{Reconciliation -- ``Generate Everything''}\label{sec:idlgy:generate_everything}

While it is difficult to have a computer act on a huge set of natural language,
we do have well-understood thoughts on specific \textit{terms} and how they can
be interpreted. More specifically, there are sub-languages of natural language
that we can use to describe specific kinds of knowledge, and problems and
solutions. Here, we recognize that selecting or creating domain-specific
languages for specific buckets of knowledge and creating domain-specific
languages that connect buckets of other domain-specific languages, we can
effectively create rigid sub-languages of our natural language used to describe
programs. These rigid sub-languages being encoded as domain-specific languages
will have a well-defined and formalized AST\todo{AST acronym}, which allows us
to write interpreters that can take them and transform them into other fragments
of knowledge (us being most interested in ultimately generating software
artifacts). Thus, with enough effort, and through sequencing and connecting
terms of domain-specific languages, we can effectively model what the discussed
software developers are trying to build, allowing the computer to be able to
better understand what they are trying to build.

Note, however, that this idea likely only thrives in domains of knowledge that
is ``well-understood'' \cite{well-understood}.

To sum up, this should appear, in theory, as a knowledge management
system\todo{acronym}, where generation phases are passed through the knowledge
registry until a final knowledge registry is constructed such that it satisfies
the requirements of the user (e.g., generating some desired software artifacts).

\section{Prospective Workflow \& Roles}\label{sec:idlgy:prospective_workflow}

Following this ideology, there will be at least 3 key roles: the
\textbf{knowledge encoder}, the \textbf{knowledge user}, and the final
\textbf{software artifact user}. 

The knowledge encoder should be a master of a particular domain. They are
expected to encode the knowledge discussed in their respective domain in such a
way that is accessible to those without knowledge of their domain. Additionally,
they should encode information about the ways in which the knowledge can be
transformed into other forms of knowledge (including that which is
interdisciplinary). The knowledge they would be encoding should be as well-known
and globally standardized as possible. As discussed in
\autoref{sec:idlgy:generate_everything}, it is likely that the knowledge encoder
will focus on writing a series of highly specific domain-specific languages,
where the languages may be restricted to as specific as one term.

The knowledge user/orchestrator should at least be familiar with a particular
domain, and have goals in mind for information that they would like to ``seed''
into their knowledge management system. \todo{continue}

The final end-user should find the most delight from this ideology. If the tasks
assigned to the knowledge encoders and the knowledge users are performed
correctly, then the end-user should have strong confidence in the fruits of
their labour as they were built with strict adherence to the knowledge captured
at \textit{every step of the way}. As such, one should confidently expect the
final software artifacts to be completely devoid of unexpected things (including
errors, unconformities to specifications, etc.).

%% TODO: Move the below stuff up
As a domain expert transcribing knowledge encodings of some well-understood
domain, one will largely be discussing the ways in which pieces of knowledge are
\textit{constructed} and \textit{relate to each other}. In order for this
abstract knowledge encodings to be \textit{usable} in some way, it is vital to
have ``names'' (\textit{types}) for the knowledge encodings. In working to
capture the working knowledge of a domain, it's of utmost importance to ensure
that all ``instances'' of your ``names'' (types) are \textit{always} usable in
some meaningful way. In other words, all knowledge encodings should create a
stringent, explicit set of rules for which all ``instances'' should conform to,
and, arguably, also creates a justification for the need to create that
particular knowledge/data type. As such, optimally, a domain expert would write
their knowledge encodings and renderers in a general purpose programming
language with a sound type system (e.g., Haskell, Agda, Java, etc.) --
preferring ones with a type system based on formal type theories for their
feature richness.

\section{Feasibility}\label{sec:idlgy:feasibility}

\begin{itemize}

    \item Already proven for low depth -- see WordPress\todo{cite} and other
          Content Management Systems (CMSs).

    \item For higher depths, we need to restrict this to areas of
          well-understood knowledge.

\end{itemize}
