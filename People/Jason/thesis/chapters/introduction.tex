\section{Context: Knowledge Capture \& Encoding}
todo

TODO: While my MSc focus isn't necessarily exactly ``When Capturing Knowledge Improves Productivity'', I should have a reasonable amount of discussion here about it, or else readers might question the need for my work entirely. I guess I would also need a bit of justification for that programming ideology as well.

\section{Problem Statement}
As a domain expert transcribing knowledge encodings of some well\-understood domain, one will largely be discussing the ways in which pieces of knowledge are \textit{constructed} and \textit{relate to each other}.
In order for this abstract knowledge encodings to be usable, it is vital to have ``names'' (\textit{types}) for our knowledge encodings.
In working to capture knowledge of a domain, it's of utmost importance to ensure that all ``instances'' of your ``name'' (type) are \textit{always} usable in some meaningful way.
In other words, all knowledge encodings should create an stringent, explicit set of rules for which all ``instances'' should conform to, and, arguably, also creates a justification for the need to create that particular knowledge/data type. 
As such, optimally, a domain expert would write their knowledge encodings and renderers in a general purpose programming language with a sound type system (e.g., Haskell, Agda, Java, etc) -- preferring ones with a type system based on formal type theories for their feature richness.

In Drasil, we are focused in understanding families of scientific software, and creating systematic rules to generate families of software solutions (for any instance of a scientific problem that requires a scientific software solution).
Specifically, Drasil is focused on mathematics and physics-based models.
In both areas, we are concerned with what \textit{kinds of theories} are well\-understood, and ensuring that all created mathematical expressions are ``valid''. TODO: Still a bit awkward

\section{Thesis Outline}
todo
