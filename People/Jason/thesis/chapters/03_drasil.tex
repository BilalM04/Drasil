\chapter{Drasil}
\label{chap:drasil}

\footnotetext[1]{\url{https://jacquescarette.github.io/Drasil/}}

\begin{mdleftbar}
      ``Drasil is a framework for generating families of software artifacts from
      a coherent knowledge base, following its mantra, ``Generate All The
      Things!''. Drasil uses a series of variably sized \ACFP{dsl} to describe
      various fragments of knowledge that domain experts and users alike may use
      to piece together fragments of knowledge into a coherent ``story''.
      Through forming some coherent ``story'' in a domain captured by Drasil, a
      representational software artifact may be generated. Drasil currently
      focuses on \ACF{scs}, following Smith and Lai's \ACF{srs} template as
      described in \cite{SmithAndLai2005}. Behind the scenes of the \acs{srs}, a
      mathematical language is used to describe various theories, and have
      representational software constructed via compiling to \ACF{gool}
      \cite{Carette2019}. Through encoding knowledge in Drasil, an increase in
      productivity (and maintainability) in building reliable and traceable
      software artifacts is observed \cite{SzymczakEtAl2016}, specifically in
      \acs{scs} \cite{Smith2018}. Drasil's source code (Haskell), case studies,
      and documentation studies can be found on its
      \porthref{website}{https://jacquescarette.github.io/Drasil/}.''
      \cite{Balaci2021Poster}
\end{mdleftbar}

\section{An Exploration}

Originally known as \ACF{lss}, Drasil is an exploration of this ideology
described in \Cref{chap:ideology}. Drasil's largest domain of knowledge
covered originates from \acs{lss}: \acl{scs}. \porthref{Dr. Jacques
      Carette}{https://www.cas.mcmaster.ca/~carette/} and \porthref{Dr. Spencer
      Smith}{https://www.cas.mcmaster.ca/~smiths/} are the principal
investigators of Drasil. Drasil is deeply embedded in Haskell
\cite{Haskell2010}, relying on Stack \cite{HaskellStack}, and compiling
against \acs{ghc} 8.8.4 \cite{GHC884}. Haskell is the language of choice
for various reasons, but the most important reasons are regarding its
paradigm: purely functional, with immutable data and a strong, sound type
system. This provides a sound system which developers may use to classify,
create, and work with knowledge.

Drasil is currently capable of generating usable software through compiling to
\ACF{gool}, which is capable of producing Java, C++, Python, C\#
\cite{MacLachlan2020}, and Swift (not discussed in MacLachlan's Master's thesis,
but created by him as well, and available similarly). Drasil contains renderers
for HTML, Makefile, basic Markdown (enough for README), GraphViz DOT (graph
description language) \cite{Gansner1993}, plaintext documents, \LaTeX{}, and
\TeX{}. Drasil's source code is publicly available on
\porthref{GitHub}{https://github.com/JacquesCarette/Drasil}, and Drasil's
documentation
(\porthref{user-facing}{https://jacquescarette.github.io/Drasil/docs/index.html},
and
\porthref{internal}{https://jacquescarette.github.io/Drasil/docs/full/index.html})
is available on the Drasil project
\porthref{homepage}{https://jacquescarette.github.io/Drasil/}. Drasils public
wiki is hosted on the same \porthref{GitHub
      repository}{https://github.com/JacquesCarette/Drasil/wiki}, containing
information on potential future Drasil projects, Drasil-related papers, a
\porthref{developer workspace configuration and ``quick start''
      guide}{https://github.com/JacquesCarette/Drasil/wiki/New-Workspace-Setup}, and a
guide for \porthref{building your own project with
      Drasil}{https://github.com/JacquesCarette/Drasil/wiki/Creating-Your-Project-in-Drasil}.

\section{Methodology}

Drasils development is strongly influenced by its case studies, which focus on
building \ACF{scs} based on solving undergraduate-level physics problems, and
using a formalized \acs{scs} \ACF{srs} template \cite{SmithAndLai2005}. A
``control'' program and \acs{srs} are manually created and used as a baseline
target for the final artifact. Development is focused on creating a system of
justification for each component of the baseline program, following a
``bottom-up'' agile development approach. Drasil has been constructed following
the demands and requirements of eight (8) case studies:

\caseStudiesTable{}

\subsection{Knowledge Dissection and Capture via \textquotedblleft{}Bottom-Up\textquotedblright{} Gathering}

Once baseline/target artifacts are constructed, the essence of the artifacts are
dissected and broken into various \textit{fragments} of knowledge. The fragments
of knowledge are continuously broken up and explored/understood (via capture
through creating knowledge encodings in Haskell) until a sufficient holistic
understanding is formed, generalized, and proven to be capable of re-generating
the baseline artifacts. Hopefully, at the end, the baseline artifacts will have
their flaws highlighted and resolved in the produced artifacts.

In the end of the bottom-up gathering approach, there will need to be knowledge
encodings for, at least:

\begin{enumerate}

      \item The artifact knowledge encodings \textemdash{} Java, JSON, Python,
            HTML, SVG, etc. all use a different language.

      \item The meta-level knowledge encodings \textemdash{} information lost in the
            implementations of specific programs needs to be regained so that we
            can reconstruct it in the original (and other) languages, where
            possible.

\end{enumerate}

To some degree, the artifact-knowledge encodings may be thought of as a
``phenomenon'' to Drasil. They may become part of the meta-level knowledge
encodings as well, but they are also generally where the side effects of Drasil
appear, and file artifacts created.

\subsection{Overview}

The most immediate fragments of relevant knowledge are those that discuss the
produced artifacts. Naturally, Drasil is limited to software-based side effects,
but focused on producing \textit{files} on a computer.

The case studies focus on constructing scientific software that conform to a
precise \acs{srs} \cite{SmithAndLai2005} scheme. The \acs{srs} scheme provides a
coherent scheme for users to input information into a ``story'' that developers
can use to guide their software development. The \acs{srs} requires that experts
break down a scientific problem into a series of inputs and outputs, theoretical
models and instances of them, symbols, assumptions, amongst other things.

The encodings of the desired artifact files are next most important. For
example, an \acs{srs} artifact is typically a file that an end-user will be
viewing (read-only). Thus, a PDF is appropriate, but also inappropriate due to
its complexity, and, hence, we look to constructing \acs{srs} documents using a
series of files built with a textual markup language, and designating that a
secondary compiler should be used to compile the files into a PDF. Naturally,
\LaTeX{}, \TeX{}, and \acs{html} are the immediately desired textual markup
languages for their usability in scientific scenes, portability, large
user-base, and free nature (as opposed to proprietary languages). For a software
artifact, there are likely a few relevant fragments, but those most immediately
important are the language (programming or otherwise) of the software artifacts
and the method to use and compile the software artifact. Generating a Java
program will require for Drasil to have a working encoding of Java programs, and
a means of directing users on how to use a residual Java program.

Once satisfactory encodings of file types are formed, we must look to the
purpose of the files and justify their existence. While a JSON file is strictly
for data serialization and deserialization, a Java file is either the ``entry''
to a program (via it's \texttt{main} method) or a file that might be some useful
to, and/or used in, a Java program. Therefore, we must create a system of
reasoning for why files belong where they do (this might be a part of the
language encoding itself as well). For executable software artifacts, the desire
is to have an end-user \textit{execute} the final software artifacts produced,
and thus we must gain information on what the user expects, and desires. Where
feasible, also creating a conceptual understanding how a program meets their
expectations and desires. With each of Drasil's case studies, a series of
\textit{inputs} are provided to an executable program, and the program is
expected to produce \textit{outputs} that somehow make use of the inputs.
Drasils case studies form a system of relations for each study that relates the
inputs to the outputs.

As Drasil is focused on forming scientific software that adheres to a specific
\acs{srs} template \cite{SmithAndLai2005}, the ``system of relations'' is
generally a language of forming a sequence of mathematical calculations.
Knowledge is continuously captured until a sufficient fundamental knowledge-base
is formed such that a user may generate families of software problems using any
relevant domain knowledge. In Drasils case studies, the knowledge capture is
heavily influenced by the \acs{srs} template \cite{SmithAndLai2005}.

This approach requires that there is sufficient implicit and explicit knowledge
around the \acs{srs} document such that a user/developer/machine may follow it,
without any external assistance (e.g., to clear up ambiguities). This ultimately
relies on developing a stable framework for collecting a collection of
``cookie-cutter'' pieces of knowledge, and interacting with them. By breaking
down knowledge into small units of knowledge, the end-result development cycle
should appear similar to using a projectional editor with appropriate \acs{ide}
tooling.

\drasilLogoImg{}

\drasilPersonification

\section{Architecture}

Drasil relies on a domain expert encoding knowledge as a deeply embedded
\ACF{dsl} or record in Haskell. Instances of the knowledge also appear in the
Haskell source code. When knowledge is encoded in Drasil, we refer to it as a
\textit{Chunk} or \textit{knowledge fragment} (they will be used interchangeably
for the purposes of this thesis). The chunks relevant Haskell-level type is its
classifier, and each chunk must have a \ACF{uid} so that one might be able to
refer to a specific occurrence of knowledge. The primitive data held within the
chunks rely on Haskells primitives, \acsp{adt}, and \acsp{gadt}. Each chunk is a
Haskell record, consisting of other encoded data, including instances of
\ACFP{dsl}s encoded in Drasil. As chunk types are created, connected, composed,
referenced, and intertwined, cookie-cutter principles stories become possible,
similar to projectional-style editing. Chunks are funneled into a \textit{Chunk
      Database} (in Haskell, a \ChunkDB{}).

\originalChunkDBHaskell{}

A \ChunkDB{} is currently limited to the above listed chunks
(\refOriginalChunkDBHaskell{}), but one should assume that more data types are
also \textit{chunks} because this is merely a temporary restriction at the
moment. The types of each record item is approximately a map from a \UID{} to an
instance of a chunk (\refOriginalChunkDBTypeMapsHaskell{}). Relating back to the
ideology (as discussed in \autoref{chap:ideology}), in Drasil, the \ChunkDB{} is
where the knowledge is collected, placed, and grown (generated).
\textit{Transformers} are used to operate on the \Chunk{}s in the \ChunkDB{}.
The transformers take in some knowledge, optional \textit{refinement}
information, and convert it into some other knowledge fragment. As expected,
these transformations are ``against the grain'' of the ``bottom-up'' information
gathering\footnote{In a sense that we obtain the transformer of the ``input'' by
abstracting from instances of the ``output.''}. Areas of precise and highly
specific fragments of knowledge have their key components used in generating
something in another language (this typically involves some sort of information
loss/strip in order to restrict the knowledge for a specific purpose). The
produced fragments are directed by the developer. By composing a series of
miniature, ``bite-sized'', transformations, we are able to create large and
highly complex (but well-thought-out and understood).

The largest, most prominent transformer used in Drasil is that of the
\textit{SmithEtAl} knowledge transformer\qtodo{I'm naming this, which I really
      shouldn't be doing, but I need a name for the SRS+Code generator.}. It
is currently represented using the entry point to Haskell programs, the
\inlineHs{main :: IO ()} function is representative of the transformation
capabilities of our understanding of our case studies into sufficient \acs{srs}
documents, and ready-to-use software.

\currentGlassBRMainFHaskell{}

\refCurrentGlassBRMainHaskell{} displays an example, from the \acs{glassbr} case
study, of what a transformer looks like in Drasil. This transformer takes
knowledge collected, configures a transformation task for the desired outputs,
as prescribed by an orchestrating developer, and performs an \inlineHs{IO ()}
effect that dumps an \acs{srs} and a software artifact to the host computer, in
the local working directory. While not seemingly coupled, they indeed are. The
code generator focuses on understanding fragments/ideas relevant to generating
the \acs{srs}, with more stringent rules than the \acs{srs} generator to create
a coherent software artifact. The data collected in the \ChunkDB{}s are
collected at Drasil Haskell-level compile-time, instead of the Drasil run-time,
because the chunks are captured within Haskell source code instead of an
external resource. In other words, the data is embedded in Haskell itself.

\intodo{Code snippet: ``genSRS''}

\intodo{Code snippet: ``genCode''}

\intodo{Code snippet: ``PrintingInformation'', ``SystemInformation'', \& ``Choices''}

\intodo{Discuss how PrintingInformation, SystemInformation, and Choices are used
      to configure a SmithEtAl transformer run.}

The \acs{srs} and software artifacts are optional features. They are also not
always guaranteed to function merely because they were executed. It is up to the
knowledge transcribers to ensure that the body of knowledge recorded is
logically consistent and well-understood to Drasil. In particular, the SmithEtAl
transformer requires that the recorded \textit{theories}/models are consistent,
and that, together, they are able to produce a meta-level holistic single theory
that connects a list of inputs to a list of outputs of a desired calculation.
The calculations should also be reasonably convertible in some programming
language if we decide to use the calculations for code generation as well.

\caseStudiesCodeTable{}

\Cref{tab:drasilCaseStudiesCode} shows a quick overview of what final artifacts
Drasil is capable of generating for each case study. The \acs{srs} is the
simplest to generate because it is not tested for logical consistency and
usability by a software developer. As such, the \acs{srs} is built for each for
case study. Some case studies (\acs{gamephysics}, \acs{hghc}, and \acs{ssp}) are
still actively being developed, but are left incomplete at the time of writing.
The \acs{srs} is currently generated in both \LaTeX{}\ and \acs{html} flavours,
with the \LaTeX{}\ variant having supplementary build information for building
to a single \acs{pdf} file, and the \acs{html} variant accessible from a web
browser compliant with \acs{html} version 5 standards. The \acs{glassbr},
\acs{projectile}, \acs{pdcontroller}, and \acs{nopcm} case studies each are
capable of generating representational software. \acs{nopcm} is usable in all
languages supported by \acs{gool} except for Swift due to the lack of a
Drasil-supported \acs{ode} solving library for the Swift \acs{gool} renderer.
\acs{pdcontroller} was built outside of the normal means of Drasils case studies
development, being built by a student\eztodo{Cite Naveens PR.} at McMaster
University for a class taken. Code generation for \acs{pdcontroller} is not
impossible, it just requires more investigation for the needs of the case study.
However, both the issues related to \acs{nopcm} and \acs{pdcontroller} are
outside the scope of this work.

The code generator works by following the requirements as set in the \acs{srs}
document, applying transformations until a sufficient knowledge-base can be
formed such that a residual \acs{oo} program can be generated via \acs{gool}.
The \acs{srs} document (and relevant encodings of the knowledge in the \acs{srs}
documents) outlines the input and output variables of a \acs{scs}. The Data
Definitions, Instance Models, and symbols are all used to form a ``calculation
path'' that relates the inputs to the outputs of the desired software artifact.
If such calculation path cannot be found, then no program will be generated, and
an error will be displayed to the user at Drasils runtime (e.g., it is not
checked at Drasil compile-time). Each executable software artifact produced is
coupled with a means (currently, a generated Makefile) to run and, where
required, build the software artifact. Additionally, for supported languages,
Doxygen \cite{Doxygen} configurations may be built.

The remaining case studies that do not generate code are still left. Issues
related to their encodings of mathematical knowledge plague their usability in
the generation process.
