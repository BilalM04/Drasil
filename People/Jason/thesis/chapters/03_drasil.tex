\footnotetext[1]{\url{https://jacquescarette.github.io/Drasil/}}

\todo{Can I edit my own quote? Dr. Smith previously had some notes on this.}
\begin{mdleftbar}
      ``Drasil is a framework for generating families of software artifacts from
      a coherent knowledge base, following its mantra, ``Generate All The
      Things!''. Drasil uses a series of variably sized \ACFP{dsl} to describe
      various fragments of knowledge that domain experts and users alike may use
      to piece together fragments of knowledge into a coherent ``story''.
      Through forming some coherent ``story'' in a domain captured by Drasil, a
      representational software artifact may be generated. Drasil currently
      focuses on \ACF{scs}, following Smith and Lai's \ACF{srs} template as
      described in \cite{SmithAndLai2005}. Behind the scenes of the \acs{srs}, a
      mathematical language is used to describe various theories, and have
      representational software constructed via compiling to \ACF{gool}
      \cite{Carette2019}. Through encoding knowledge in Drasil, an increase in
      productivity (and maintainability) in building reliable and traceable
      software artifacts is observed \cite{SzymczakEtAl2016}, specifically in
      \acs{scs} \cite{Smith2018}. Drasil's source code (Haskell), case studies,
      and documentation studies can be found on its
      \porthreftm{website}{https://jacquescarette.github.io/Drasil/}.''
      \cite{Balaci2021Poster}
\end{mdleftbar}

\section{An Exploration}

Originally known as Literate Scientific Software (LSS) \todo{cite}, Drasil is an
exploration of this ideology described in \Cref{chap:ideology}. Drasil's largest
domain of knowledge covered originates from LSS: scientific computing software
(SCS). \porthref{Dr. Jacques Carette}{https://www.cas.mcmaster.ca/~carette/} and
\porthref{Dr. Spencer Smith}{https://www.cas.mcmaster.ca/~smiths/} are the
principal investigators of Drasil. With a focus on building \ACF{scs}, Drasil
uses knowledge to build software conforming to a scientists \ACF{srs} document
by ``learning'' about the knowledge required to form the document, and
generating a representational software as a result. This approach requires that
there is sufficient implicit and explicit knowledge around the \acs{srs}
document such that a user/developer/machine may follow it without any ambiguity.
Drasil is currently focused on building common undergraduate-level physics
models and problems, and generating software that solves them. Deeply embedded
in Haskell \cite{Haskell2010} (and currently built against \acs{ghc} $8.8.4$
\cite{GHC884}), Drasil currently focuses on building tooling around 8 case
studies, each following the formalized \ACF{srs} template
\cite{SmithAndLai2005}:

\intodo{Add citations to each row of the below:}

\caseStudiesTable

Drasil is currently capable of generating usable software through compiling to
\ACF{gool}, which is capable of producing Java, C++, Python, C\#
\cite{MacLachlan2020}, and Swift (not discussed in MacLachlan's Master's thesis,
but created by him as well, and available similarly). Drasil contains renderers
for HTML, Makefile, basic Markdown (enough for README), GraphViz DOT (graph
description language) \cite{Gansner1993}, plaintext documents, and \LaTeX\ /
\TeX. Drasil's source code is publicly available on
\porthref{GitHub}{https://github.com/JacquesCarette/Drasil}, and Drasil's
documentation
(\porthref{user-facing}{https://jacquescarette.github.io/Drasil/docs/index.html},
and
\porthref{internal}{https://jacquescarette.github.io/Drasil/docs/full/index.html})
is available on the Drasil project
\porthref{homepage}{https://jacquescarette.github.io/Drasil/}. A public Drasil
wiki is hosted on the \porthref{same GitHub
      project}{https://github.com/JacquesCarette/Drasil/wiki}, containing information
on potential future Drasil projects, Drasil-related papers, a
\porthref{developer workspace configuration and ``quick start''
      guide}{https://github.com/JacquesCarette/Drasil/wiki/New-Workspace-Setup}, and a
guide for \porthref{building your own project with
      Drasil}{https://github.com/JacquesCarette/Drasil/wiki/Creating-Your-Project-in-Drasil}.

\section{Methodology}

\begin{itemize}

      \item Primarily focused on undergraduate-level science (primarily physics)
            problems

      \item Generating scientific software artifacts
            \begin{itemize}
                  \item Develop a stable knowledge base for physics problems

                  \item Develop a stable framework for laying cookie-cutter
                        problems.

                  \item Make it as simple as using a projectional editor.

            \end{itemize}

      \item Approach to encoding knowledge -- e.g., ``bottom-up''

\end{itemize}

\begin{figure}
      \centering
      \caption{Drasil's Logo}
      \label{fig:drasilLogo}
      \includegraphics[width=0.6\linewidth]{\drasilLogo}
\end{figure}

\intodo{Description of logo? Alternative to the image with the descriptions in the image.}

\section{State of Architecture}

\begin{itemize}

      \item Knowledge Encodings \& Organization
            \begin{itemize}
                  \item Chunks
                  \item ChunkDB
            \end{itemize}

      \item SmithEtAl template for SRS documents

      \item Haskell/GHC 8.8.4

      \item Uses a series of \ACFP{dsl} to build up the knowledge-base and
            ``story'' of a scene/project, until sufficient knowledge is built
            such that a generator may take the whole story and generate
            representational software artifacts.
            \intodo{Should add an example of the encoding of a theory.}

      \item Generates \acs{oo} programs; \acs{gool} is used to generate code in
            C++/C, Java, Python, Swift, and C\#.

      \item Doxygen, Markdown, HTML, TeX, Makefile, CSS, and more
            non-programming software artifacts.

      \item Creates a ``calculation path'' for a series of $x = f(a,b,c,...)$
            formed equations (e.g., simple LHS = complex RHS), and ODEs
            supported by python libraries.

      \item Multiple code generating examples (each with their own
            requirements):
            \begin{itemize}
                  \item GlassBR
                  \item NoPCM
                  \item PDController
                  \item Projectile
            \end{itemize}

      \item The calculation path relies (implicitly) on:
            \begin{itemize}

                  \item The Exprs of the RelationConcept being of the form: $x =
                              f(a,b,c,...)$

                  \item The ODEs being described by the expression language, but
                        we do not rely on being able to generate representing
                        computational code due to needing extra external
                        information (e.g., we require a supplementary packet of
                        information [ODEInfo]).

            \end{itemize}

\end{itemize}

\caseStudiesCodeTable

\intodo{Short-term problems -- leading into topics}


