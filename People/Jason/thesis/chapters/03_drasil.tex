With a focus on building Scientific Computing Software,
Drasil\thinspace\cite{Drasil2021} is an exploration of the ideology described in
\autoref{chap:ideology}. Rather than building one's software project in any
single or combination of general purpose programming language, the usage of a
sequence of domain-specific languages together in Drasil-based projects can be
used to describe common undergraduate level physics models and problems and
describe the target program that simulates said models.

\intodo{Last sentence of above: be more generic, and then mention that it is
    \textit{currently} focused on physics, etc.}

\begin{itemize}
    \item \porthref{public project
              webpage}{https://jacquescarette.github.io/Drasil/}
    \item \porthref{source code is hosted on
              GitHub}{https://github.com/JacquesCarette/Drasil}
    \item \porthref{internal source code documentation is available in the form
              of Haddock
              documentation}{https://jacquescarette.github.io/Drasil/docs/full/index.html}
    \item \porthref{public variant of source code documentation is also
              available}{https://jacquescarette.github.io/Drasil/docs/index.html}
    \item \porthref{public wiki is available on the same GitHub
              project}{https://github.com/JacquesCarette/Drasil/wiki}
          \begin{itemize}
              \item Specifically, \porthref{a workspace configuration and
                        ``quick start'' guide is available on the
                        wiki.}{https://github.com/JacquesCarette/Drasil/wiki/New-Workspace-Setup}
          \end{itemize}
    \item Principal investigators: \porthref{Dr. Jacques
              Carette}{https://www.cas.mcmaster.ca/~carette/} \& \porthref{Dr.
              Spencer Smith}{https://www.cas.mcmaster.ca/~smiths/}
\end{itemize}

\section{Focus}

\begin{itemize}
    \item Primarily for undergraduate-level science (primarily physics) problems
    \item Generating scientific software artifacts
          \begin{itemize}
              \item Develop a stable knowledge base for physics problems
              \item Develop a stable framework for laying cookie-cutter
                    problems.
              \item Make it as simple as using a projectional editor.
          \end{itemize}
\end{itemize}

\section{Methodology}

\begin{itemize}
    \item Approach to encoding knowledge -- e.g., ``bottom-up''
\end{itemize}

\section{Architecture}

\begin{itemize}

    \item Knowledge Encodings
          \begin{itemize}
              \item Chunks
              \item ChunkDB
          \end{itemize}

    \item SmithEtAl template for SRS documents

    \item Haskell/GHC 8.8.4

\end{itemize}

\section{State}

\begin{itemize}

    \item Uses a series of \acfp{dsl} to build up the knowledge-base and
          ``story'' of a scene/project, until sufficient knowledge is built such
          that a generator may take the whole story and generate
          representational software artifacts.
          \intodo{Should add an example of the encoding of a theory.}

    \item Generates OO programs; GOOL is used to generate code in C++/C, Java,
          Python, Swift, and C\#.
          \intodo{non-code artifacts}

    \item Creates a ``calculation path'' for a series of $x = f(a,b,c,...)$
          formed equations (e.g., simple LHS = complex RHS), and ODEs supported
          by python libraries.

    \item Multiple examples:
          \begin{itemize}
              \item DblPendulum
              \item Game physics
              \item GlassBR
              \item HGHC
              \item NoPCM
              \item PDController
              \item Projectile
              \item SglPendulum
              \item SSP
              \item SWHS
          \end{itemize}

    \item Multiple code generating examples (each with their own requirements):
          \begin{itemize}
              \item GlassBR
              \item NoPCM
              \item PDController
              \item Projectile
          \end{itemize}

    \item The calculation path relies (implicitly) on:
          \begin{itemize}

              \item The Exprs of the RelationConcept being of the form: $x =
              f(a,b,c,...)$

              \item The ODEs being described by the expression language, but we
                    do not rely on being able to generate representing
                    computational code due to needing extra external information
                    (e.g., we require a supplementary packet of information
                    [ODEInfo]).

          \end{itemize}

\end{itemize}

\intodo{Short-term problems -- leading into topics}


