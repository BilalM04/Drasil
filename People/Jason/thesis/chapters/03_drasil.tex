\chapter{Drasil}
\label{chap:drasil}

\begin{writingdirectives}

      \item What is Drasil?
      \begin{enumerate}
            \item What does it do?
            \item Where can we find information about it?
            \item What are its successes?
      \end{enumerate}

      \item How does Drasil work?
      \begin{enumerate}
            \item SRS? Generation?
            \item Specifically, what are its current problems?
      \end{enumerate}

\end{writingdirectives}

\drasilLogoImg{}

\section{What is it? What can it do?}
\label{chap:drasil:sec:what-is-it-what-can-it-do}

Principally investigated by \porthref{Dr. Jacques
      Carette}{https://www.cas.mcmaster.ca/~carette/} and \porthref{Dr. Spencer
      Smith}{https://www.cas.mcmaster.ca/~smiths/},
\porthref{Drasil}{https://jacquescarette.github.io/Drasil/} is a software suite
for generating software for well-understood problems through a knowledge-first
approach \cite{Drasil2021}. Drasil captures the background knowledge involved
with software development to make it \textit{reusable}, improve
\textit{maintainability} of software, and strengthen \textit{traceability}
between desired ``software artifacts''\footnote{``Software artifacts'' being any
      file with a well-defined structure, such as plaintext files, Python code,
      \LaTeX{} code, \acs{html}, or \acs{json}.} and the background knowledge
\cite{SzymczakEtAl2016}. Currently, Drasil focuses on generating software
artifacts for \ACF{scs}, where it has been shown to improve software qualities,
such as \textit{verifiability}, \textit{reliability}, and \textit{usability}
\cite{Smith2018}.

\roughNetworkOfDomains{}

Drasils knowledge-capture approach to software development allows users to
remove themselves from the discussion of ``code'' and focus on the important
bits: the problem the code solves and how ``code'' ultimately relates to
it\footnote{Drasil allows users to ``keep at a safe distance'' from software,
      but only so far as Drasil has encoded the terminology the users rely on for
      conveying their problem to Drasil.}. Drasil relies on a \textit{network of
      domains} \cite{Czarnecki2005} to capture the knowledge required to generate
artifacts for a series of case studies (\refCaseStudiesTable{}) that Drasil
builds and uses to navigate development. Roughly, the network is as per
\refRoughNetworkOfDomains{}, where nodes are the major categories of
domains\footnote{In other words, each node contains its own subdomain as well}
and arrows are mappings between them. The case studies use structured \ACF{srs}
\cite{SmithAndLai2005} abstraction to describe scientific problems and the
background knowledge necessary for developers to manually build software.
Through sufficient capture of the background scientific knowledge\footnote{Such
      as the key theories, input and output variables, and assumptions.}, Drasil
generates software artifacts that solve the precise problem descriptions
(\refCaseStudiesCodeTable{}). One notable success of the knowledge capture is
the reusability of it to regenerate artifacts in different, but similarly
applicable, languages. For example, the \acs{glassbr} case study had
\porthref{software
      artifacts}{https://github.com/smiths/caseStudies/tree/master/CaseStudies/glass}
manually built. Once the knowledge was codified in Drasil, the same knowledge
allows re-creation in \porthref{other
      languages}{https://github.com/JacquesCarette/Drasil/tree/master/code/stable/glassbr}.

\caseStudiesTable{}

Drasil is able to generate a host of \ACF{oo} programming language source codes
through compiling to \ACF{gool} \cite{Carette2019,MacLachlan2020}, which
compiles to several \acs{oo} languages (such as Java, Python, C/C$++$, C\#, and
Swift\footnote{Note that Swift was not discussed in \cite{MacLachlan2020}, but
      the renderer was built by Brooks as well.}). Drasil also contains renderers for
printing \acs{html} files, Makefiles, basic Markdown (enough for ``READMEs''),
GraphViz DOT \cite{Gansner1993} diagrams, and plaintext, \LaTeX{} documents.
\acs{srs} abstractions are renderable in either \LaTeX{} or \acs{html}.

\section{How does it work? How is it used?}
\label{chap:drasil:sec:how-does-it-work-how-is-it-used}

As mentioned in \Cref{chap:drasil:sec:what-is-it-what-can-it-do}, Drasil relies
on building a tree of knowledge that contains sufficient information such that
software artifacts can be ``grown'' from them. The individual pieces of
knowledge are known as \textit{chunks} and are encoded as either \ACFP{adt} or
\ACFP{gadt}. Drasil, and all knowledge captured in Drasil, is deeply embedded in
Haskell \cite{Haskell2010} \porthref{source
      code}{https://github.com/JacquesCarette/Drasil/}\footnote{The source code
      compiles against \acs{ghc} 8.8.4 \cite{GHC884} and uses \acs{ghc} language
      extensions.}. Each chunk has a \textit{type} which defines its structural
information. Chunks contain information encoded with various \ACFP{dsl}. The
network of domains (roughly, \refRoughNetworkOfDomains{}) is made up of a series
of chunks connecting and discussing one another, similar to how we might discuss
abstract concepts.

The ``coherent \acs{srs} abstraction'' of \refRoughNetworkOfDomains{} is
modelled after the \textit{Smith et al.} formal \acs{srs} template
\cite{SmithAndLai2005}, while the ``scientific knowledge'' higher up is a set of
interconnected chunks (and, hence, \acsp{dsl}). The ``scientific knowledge''
chunks are used to fill in the ``gaps'' of the \acs{srs} template. For example,
if we wanted to encode a variable, \(\hat{q}_{\text{tol}}\), representing a real
number, ``Tolerable load,'' we might write it as
\refOriginalQuantityDictExampleHaskell{}, where it is of type \QuantityDict{}
(\refOriginalQuantityDictHaskell{}), the type of variable encodings.

\originalQuantityDictExampleHaskell{}

\originalQuantityDictHaskell{}

Notably, in \refOriginalQuantityDictExampleHaskell{}, the symbol,
\(\hat{q}_{\text{tol}}\), is built using a \Symbol{} \acs{dsl}. The capture of
domain-specific knowledge is what sets \acsp{dsl} apart from general-purpose
programming languages. Domain-specific abstractions create opportunities for
domain-specific \textit{interpretation and transformation} (e.g., optimization,
analysis, error checking, tool support, etc.) \cite{Czarnecki2005}. For example,
with the symbol for ``tolerable load,'' we have information about the structure
of the symbol itself: that ``q'' has a ``hat'' and a subscript ``tol.'' From
this, we can output the same information in alternative flavours if we desired,
such as plain text, or with Java-compatible naming convention (e.g.,
``qHatTol'').

Drasils \acs{srs} template contains more ``holes''\footnote{Or ``blanks'' if you
      think of the template as a ``fill-in-the-blanks'' puzzle.} for other
information necessary to creating a whole ``story'' about how
\textit{output variables} can be calculated according to a set of
\textit{input variables} and algorithm derived through a series of
\textit{theories}. With sufficient knowledge \textit{depth}\footnote{Note
      that the \acs{srs} template provides the \textit{breadth} needed by
      design!} for each relevant fragment, Drasil is able to automatically
``check'' it for consistency and coherence, and generate representational
code\footnote{``Representational code'' meaning software that solves the
      problem the related \acs{srs} abstraction describes, using the algorithm
      outlined.}.

Unfortunately, not all of Drasils case studies are capable of generating
representational code (\refCaseStudiesCodeTable{}). Some case studies
(\acs{gamephysics}, \acs{hghc}, and \acs{ssp}) are still actively being
developed, but are left incomplete at the time of writing. \acs{nopcm} is usable
in all languages supported by \acs{gool} except for Swift due to the lack of a
Drasil-supported \acs{ode} solving library for the Swift \acs{gool} renderer.
\acs{pdcontroller} was built \cite{DrasilPR2289Naveen} outside  the normal means
of Drasils case studies development. Code generation for \acs{pdcontroller} is
not impossible, it just requires more investigation by a domain expert for the
needs of the case study. However, both the issues related to \acs{nopcm} and
\acs{pdcontroller} are outside the scope of this work. In this work, we will
focus on a critical common denominator between all examples: capturing
mathematical knowledge for reliable \acs{srs} artifact generation. In
particular, we will focus on 2 primary aspects of mathematical knowledge: the
\textit{theories} and the \textit{expressions}.

\caseStudiesCodeTable{}
