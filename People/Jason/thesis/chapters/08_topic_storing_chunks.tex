\chapter{\textquotedblleft{}Store All The Things\textquotedblright{}}
\label{chap:storingChunks}

\begin{writingdirectives}

      \item \textit{Problem}: Drasil can have any number of \textit{types} of
            chunks involved in any particular generation operation. While the
            types are useful in many regards, the types encumber us when we try
            to reference a specific chunk (by its \UID{}) without surrounding
            type information, which may not be important.

      \item Drasil's \ChunkDB{} (\refOriginalChunkDBHaskell) collects chunks
            using a series of maps, one for each chunk type.

      \item \textit{Practical Problems}:
            \begin{itemize}

                  \item Does not allow us to ignore type information when it's
                        not really needed. i.e., in areas where we use bounded
                        polymorphism.

                  \item Scaling against new types of chunks.
                  
                  \item Scaling against type parameters.
                  
                  \item Data consistency.
                  
                  \item Can't properly answer ``what chunk does X UID refer
                        to?'' No definitive answer to examining a specific chunk
                        by its \UID{}.

            \end{itemize}

      \item \textit{Solution}: Without discard, mask the type information,
            ignoring differences in the chunks other than that which is
            minimally required of them, and unmask it as needed.

      \item \textit{Practical Solution}: We can use \ExistentialQuantification{}
            with new data type that can carry anything that provides a minimum
            needed functionality (i.e., providing \UID{}s for their own data,
            and for references). In doing this, we effectively ``drop the type
            information'' when it's placed into the chunk database. However, we
            allow for that chunk to be unpacked (with type information retained)
            later through using Haskell's \inlineHs{Data.Typeable} (part of
            \inlineHs{base} package). As such, we can merge the series of maps
            into a single map that collects all chunks.

\end{writingdirectives}

In Drasil vernacular, as discussed in \Cref{chap:drasil}, we call encoded
fragments of knowledge, ``chunks.'' In practice, users encode chunks using
Haskells basic \inlineHs{data} types (example later), tagging it with a
\ACF{uid}. \acsp{uid} are represented by the \UID{} data type (a string of text,
\refOriginalUidHaskell{}). For example, \refOriginalQuantityDictExampleHaskell{}
has the manually defined \acs{uid}: \textit{tolLoad}.

More importantly, all chunks also have a specific \textit{type} they belong to.
For example, \refOriginalQuantityDictExampleHaskell{} is a \QuantityDict{}
(\refOriginalQuantityDictHaskell{}). Drasil leverages Haskells type system to
create a system of reasoning about chunks. As such, when we create instances of
chunks, they are ``typed'' with a single fully monomorphic type signature. There
is no hard criteria yet for knowing what \inlineHs{data} types are chunks or
not, but we do have basic requirements: they must have a \acs{uid}, they must
expose, in some way, what chunks they depend on, and they must be usable in some
way. Commonly, we create chunks to discuss relations between other chunks. To
discuss the relationships, we refer to related chunks by their \acsp{uid} rather
than the whole constituent chunk to ensure that all references are the same with
respect to the global set of \acsp{uid}.

We call the global set of chunks (and \acsp{uid}) the ``chunk database.'' The
chunk database (\ChunkDB{}, \refOriginalChunkDBHaskell{}) is built using a
series of maps (\refOriginalChunkDBTypeMapsHaskell) that map \UID{} keys to
chunk data. Each map is a map from a \UID{} to a specific data monomorphic data
type. As such, in order to access a chunk by its \UID{}, the associated type of
the chunk must be known, even if operations we intended to use on the chunk are
generic, applicable to any similar category of chunks (such as those satisfying
a particular typeclass constraint set) or all chunks. Additionally, since there
is one map per chunk type, all chunk types must be known and coded in the
\ChunkDB{} beforehand. Of course, this means the set of types that Drasil can
work with is fixed and difficult to extend. For example, if we have \(n\) types
(with no type parameters), then there must be \(n\) maps in the \ChunkDB{} data
type. However, for each chunk type with type parameters, each argument
combination would need to also be registered. Together, this leads to an
ever-growing series of maps, which scales poorly against new knowledge being
``taught'' to (encoded in) Drasil. Additionally, since we have a collection of
maps, ensuring \UID{} collisions never occur becomes tedious.

\originalChunkDBHaskell{}

\section{Scaling Against New Types}
\label{chap:storingChunks:sec:scaling-against-new-types}

Fundamentally, the issue with the \ChunkDB{} is that it caters to each
individual type by creating a map for each type, leading to tedious busy work to
make up for inflexibility.

In order to scale against new types, we need to make the database maps
type-agnostic, merging them all together. It should contain no hard-coded
references to any specific type, but describe what kinds of types are admissible
(which should be any valid chunk). However, this isn't to say that we should
erase all types. Rather, we should mask the type information for the purpose of
storage and minimum viable usage, but ensure that the type information can be
unmasked for retrieval and normal usage. Thus, the question arises: how can we
achieve this?

To collapse the series of maps into a single one, we need to have a common
``box'' data type that can be used as the value, which carries the actual chunk
data. Notably, we need a box that retains type information that we can use to
unbox them to use the chunks as normally.

\badUniversalChunkCarriage{}

Now, we have a common data type to wrap our data in: \Chunk{}. The type
parameter exposes type information for us to be able to unpack the \Chunk{} box
back to its actual chunk content, so that we may use it as normal. However, this
type parameter is also an issue for storage, we run into the same issue of
needing to create a series of maps, one for each type:

\showingBadChunkWouldFailChunkDB{}

Hence, carrying type information as a parameter is not the solution. We need to
have a single monomorphic data type that we can use to carry all chunk data
along with their type information. The type information allows us to
\textit{safely} attempt to cast the data back to its original chunk form, before
being placed into the \ChunkDB{}.

Thankfully, \acs{ghc} provides us with the ``Existential Quantification''
language feature \cite{GHC2020ExistentialQuantification}\footnote{We can enable
it by placing a small compiler option at the top of our Haskell files (e.g.,
\mintinline{haskell}{{-# LANGUAGE ExistentialQuantification #-}}), or, in our
\texttt{package.yaml} files}. ``Existential Quantification'' allows us to
quantify over the type parameter in the \Chunk{} constructor while retaining the
context (the type information). In  particular, this is useful for heterogeneous
lists with elements, sub-typed against a set of constraints, in Haskell (which
isn't normally encouraged, nor directly possible). In other words, it is useful
because it allows us to ignore the type while retaining the value and a common
basic functionality to all sub-typed elements through accessing the structural
information through a set of constraints. With this, we can re-create \Chunk{}
using \ExistentialQuantification{} to hide the type parameter:

\voidDataChunks{}

Great! Now we can create a single map for our chunk collection.

\voidDataChunkDB{}

Ok, now we've created a mechanism to collapse all of our chunk maps into one,
but, we've encountered an issue: we've neglected retrieval functionality. Let's
see:

\brokenChunkRetriever{}

\Chunk{}s are currently an informational void. We have the type information
accessible, but no way to use that type information to restore the chunk data to
its original typed form. So, this doesn't quite work yet. For example, in
\refChunkRetrieverError{}, Haskell is unable to coerce the rigid parameter type
\(a\) into the \QuantityDict{} type.

\chunkRetrieverError{}

So, how can we try to fix this? \inlineHs{Data.Typeable} \textit{to the rescue!}
With \inlineHs{Data.Typeable}, we're able to work with any piece of data given
to us without being provided any information about the data directly.
\inlineHs{Data.Typeable} allows us to create constrained types with extra
functionality for casting (amongst other common reflection operations). First,
we need to alter our \Chunk{} data type to make the contained data satisfy the
\Typeable{} constraint that the \Typeable{} module needs for most of its
functionality. As of \acs{ghc} 7.10 (for which our targeted version of 8+ is
newer than), \acs{ghc} automatically instantiates the \Typeable{} typeclass for
all types.

\examinableChunkBox{}

Ok, now we should be able to retrieve chunks, and cast the chunk value types as
needed.

\workingChunkRetriever{}

To sum up, at this point:

\begin{itemize}

      \item We were able to mask individual chunk types by hiding them in
            \Chunk{} boxes, allowing us to avoid discussing specific chunk types
            when forming our \ChunkDB{}.

      \item \ChunkDB{} is now merely a single map that allows us to easily
            ensure that there are no \UID{} collisions.

      \item \ChunkDB{} scales against new chunk type creation because it
            discusses no specific types as it only discusses types through
            constraints. Through this, it scales against the creation of type
            parameters for Haskell-level types.

\end{itemize}

Now, let's impose restrictions on the admissible chunks, to ensure that they are
indeed ``admissible,'' namely we desire the following restrictions:

\begin{enumerate}
      
      \item They should contain \UID{}s.
      
      \item They should know what chunks they directly depend on.

\end{enumerate}

One implicit contract we had with the chunks in the past, was that they were all
expected to have \UID{}s. Now, we can try to make this contract more explicit
through imposing typeclass constraints. For example, \refUidOwnershipContract{}
is one possible way.

\uidOwnershipContract{}

Afterwards, we would alter our \Chunk{} definition
(\refChunksWithUidConstraint{}) to add it.

\chunksWithUidConstraint{}

Great! Now all chunks must ``have a \acs{uid}'' (exposed via their \HasUID{}
instances), but we don't have a guarantee that they are unique and owned solely
by a single instance of any chunk. This is fine for now, and is left for future
work as part of future work (\Cref{chap:futureWork:sec:chunks}).

For the last restriction we want to impose, we want to ensure that chunks are
always ``understandable'' in a sense that all chunks they depend exist and have
already been registered in the chunk database as well. In other words, all
information they depend on should have already been entered before they are
entered.

\chunkDependenciesContract{}

Once again, we would alter our \Chunk{} definition accordingly
(\refChunksWithUidAndRefListConstraint{}).

\chunksWithUidAndRefListConstraint{}

Adding restrictions on all chunks is now also considerably simpler\footnote{We
can also wrap up the constraints together under a single constraint by using the
``constraint kinds'' language extension \cite{GHC2020ConstraintKinds} (e.g.,
\refProtoIsChunkHaskell{}).}. If we're interested, we can add a ``dump to X''
mechanism, where $X$ is any encoding of our choice (such as \acs{json}),
similarly. This is omitted for brevity, but we chose to add this. For now, this
chunk structure will be our final specification of a chunk: an arbitrary
\textit{thing} that has a type which indicates how it's usable, a \acs{uid}, and
a list of \textit{things} it references (and hence depends on) to be
sufficiently ``understood.''

Now, we may return to discussing the \ChunkDB{} structure. We often perform
actions (e.g., generation, validation, etc.) on all chunks of a particular type
or belonging to a set of types. Thus, \textit{retrieval} by type is a needed
feature of \ChunkDB{}s. Originally, this was done by selecting the right map and
using it normally. However, with the upgraded variant of chunk databases
(\refProtoChunkDBHaskell{{}}), this is not as simple, albeit more flexible. We
can re-create this for the new \ChunkDB{} as well by using
\inlineHs{Data.Typeable}s \TypeRep{} signature representation of Haskell types
(\refPseudoGrabAllChunksFromNewChunkDB{}).

\pseudoGrabAllChunksFromNewChunkDB{}

However, this is an expensive operation because it tests every chunk registered
in the chunk database. Since retrieval is a commonly performed task, it will
slow down Drasil. So, in order to make the \ChunkDB{} more
``industrial-strength,'' we can add an extra mechanism for caching (in various
ways) to trade a bit of memory for a frequent and expensive search operation:
\refProtoChunkDBHaskell{}.

\protoChunkDBHaskell{}

Now, we have extra functionality for caching by \TypeRep{}s without needing to
calculate it every time we need it, saving \acs{cpu} time. Additionally, we
added a traceability matrix to quickly find which chunks depend on which (direct
ones only, indirect ones can be calculated later if needed)\footnote{i.e., a
mechanism for listing which chunks depend on a specific chunk.}.

\protoTypedUIDHaskell{}

With this upgraded \ChunkDB{}, we can also now build typed \acs{uid} references:
\refProtoTypedUIDHaskell{}. These are useful for chunk-level assurance that
\acsp{uid} relate to something of an expected type.

\ChunkDB{} is a usable core across Drasil-like projects (ones thriving on the
same ``knowledge-based programming'' ideology). At the moment, the database
prototype lies in a separate Haskell project. Work on incorporating the database
in Drasil has been halted due to \acs{uid} conflicts arising in the case
studies. The port lies in both a \porthref{separate
repository}{https://github.com/balacij/ProtoChunkDB} (in prototype form) and a
\porthref{separate
branch}{https://github.com/JacquesCarette/Drasil/tree/newChunkDB} on Drasils git
repository, but can only be merged into the stable branch once we resolve some
\acs{uid} conflicts.

\matryoshkaDollsImg{}

The \acs{uid} conflicts occur because of how they were structured. The chunks
follow a ``gradual building'' pattern, similar to Matryoshka nesting
dolls (\refMatryoshkaDollsImg{}).

\pseudoChunkNestingHaskell{}

For example, in \refPseudoChunkNestingHaskell{}\footnote{Which does not exist in
Drasil, but is here purely for explanation purposes.}, chunks rely on a common
data type for their \acs{uid}. In Drasil, this is a common structure for
building up chunks using smaller ``inner'' chunks to quickly get some basic
features in a chunk, such as a name, short name, or abbreviation. In the
previous generation of \ChunkDB{}, \acs{uid} sharing across types is allowed
because there was no \UID{} uniqueness checking across types. However, the new
\ChunkDB{}s strictly require \UID{} uniqueness across chunk types because there
is only one ``chunk'' map under it. Unfortunately, this leads to compatibility
issues. As such, the structure of chunk building comes in to question. They're
currently built in such a way that relies on assuming that certain chunk types
exist and are related to other chunk types that share a \UID{} with them. In
other words, a ``whole chunk'' is the union of the chunks with the same \UID{}
spread across the other chunk maps. There are many ways that this can be
resolved, all depending on the needs of ``chunk building.'' For example, there
are at least three (3) possible resolutions: 
\begin{inparaenum}[(i)]
      
      \item forcibly/naively changing the \UID{}s, creating a strict distinction
            between them, 

      \item forcibly/naively collapsing the chunk structure, and 
      
      \item restructuring the chunk structure against the chunks, possibly
            learning how to re-create the easy chunk building style through
            other features.

\end{inparaenum}
However, a solution has not yet been decided because this issue is largely out
of scope, and requires considerable work. 
