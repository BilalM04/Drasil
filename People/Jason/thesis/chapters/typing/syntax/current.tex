An idealized version of the current syntax.

\startSyntaxTable
    \newsyntaxRow{Type}{\tau}{Integer}{\bb{Z}}{Integer numbers}
    \syntaxRow{Real}{\bb{R}}{Real numbers}
    \syntaxRow{String}{String}{Text}
    \syntaxRow{Bool}{\bb{B}}{Truth values (true/false)}
    \syntaxRow{Vector($\tau$, $n$)}{[\tau]_n}
        {Vectors (single element type, fixed length)}
    \syntaxRow{Tuple($\tau_1...\tau_n$)}
        {\tau_1 \times \tau_2 \times ... \times \tau_n}
        {Alternative vectors/tuples (fixed length+differently typed elements)}
     \\

    \newsyntaxRow{Literal}{l}{Integer[$n$]}{n}{Integer number}
    \syntaxRow{Real[$r$]}{r}{Real number}
    \syntaxRow{String[$s$]}{``s''}{Text}
    \syntaxRow{Bool[$b$]}{b}{Boolean value}
    \syntaxRow{Vector($l_1...l_n$)}{<l_1, ..., l_n>}{Vectors}
     \\

    \newsyntaxRow{UnaryOp}{\ominus}{Not}{\lnot \_}{Logical negation}
    \syntaxRow{Neg}{- \_}{Numeric negation}
    \syntaxRow{...}{}{}
     \\

    \newsyntaxRow{BinaryOp}{\oplus}{Sub}{\_ - \_}{Subtraction}
    \syntaxRow{Pow}{\_^\_}{Powers}
    \syntaxRow{...}{}{}
     \\
    
    \newsyntaxRow{AssocBinOp}{\otimes}{Add}{\_ + \_}{Addition}
    \syntaxRow{Mul}{\_ \times \_}{Multiplication}
    \syntaxRow{...}{}{}
     \\

    \newsyntaxRow{UID}{u}{UID(s)}{\texttt{UID ``s''}}{UIDs}
     \\

    \newsyntaxRow{Expr}{e}{Literal($l$)}{l}{Literal values}
    \syntaxRow{Vector($e_1...e_n$)}{<e_1, ..., e_n>}{Vectors}
    \syntaxRow{Var($u$)}{u}{Variable (QuantityDict Chunk)}
    \syntaxRow{FuncCall($f, e_1...e_n$)}{f(e_1...e_n)}
        {``Complete'' function application}
    \syntaxRow{UnaryOp($\ominus$,$e$)}{\ominus\ e}{Unary operations}
    \syntaxRow{BinaryOp($\oplus$,$e_1$,$e_2$)}{e_1 \oplus e_2}{Binary operations}
    \syntaxRow{AssocOp($\otimes$, $e_1...e_n$)}{e_1 \otimes ... \otimes e_n}
        {Associative binary operations}
    \syntaxRow{Case($e_{1c}e_{1e}...e_{nc}e_{ne}$)}
        {if\ e_{1c}\ then\ e_{2e}\ elif\ e_{2c}\ ...}
        {If-then-else-if-then-else (Switch-like statements)}
    \syntaxRow{BigAsBinOp($\otimes, e_1, e_2$)}{\bigotimes_{i=e_1}^{e_2} i}
        {Apply a ``big'' op to a discrete range defined as a range}
    \syntaxRow{IsInRlItrvl($u, e_1, e_2$)}{u \in [e_1, e_2]}{Variable in range}

\closeSyntaxTable
