\todo{tidy up}

With a focus on building Scientific Computing Software, Drasil is an
exploration of this idea. Rather than building one's software project in any
single or combination of general purpose programming language, the usage of a
sequence of domain-specific languages together in Drasil-based projects can be
used to describe common undergraduate level physics models and problems and
describe the target program that simulates said models.

\begin{itemize}
    \item Website: \url{https://jacquescarette.github.io/Drasil/}
    \item Hosted at: \url{https://github.com/JacquesCarette/Drasil}
    \item Wiki: \url{https://github.com/JacquesCarette/Drasil/wiki}
        \begin{itemize}
            \item Workspace configuration instructions: \url{https://github.com/JacquesCarette/Drasil/wiki/New-Workspace-Setup}
        \end{itemize}
    \item Principal investigators: Dr. Jacques Carette \& Dr. Spencer Smith
\end{itemize}

\section{Focus}

\begin{itemize}
    \item Primarily for undergraduate-level science (primarily physics) problems
    \item Generating scientific software artifacts
        \begin{itemize}
            \item Develop a stable knowledge base for physics problems
            \item Develop a stable framework for laying cookie-cutter problems.
            \item Make it as simple as using a projectional editor.
        \end{itemize}
\end{itemize}

\section{Methodology}

\begin{itemize}
    \item Approach to encoding knowledge -- e.g., ``bottom-up''
\end{itemize}

\section{Architecture}

\begin{itemize}
    \item Chunks
    \item ChunkDB
\end{itemize}

\section{State}
todo

\subsection{Short-term problems -- leading into topics}
todo
