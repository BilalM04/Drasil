\chapter{Future Work}
\label{chap:futureWork}

While this work does contribute to the Drasil research project and the ideology
underpinning it, there are still many questions and concerns left unanswered.

\section{Theories}
\label{chap:futureWork:sec:theories}

While \Cref{chap:modelkinds} improves upon the mathematical expression-directed
method of encoding ``theory knowledge,'' it is not without it's own issues.
Namely, the current implementation has two focal problems that we hope to
resolve in the future: extensibility, and the questionable type parameter
belonging to \ModelKind{} (\refCurrentModelKindsHaskell{}).

As \ModelKinds{} is written using a GADT, it is difficult to extend its
functionality (similar to the ``expression problem''
\cite{Wadler1999ExpressionProblem}). A work in progress re-design of
\ModelKinds{} \cite{DrasilIssue2853AlternativeModelKinds}, relying on \acs{ghc}s
\ConstraintKinds{} \cite{GHC2020ConstraintKinds} language extension, uses
Haskell-level typeclasses to describe what a model satisfying the needs of being
a ``model kind'' must provide in order to be treated as a ``model kind''. These
typeclass instances may be thought of as proofs that satisfy the requirements.
The primary benefit to adding extensibility to theory kinds is exactly that we
become able to add any kind of \textit{previously unknown to Drasil} theorem
without needing to touch too much other code. It also allows us to restrict
\textit{kinds} of theories to particular areas better.

Continuing, the type parameter in \ModelKinds{} is peculiar. It is used to
indicate usability of a certain model in code generation through either having
an \Expr{} or a \ModelExpr{}. A \inlineHs{ModelKind Expr} indicates that a model
is usable in code generation, while a \inlineHs{ModelKind ModelExpr} indicates
that the model is not to be used in code generation.

However, these two (2) issues are relatively unimportant for now. The issue of
extensibility has a proof of concept solution (as discussed above) already
constructed, and the issue of the type parameter is resolvable as a side effect.

Less operationally important, there is current discussion of renaming
\ModelKinds{} to \TheoryKinds{} \cite{DrasilIssue2599RenamingModels}, as
``model'' is a heavily overloaded term, and ``theory'' is a guaranteed
discussion of abstraction, unlike ``model,'' which is typically instantiated to
some ``theory.''

\section{Chunks}
\label{chap:futureWork:sec:chunks}

While \Cref{chap:storingChunks} answers questions regarding \textit{storing}
more kinds of chunks and has created a basic set of constraints that all chunks
must satisfy (\HasUID{} and \HasChunkRefs{}), we've embedded their solutions in
Haskell code rather than Drasil itself. In order for solutions to be included
``in Drasil itself,'' we need to encode it such that Drasil allows users to
interact with them, dynamically, without Template
Haskell\cite{GHC2020TemplateHaskell} or other things ``deep'' in Haskell.
Similarly, we have mysterious \Typeable{} usage left unknown, which should
eventually be replaced with something well-understood to us. In the future, we
hope to improve chunk building fundamentally, perhaps by using a \acs{dsl}
instead of leaning on built-in Haskell functionality. This would allow us to
better analyze Drasil and its projects.

Furthermore, while we've added requirements that \acsp{uid} be unique, we
haven't discussed how \UID{}s should be built (automatic or manual, and how?),
nor ensured that \UID{} uniquely refer to the chunks they were intended to link
to. We desire for them to fully be \textit{rigid designators}
\cite{Kripke1972NandN}. Perhaps these questions will naturally resolve
themselves when we try to switch to using a Drasil \acs{dsl} to build chunks.

As described in \Cref{chap:storingChunks}, once we resolve the issue regarding
\UID{} collisions, we should be able to register more of our currently underused
chunks in our new \ChunkDB{}s. In doing this, we will be able to perform a wider
range of analysis on our necessary ``knowledge'' (chunks). For example, we
should be able to better understand how much mental effort is needed to produce
software artifacts.

\subsection{Math-specific Instances}
\label{chap:futureWork:sec:chunks:sub:mathSpecific}

In \Cref{chap:modelkinds,chap:typedExpr}, we discussed how we can improve the
reliability of mathematical language usage in Drasil in different facets. One
notable facet in practice is the unit and dimension of numbers. We hope that we
may, in the future, create a strong and reliable system for units and dimensions
for Drasil, allowing users to discuss precision and accuracy of the generated
\acs{scs} solver artifacts. Furthermore, in \Cref{chap:typedExpr}, we neglected
to ensure that symbols referenced, by the expression languages, are all
appropriate for the use-cases. For example, the \Expr{} should language should
only allow for reference of concrete and usable symbols, while \ModelExpr{}
should only allow for referencing of symbols of abstract symbols. Similarly,
\CodeExpr{} should only allow referencing of symbols that belong to ``code''
rather than to the mathematical models (for this, a switch from a
\QuantityDict{} is needed as well).

\subsection{Database}
\label{chap:futureWork:sec:chunks:sub:database}

Once the \ChunkDB{} prototype finds its way into Drasil, we may focus on the
ways we can manipulate them. For example, if we consider them sets we should
consider implementing ``union'' and ``intersection'' operations to assist in
importing and analyzing bodies of chunks. Additionally, if we consider them
``tables'' (as in common relational databases), we might benefit from ``inner''
and ``outer'' joins, or even using a \acs{sql} server as a backend to inputting
(or handling) chunks. Using an \acs{sql} server may prove beneficial as there
are already many \acs{gui} frontends for inputting \acs{rdbms} data.
Furthermore, with the new database structure, ``tree shaking'' should be a
fairly simple operation, reduced to searching for chunks that are unreachable
from the universe of references of an input set of chunks. For example, for the
\textit{SmithEtAl} template, we may consider tree shaking against the directly
captured instances of inputs and outputs to remove any unnecessary assumptions,
theories, symbols, etc.

Finally, one pain-point in the case studies is inputting the transcribed chunks
into the chunk database. At the moment, it requires manually inputting the data
into lists (\refCurrentExampleGatherChunkIntoListsHaskell{}), and then using the
lists to form the \ChunkDB{}
(\refCurrentExampleGatherChunkListsIntoDBHaskell{}). We could attempt to
automatically register the chunks in the database, either by analyzing Haskell
(meta-linguistic), or de-embedding the chunks from our Haskell code and
registering on parsing them \cite{DrasilIssue2873ChunkDBCaretteResponse}. 
