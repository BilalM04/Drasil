\chapter{Future Work}
\label{chap:futureWork}

While this work does contribute to the Drasil research project and the ideology
underpinning it, there are still many questions and concerns left unanswered.

\section{Chunks}

While \Cref{chap:storingChunks} answers questions regarding \textit{storing}
more kinds of chunks and has created a basic set of constraints that all chunks
must satisfy (\HasUID{} and \HasChunkRefs{}), we've embedded their solutions in
Haskell code rather than Drasil itself. In order for solutions to be included
``in Drasil itself,'' we need to encode it such that Drasil allows users to
interact with them, dynamically, without Template Haskell \todo{Cite something
about TemplateHaskell} or other things ``deep'' in Haskell. Similarly, we have
mysterious \Typeable{} usage left unknown, which should eventually be replaced
with something well-understood to us. In the future, we hope to improve chunk
building fundamentally, perhaps by using a \acs{dsl} instead of leaning on
built-in Haskell functionality. This would allow us to better analyze Drasil and
its projects.

Furthermore, while we've added requirements that \acsp{uid} be unique, we
haven't discussed how \UID{}s should be built (automatic or manual, and how?),
nor ensured that \UID{} references ultimately link to the chunks they were
intended to link to. We desire for them to fully be \textit{rigid designators}
\todo{Cite something about rigid designators.}. Perhaps these questions will
naturally resolve themselves when we try to switch to using a Drasil \acs{dsl}
to build chunks.

As described in \Cref{chap:storingChunks}, once we resolve the issue regarding
\UID{} collisions, we should be able to register more of our currently underused
chunks in our new \ChunkDB{}s. In doing this, we will be able to perform a wider
range of analysis on our necessary ``knowledge'' (chunks). For example, we
should be able to better understand how much mental effort is needed to produce
software artifacts.

\subsection{Math-specific Instances}

In \Cref{chap:modelkinds,chap:typedExpr}, we discussed how we can improve the
reliability of mathematical language usage in Drasil in different facets. One
notable facet in practice is the unit and dimension of numbers. We hope that we
may, in the future, create a strong and reliable system for units and dimensions
for Drasil, allowing users to discuss precision and accuracy of the generated
\acs{scs} solver artifacts.
