The essence of this ideology lies in naturally obtaining mechanization
techniques through formalizing \textit{everything}. The ideology forces us to
question modern software development practices: cognitive stress and normal
errors aside, if a developer doesn't \textit{truly understand} the knowledge
they are encoding in a software product, then it should be normal to expect
logical issues and an ``imperfect'' program. Taking cognitive stress into
consideration, there should be considerably less as knowledge only need be
transcribed once, and re-used infinitely. Through mechanization, cognitive
stress of re-writing knowledge is alleviated for all proceeding instances. A
formalized-knowledge-first approach to software development should highlight
areas of issue (poor understanding), never produce bugs, and create software
with the same quality as the encoded knowledge.

Concretely, this ideology demands that we systematically build programs by
describing them through specifications. In doing this, ``knowledge'' becomes
directly reusable, codified (and easily transferrable), traceable in the
artifacts, and allowed to be put to better use. Finally, ``quality'' of the
generate-able software products then becomes a clear reflection of the
``quality'' of the codified knowledge.
