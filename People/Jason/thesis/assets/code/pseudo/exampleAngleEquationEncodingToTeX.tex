\begin{pseudocode}{tex}{Example Angle Equation SRS Conversion}{exampleAngleEquationEncodingToTeX}
\begin{minipage}{\textwidth}
\begin{tabular}{...}
\toprule \textbf{Refname} & \textbf{IM:firingAngleFormula}
\phantomsection 
\label{IM:firingAngleFormula}
\\ \midrule \\
Label & Firing angle formula

\\ \midrule \\
Input & $\textit{targetDistanceFromCannon}$, $v$

\\ \midrule \\
Output & $\theta{}_c$
         
\\ \midrule \\
Input Constraints & --

\\ \midrule \\
Output Constraints & \begin{displaymath}
                     {\theta{}_c}\gt{}0
                     \end{displaymath}
\\ \midrule \\
Equation & \begin{displaymath}
           {\theta{}_c} = \frac{\arcsin{} (\frac{\textit{targetDistanceFromCannon} \cdot{} \mathbf{g}}{v^{2}})}{2}
           \end{displaymath}
\\ \midrule \\
Description & \begin{symbDescription}
              \item{${\theta{}_c}$ is the firing angle (${\text{rad}}$)}
              \item{${v}$ is the launch speed ($\frac{\text{m}}{\text{s}}$)}
              \item{$targetDistanceFromCannon$ is the distance between the cannon and target (${\text{m}}$)}
              \item{$\mathbf{g}$ is the gravitational acceleration ($\frac{\text{m}}{\text{s}^{2}}$)}
              \end{symbDescription}
\\ \midrule \\
Notes & --
        
\\ \midrule \\
Source & --
         
\\ \midrule \\
RefBy & ...

\\ \bottomrule
\end{tabular}
\end{minipage}
\paragraph{Detailed derivation of landing position:}
...
\end{pseudocode}
