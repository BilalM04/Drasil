\begin{haskell}{Original Expression language}{origExpr}{https://github.com/JacquesCarette/Drasil/blob/9c26b43d3e30c3f618e534a3f176a5152729af74/code/drasil-lang/Language/Drasil/Expr.hs\#L40-L82}
-- | Drasil Expressions
data Expr where
  Dbl      :: Double -> Expr
  Int      :: Integer -> Expr
  Str      :: String -> Expr
  Perc     :: Integer -> Integer -> Expr
  AssocA   :: ArithOper -> [Expr] -> Expr
  AssocB   :: BoolOper  -> [Expr] -> Expr
  -- | Derivative, syntax is:
  --   Type (Partial or total) -> principal part of change -> with respect to
  --   For example: Deriv Part y x1 would be (dy/dx1)
  Deriv    :: DerivType -> Expr -> UID -> Expr
  -- | C stands for "Chunk", for referring to a chunk in an expression.
  --   Implicitly assumes has a symbol.
  C        :: UID -> Expr
  -- | F(x) is (FCall F [x] []) or similar.
  --   FCall accepts a list of params and a list of named params.
  --   F(x,y) would be (FCall F [x,y]) or sim.
  --   F(x,n=y) would be (FCall F [x] [(n,y)]).
  FCall    :: UID -> [Expr] -> [(UID, Expr)] -> Expr
  -- | Actor creation given UID and parameters 
  New      :: UID -> [Expr] -> [(UID, Expr)] -> Expr 
  -- | Message an actor: 
  --   1st UID is the actor, 
  --   2nd UID is the method
  Message  :: UID -> UID -> [Expr] -> [(UID, Expr)] -> Expr
  -- | Access a field of an actor:
  --   1st UID is the actor,
  --   2nd UID is the field
  Field :: UID -> UID -> Expr
  -- | For multi-case expressions, each pair represents one case
  Case     :: Completeness -> [(Expr,Relation)] -> Expr
  Matrix   :: [[Expr]] -> Expr
  UnaryOp  :: UFunc -> Expr -> Expr
  BinaryOp :: BinOp -> Expr -> Expr -> Expr
  -- | Operators are generalized arithmetic operators over a |DomainDesc|
  --   of an |Expr|.  Could be called |BigOp|.
  --   ex: Summation is represented via |Add| over a discrete domain
  Operator :: ArithOper -> DomainDesc Expr Expr -> Expr -> Expr
  -- | element of
  IsIn     :: Expr -> Space -> Expr 
  -- | a different kind of 'element of'
  RealI    :: UID -> RealInterval Expr Expr -> Expr 
\end{haskell}
