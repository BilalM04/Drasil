\begin{haskell}{ModelExpr Language}{curModelExpr}{https://github.com/JacquesCarette/Drasil/blob/ab9e091dabd81685ddef86b0d218582c9f75cb20/code/drasil-lang/lib/Language/Drasil/ModelExpr/Lang.hs\#L82-L151}
-- | Expression language where all terms are supposed to have a meaning, but
--   that meaning may not be that of a definite value. For example,
--   specification expressions, especially with quantifiers, belong here.
data ModelExpr where
  -- | Brings a literal into the expression language.
  Lit       :: Literal -> ModelExpr
  
  -- | Introduce Space values into the expression language.
  Spc       :: Space -> ModelExpr
  
  -- | Takes an associative arithmetic operator with a list of expressions.
  AssocA    :: AssocArithOper -> [ModelExpr] -> ModelExpr
  -- | Takes an associative boolean operator with a list of expressions.
  AssocB    :: AssocBoolOper  -> [ModelExpr] -> ModelExpr
  -- | Derivative syntax is:
  --   Type ('Part'ial or 'Total') -> principal part of change -> with respect to
  --   For example: Deriv Part y x1 would be (dy/dx1).
  Deriv     :: Integer -> DerivType -> ModelExpr -> UID -> ModelExpr
  -- | C stands for "Chunk", for referring to a chunk in an expression.
  --   Implicitly assumes that the chunk has a symbol.
  C         :: UID -> ModelExpr
  -- | A function call accepts a list of parameters and a list of named parameters.
  --   For example
  --
  --   * F(x) is (FCall F [x] []).
  --   * F(x,y) would be (FCall F [x,y]).
  --   * F(x,n=y) would be (FCall F [x] [(n,y)]).
  FCall     :: UID -> [ModelExpr] -> [(UID, ModelExpr)] -> ModelExpr
  -- | For multi-case expressions, each pair represents one case.
  Case      :: Completeness -> [(ModelExpr, ModelExpr)] -> ModelExpr
  -- | Represents a matrix of expressions.
  Matrix    :: [[ModelExpr]] -> ModelExpr
  
  -- | Unary operation for most functions (eg. sin, cos, log, etc.).
  UnaryOp       :: UFunc -> ModelExpr -> ModelExpr
  -- | Unary operation for @Bool -> Bool@ operations.
  UnaryOpB      :: UFuncB -> ModelExpr -> ModelExpr
  -- | Unary operation for @Vector -> Vector@ operations.
  UnaryOpVV     :: UFuncVV -> ModelExpr -> ModelExpr
  -- | Unary operation for @Vector -> Number@ operations.
  UnaryOpVN     :: UFuncVN -> ModelExpr -> ModelExpr
  
  
  -- | Binary operator for arithmetic between expressions (fractional, power, and subtraction).
  ArithBinaryOp :: ArithBinOp -> ModelExpr -> ModelExpr -> ModelExpr
  -- | Binary operator for boolean operators (implies, iff).
  BoolBinaryOp  :: BoolBinOp -> ModelExpr -> ModelExpr -> ModelExpr
  -- | Binary operator for equality between expressions.
  EqBinaryOp    :: EqBinOp -> ModelExpr -> ModelExpr -> ModelExpr
  -- | Binary operator for indexing two expressions.
  LABinaryOp    :: LABinOp -> ModelExpr -> ModelExpr -> ModelExpr
  -- | Binary operator for ordering expressions (less than, greater than, etc.).
  OrdBinaryOp   :: OrdBinOp -> ModelExpr -> ModelExpr -> ModelExpr
  -- | Space-related binary operations.
  SpaceBinaryOp :: SpaceBinOp -> ModelExpr -> ModelExpr -> ModelExpr
  -- | Statement-related binary operations.
  StatBinaryOp  :: StatBinOp -> ModelExpr -> ModelExpr -> ModelExpr
  -- | Binary operator for @Vector x Vector -> Vector@ operations (cross product).
  VVVBinaryOp   :: VVVBinOp -> ModelExpr -> ModelExpr -> ModelExpr
  -- | Binary operator for @Vector x Vector -> Number@ operations (dot product).
  VVNBinaryOp   :: VVNBinOp -> ModelExpr -> ModelExpr -> ModelExpr
  
  
  -- | Operators are generalized arithmetic operators over a 'DomainDesc'
  --   of an 'Expr'.  Could be called BigOp.
  --   ex: Summation is represented via 'Add' over a discrete domain.
  Operator :: AssocArithOper -> DomainDesc t ModelExpr ModelExpr -> ModelExpr -> ModelExpr
  -- | A different kind of 'IsIn'. A 'UID' is an element of an interval.
  RealI    :: UID -> RealInterval ModelExpr ModelExpr -> ModelExpr
  
  -- | Universal quantification
  ForAll   :: UID -> Space -> ModelExpr -> ModelExpr
\end{haskell}
