% Command based on: https://tex.stackexchange.com/questions/266811/define-a-new-command-with-parameters-inside-newcommand
\newcommand{\codeName}[1]{\expandafter\newcommand\csname #1\endcsname{\inlineHs{#1}}}

\codeName{Chunk}
\codeName{ChunkDB}
\codeName{CodeExpr}
\codeName{ConceptChunk}
\codeName{ConstraintKinds}
\codeName{ConstraintSet}
\codeName{DataDefinition}
\codeName{DEModel}
\codeName{DefiningExpr}
\codeName{DifferentialModel}
\codeName{EquationalConstraints}
\codeName{EquationalModel}
\codeName{EquationalRealm}
\codeName{ExistentialQuantification}
\codeName{Expr}
\codeName{Express}
\codeName{GenDefn}
\codeName{HasChunkRefs}
\codeName{HasUID}
\codeName{IdeaDict}
\codeName{InstanceModel}
\codeName{Literal}
\codeName{ModelExpr}
\codeName{ModelKind}
\codeName{ModelKinds}
\codeName{MultiDefn}
\codeName{NewDEModel}
\codeName{ODEInfo}
\codeName{OthModel}
\codeName{QDefinition}
\codeName{QuantityDict}
\codeName{Relation}
\codeName{RelationConcept}
\codeName{relToQD}
\codeName{Space}
\codeName{Stage}
\codeName{Symbol}
\codeName{TheoryKind}
\codeName{TheoryKinds}
\codeName{TheoryModel}
\codeName{Typeable}
\codeName{TypeRep}
\codeName{UID}
\codeName{UnitDefn}

% Used for showing what the blue-highlighted text is, in the reading notes section
\codeName{ExampleText}
