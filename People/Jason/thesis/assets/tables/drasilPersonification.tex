\begin{longtable}[c]{|>{\raggedright}p{0.3\linewidth}|>{\raggedright\arraybackslash}p{0.54\linewidth}|}
    \caption{Drasil Logo Personification}
    \label{tab:drasilPersonification}
    \\

    \hline

    \rowcolor{McMasterMediumGrey}
    \textbf{Component}                 & \textbf{Conceptual Counterpart/Personification}

    \\ \hline

    {Roots}                            & {The roots are where the information of
            the seed influences the earth, and makes it comfortable for the tree
            to grow tall and firm. Information influences and encourages
            re-evaluation and structural change of the ground.}

    \\ \hline

    {Ground / Foundation}              & {The most important component, it is
            where the tree stands tall, and all knowledge relies on. It contains
            the definitions of the encodings, and is required to be strong or
            else a seed will be insufficient, irrelevant of how much topsoil is
            provided.}

    \\ \hline

    {Seed}                             & {The initial bundle of information,
            from which everything originates and derives from. You provide the
            bare minimum information to describe your problem, and use nutrients
            and care to carefully grow the seed into something else.}

    \\ \hline

    {Nutrients (topsoil and sunlight)} & {Encouraged growth through
            hinting/providing extra information. This is where you configure
            growth and encourage further growth externally, artificially.}

    \\ \hline

    {Trunk}                            & {The initial display of growth of the
            tree, building a wide knowledge-base. Sometimes requires maintenance
            (trimming, or, extra information/nutrients) to grow further and
            become a strong basis for the crown.}

    \\ \hline

    {Crown + Fruits}                   & {The fruits of your labour, standing on
            the shoulder of giants, where the final product (software artifacts)
            are realized.}

    \\ \hline
\end{longtable}
