%%%%%%%%%%%%%%%%%%%%%%%%%%%%%%%%%%%%%%%%%%%%%%%%%%%%%%%%%%%%%%%%%%%%%%%%%%%%%%%
% QUESTION DIRECTED WRITING
\ifshowwritingdirectives
  \newenvironment{writingdirectives}{\begin{mdwritingdirectives}\centering\textbf{Writing Directives}\begin{itemize}}{\end{itemize}\end{mdwritingdirectives}}
\else
  \excludecomment{writingdirectives}
\fi

\newcommand{\wqanswer}[1]{\textit{#1}}

%%%%%%%%%%%%%%%%%%%%%%%%%%%%%%%%%%%%%%%%%%%%%%%%%%%%%%%%%%%%%%%%%%%%%%%%%%%%%%%
% DOUBLE SPACING OPTIONS
\newcommand{\thesisForceSingleSpacing}{\singlespacing}
\newcommand{\thesisForceDoubleSpacing}{\doublespacing}

\ifdoublespaced
  \newcommand{\thesisSpacing}{\doublespacing}
\else
  \newcommand{\thesisSpacing}{\singlespacing}
\fi

%%%%%%%%%%%%%%%%%%%%%%%%%%%%%%%%%%%%%%%%%%%%%%%%%%%%%%%%%%%%%%%%%%%%%%%%%%%%%%%
% CASE STUDIES
\newcommand{\caseStudy}[1]{\ACL{#1} (\textit{\acs{#1}})}

%%%%%%%%%%%%%%%%%%%%%%%%%%%%%%%%%%%%%%%%%%%%%%%%%%%%%%%%%%%%%%%%%%%%%%%%%%%%%%%
% JUDGMENTS

\newcommand{\newrule}[2]{\begin{equation} \infer{#2}{#1} \end{equation}} % Adds to automatically numbered equations
\newcommand{\newlblrule}[3]{\begin{equation} \infer{#2}{#1} \label{#3}\end{equation}} % Adds to automatically numbered equations, with a label
\newcommand{\exampleRule}[2]{\[ \infer{#2}{#1} \]} % Does not add a number to equations

\newcommand{\Tau}{\mathrm{T}}
\newcommand{\ty}[1]{\texttt{#1} : \tau}

\newcommand{\ofTy}[2]{#1 : \texttt{#2}}

\newcommand{\numericTy}[1]{\ofTy{#1}{\texttt{Numerics($\Tau$)}}}
\newcommand{\negNumericTy}[1]{\ofTy{#1}{\texttt{NumericsWithNegation($\Tau$)}}}

%%%%%%%%%%%%%%%%%%%%%%%%%%%%%%%%%%%%%%%%%%%%%%%%%%%%%%%%%%%%%%%%%%%%%%%%%%%%%%%
% MATH

\newcommand{\bb}[1]{\mathbb{#1}}

%%%%%%%%%%%%%%%%%%%%%%%%%%%%%%%%%%%%%%%%%%%%%%%%%%%%%%%%%%%%%%%%%%%%%%%%%%%%%%%
% SYNTAX CHARTS
\newcommand{\startSyntaxTable}{\begin{longtable}{ r c c l c l }}
\newcommand{\newsyntaxRow}[5]{#1 & \( #2 \) & $::=$ & \texttt{#3} & $#4$ & #5 \\}
\newcommand{\syntaxRow}[3]{& & $\vert$ & \texttt{#1} & $#2$ & #3 \\}
\newcommand{\closeSyntaxTable}{\end{longtable}}

%%%%%%%%%%%%%%%%%%%%%%%%%%%%%%%%%%%%%%%%%%%%%%%%%%%%%%%%%%%%%%%%%%%%%%%%%%%%%%%
% Footnotes that only show "when compiling for printing"

\ifcompilingforprinting
  \newcommand{\printOnlyFootnote}[1]{\footnote{#1}}
  \newcommand{\printOnlyFootnoteText}[1]{\footnotetext{#1}}
  \newcommand{\printOnlyFootnoteMark}{\footnotemark}
\else
  \newcommand{\printOnlyFootnote}[1]{}
  \newcommand{\printOnlyFootnoteText}[1]{}
  \newcommand{\printOnlyFootnoteMark}{}
\fi

%%%%%%%%%%%%%%%%%%%%%%%%%%%%%%%%%%%%%%%%%%%%%%%%%%%%%%%%%%%%%%%%%%%%%%%%%%%%%%%
% Portable HREFs

% Common variant
\newcommand{\porthref}[2]{\href{#2}{#1}\printOnlyFootnote{\url{#2}}}
% Custom URLs
\newcommand{\porthreft}[3]{\href{#3}{#1}\printOnlyFootnote{\href{#3}{#2}}}
% Inside of some environments, footnote marks aren't registered properly, so we
% need to manually write the "text" part
\newcommand{\porthreftm}[2]{\href{#2}{#1\printOnlyFootnoteMark}}

%%%%%%%%%%%%%%%%%%%%%%%%%%%%%%%%%%%%%%%%%%%%%%%%%%%%%%%%%%%%%%%%%%%%%%%%%%%%%%%
% Inlined TODOs
\newcommand{\intodo}[1]{\todo[inline]{#1}}
\newcommand{\imptodo}[1]{\todo[inline,backgroundcolor=orange]{#1}}

%%%%%%%%%%%%%%%%%%%%%%%%%%%%%%%%%%%%%%%%%%%%%%%%%%%%%%%%%%%%%%%%%%%%%%%%%%%%%%%
% Haskell snippet
\newenvironment{code}{\captionsetup{type=listing,skip=14pt}}{}
\SetupFloatingEnvironment{listing}{name=Source Code, listname=List of Source Codes}
\crefname{listing}{source code}{source codes}
\Crefname{listing}{Source Code}{Source Codes}
\newenvironment{haskell}[4]
    {\VerbatimEnvironment\thesisForceSingleSpacing{}\begin{code}\captionof{listing}[#1]{\protect\porthreftm{#1}{#4}}\printOnlyFootnoteText{\protect\href{#4}{#3}}\label{lst:#2}\begin{minted}[frame=lines,framerule=2pt,breaklines]{haskell}}
    {\end{minted}\end{code}\thesisSpacing{}}

\newcommand{\inlineHs}[1]{\mintinline{haskell}|#1|}

% TODO: Look into 'xurl' instead of the above hacky solution: <https://tex.stackexchange.com/questions/54946/how-to-break-a-long-url>

% Haskell env original caption style: {\VerbatimEnvironment\begin{code}\caption[#1]{\protect\porthreftm{#1}{#4}}\footnotetext{#3}\label{lst:#2}\begin{minted}[frame=lines,framerule=2pt]{haskell}}

% TODO: Figure out how I can properly space the label between "Source Code X.X:
% X" and the minted code itself. "\vspace*{-3mm}" barely works, and doesn't
% really solve the real problem here.
