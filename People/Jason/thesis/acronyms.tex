% For one-offs,
% \DeclareAcronym{acronym}{short=short-version,long=long-version}

\newcommand{\newacr}[2]{\DeclareAcronym{#1}{short=\uppercase{#1},long=#2}}

% Alphabetically sorted list of acronyms
\newacr{adt}{algebraic datatype}
\newacr{api}{application programming interface}
\newacr{ast}{algebraic syntax tree}
\newacr{cms}{content management system}
\newacr{dsl}{domain-specific language}
\newacr{ffi}{foreign function interface}
\newacr{gadt}{generalized algebraic datatype}
\newacr{ghc}{glasgow haskell compiler}
\newacr{gool}{generic object-oriented language}
\newacr{kms}{knowledge management system}
\newacr{oo}{object-oriented}
\newacr{scs}{scientific computing software}
\newacr{srs}{software requirements specification}
\newacr{ttf}{typed tagless final}
\newacr{uid}{unique identifier}

%------------------------------------------------------------------------------
%- Extra commands for more functionality -- in particular, capitalizing the
%- long form of acronyms.
%------------------------------------------------------------------------------

% Defining \ACL - to capitalize all words in an acronym
% Credits to: https://tex.stackexchange.com/a/257896
\NewDocumentCommand\ACF{sm}{%
  \begingroup
    \acsetup{uppercase/cmd=\ecapitalisewords}%
    \IfBooleanTF{#1}{\Acf*{#2}}{\Acf{#2}}%
  \endgroup
}

\NewDocumentCommand\ACFP{sm}{%
  \begingroup
    \acsetup{uppercase/cmd=\ecapitalisewords}%
    \IfBooleanTF{#1}{\Acfp*{#2}}{\Acfp{#2}}%
  \endgroup
}

\NewDocumentCommand\ACL{sm}{%
  \begingroup
    \acsetup{uppercase/cmd=\ecapitalisewords}%
    \IfBooleanTF{#1}{\Acl*{#2}}{\Acl{#2}}%
  \endgroup
}

\NewDocumentCommand\ACLP{sm}{%
  \begingroup
    \acsetup{uppercase/cmd=\ecapitalisewords}%
    \IfBooleanTF{#1}{\Aclp*{#2}}{\Aclp{#2}}%
  \endgroup
}
