\chapter{Connect Model to Libraries}
In chapter 2, we stored the information of a higher-order linear ODE in the \verb|DifferentialModel|. This data type preserve the relationship of the ODE, and we can transform the ODE to other forms. In chapter 3, we discuss how to solve a system of first-order ODEs numerically in four selected external libraries. What we have not discussed is that how to close the gap between the \verb|DifferentialModel| and external libraries so the Drasil framework can generate a program to solve a high-order ODE numerically. All external libraries do not understands what is a \verb|DifferentialModel|, so this is the problem we are going to solve in this chapter. What we know is that most way for solving ODEs are intended for first-order ODEs, and we can covert most higher-order ODEs to a system of first-order ODEs~\citep{converthigherode}. Therefore, the methodology we use in this chapter is first transform a higher-order linear ODE to a system of first-order ODEs. Then, using the external libraries to solve the system of first-order ODEs. The original higher-order linear ODE is equivalent to the system of first-order ODEs. Thus, by transitivity, the numerical solution for the system of first-order ODEs is also the numerical solution of the higher-order ODE.

In this chapter, we will first discuss how to convert a higher-order linear ODE to a system of first order equations in theory. Then, we will talk about how to connect explicit equation with external libraries. Last, we will discuss how to generate explicit equation base on \verb|DifferentialModel|.

\section{Higher Order to First Order}
\label{se_hightofirst}
Given a higher-order ODE, we can write it in form of the equation~\ref{eq_isohighode}. We put the highest derivative $y^n$ on the left hand side, and the rest of terms on right hand side. On the right hand side, $f (t, y, y', y'', \dots, y^{n-1})$ means a function depends on variables t, y, y', \dots, and $y^{n-1}$. The t is independent variable and it is time. The y, y', \dots, and $y^{n-1}$ means the dependent variable y, first derivative of y, and until n-1 derivative.
\begin{equation} \label{eq_isohighode}
  y^n = f (t, y, y', y'', \dots, y^{n-1})
\end{equation}

Later, we introduce new variables, $x_{1}$, $x_{2}$, $\dots$, and $x_{n}$. The number of newly introduced dependent variable is equal to the highest order of the ODE. The new relationship show at below.
\begin{flalign} \label{eq_newvars}
  & x_{1} = y \\ \nonumber
  & x_{2} = y' \\ \nonumber
  & \dots \\ \nonumber
  & x_{n} = y^{n-1} 
\end{flalign}

Now, we can start differentiate $x_{1}$, $x_{2}$, $\dots$, and $x_{n}$ in equation~\ref{eq_newvars}. Then, we get new relationship between each variable.
\begin{flalign} \label{eq_diffvervars}
  & x_{1}' = y' = x_{2} \\ \nonumber
  & x_{2}' = y'' = x_{3} \\ \nonumber
  & \dots \\ \nonumber
  & x_{n-1}' = y^{n-1} = x_{n}\\ \nonumber
  & x_{n}' = y^{n} = f (t, x_{1}, x_{2}, \dots, x_{n})
\end{flalign}

We assume $f (t, x_{1}, x_{2}, \dots, x_{n})$ is a linear function. Then, we can rewrite them as the following
\begin{equation}\label{eq_linear}
b_{0}(t) \cdot x_{1} + b_{1}(t) \cdot x_{2} + \dots + b_{n-1}(t) \cdot x_{n} + h(t)
\end{equation}

Based on the equation~\ref{eq_diffvervars}, we can simplify equations by removing derivatives of x and replace $f (t, y_{1}, y_{2}, \dots, y_{n})$ with equation~\ref{eq_linear}. Then, we can get:
\begin{flalign} \label{eq_diffvervarslinear}
    & x_{1}' = x_{2} \\ \nonumber
    & x_{2}' = x_{3} \\ \nonumber
    & \dots \\ \nonumber
    & x_{n-1}' = x^{n}\\ \nonumber
    & x_{n}'= b_{0}(t) \cdot x_{1} + b_{1}(t) \cdot x_{2} + ... + b_{n-1}(t) \cdot x_{n} + h(t)
\end{flalign}

Last, we can rewrite Equation~\ref{eq_diffvervarslinear} in a matrix form.
\begin{equation} \label{eq_foodeexample}
	\begin{bmatrix}
		x_{1}' \\
    x_{2}' \\
    \dots  \\
    x_{n-1}' \\
    x_{n}'
	\end{bmatrix}
    = 
  \begin{bmatrix}
		0, & 1, & 0, & \dots, & 0 \\
    0, & 0, & 1, & \dots, & 0 \\
    \dots \\
    0, & 0, & 0, & \dots, & 1 \\
    b_{0}(t), & b_{1}(t), & b_{2}(t), & \dots, & b_{n-1}(t)
	\end{bmatrix}
    \cdot
  \begin{bmatrix}
		x_{1} \\
    x_{2} \\
    \dots  \\
    x_{n-1} \\
    x_{n}
	\end{bmatrix}
    + 
  \begin{bmatrix}
    0 \\
    0 \\
    \dots  \\
    0 \\
    h(t)
	\end{bmatrix}
\end{equation}

Lastly, we abstract Equation~\ref{eq_foodeexample} into a general form, Equation~\ref{eq_foode}. The \textbf{A} is a coefficient matrix, and \textbf{c} is a constant vector. The \textbf{X} is the unknown vector contains functions of the independent variable, often time. The \textbf{X}' is a vector that consist of first derivative of functions in \textbf{X}.
\begin{equation} \label{eq_foode}
    \boldsymbol{X}' = \boldsymbol{AX} + \boldsymbol{c}
\end{equation}

Once we completed the transformation and get a system of first-order ODE in shape of Equation~\ref{eq_foode}. We encode it in Drasil and pass it to Drasil Code Generator. In the \href{https://jacquescarette.github.io/Drasil/examples/pdcontroller/SRS/srs/PDController_SRS.html#Sec:IMs}{PDContoller case study}, we convert Example~\ref{eq_odeexmaple} to Example~\ref{ex_firstorderode}. Then, we generate a program to solve Example~\ref{ex_firstorderode} numerically. By transitivity, the numerical solution is valid for Example~\ref{eq_odeexmaple} as well.

\section{Connect Explicit Equations to Libraries}
\label{se_connecteetolib}

In previous research, conducted by Brooks, we write the ODEs in a text-based form and store it in a general data pool. Although, the Drasil printer is capable to print ODEs in a SRS, the Drasil Code Generator unable to understand the ODEs in a text-based form. To make study case work, we manually recreate an object (\verb|ODEInfo|) based on the original ODE. Then, the Drasil Code Generator take the new object and generate a solution program. The detail on 
how to generate external libraries in code can be found on Brooks's thesis~\citep{brooks}. However, the previous research only complete generating external libraries for a first-order ODE. For a higher-order ODE, even we convert it to a system of first-order ODE in \verb|ODEInfo|, the Code Generator still unable to generate proper software. The Code Generator unable to retrieve proper initial values when there are more than one initial value. Therefore, we change the type of the initial value to a new type which can hold for multiple values. 


\begin{listing}[ht]
\begin{haskell1}
-- Old 
data ODEInfo = ODEInfo {
  ...
  initVal :: CodeExpr
  ...
}

-- New 
data ODEInfo = ODEInfo {
  ...
  initVal :: [CodeExpr],
  ...
}
\end{haskell1}
\end{listing}

The \verb|ODEInfo| is a datatype store the initial value, we can change the type to be a list. This will enable to store multiple initial values in this data type. Then, we have to make sure the Code Generator know how to handle the type of initial values. 

\begin{listing}[ht]
\begin{haskell1}
-- Old 
initVal info

-- New 
matrix[initVal info]  
\end{haskell1}
\end{listing}

In Drasil Code Generator, we use matrix to represent a list. A list is a special case of matrix which dimension is 1. The info is a data type \verb|ODEInfo|, the code \verb|initVal info| retrieve the initVal from \verb|ODEInfo|. In the new code, the type of initVal is no longer \verb|CodeExpr| but a list of \verb|CodeExpr|. Therefore we change wrap \verb|initVal| in a \verb|matrix|.

\begin{listing}[ht]
\begin{python1}
# Old 
  r.set_initial_value(T_init, 0.0)
  T_W = [T_init]

# New 
  r.set_initial_value([T_init], 0.0)
  T_W = [[T_init][0]]
\end{python1}
\end{listing}

the \verb|set_initial_value| will take a list of initial values instead of just one. \verb|T_W| is collection of the numerical solution. Initial values also is a part of numerical solution, so we have to add the proper initial value to the list.

\begin{listing}[ht]
\begin{csharp1}
// Old 
Vector initv = new Vector(T_init);

// New 
Vector initv = new Vector(new double[] {T_init});
\end{csharp1}
\end{listing}

In Java and C\texttt{++}, the backend code already handle the initial value as a list, so there is no change in artifact for those two languages. 

The implementation unlocks the potential for Drasil to solve any higher-order ODE with its equivalent first-order ODE. This include a higher-order non-linear ODE. Despite the \href{https://jacquescarette.github.io/Drasil/examples/dblpendulum/SRS/srs/DblPendulum_SRS.html#Sec:IMs}{Double Pendulum case study} contains a higher-order non-linear ODE, the Drasil framework now can generate a program to solve it numerically. In the double pendulum case study, we eventually want to solve Equation~\ref{eq_dblpenhigh}. There are two second-order ODEs in one system. To solve this system of ODE, we can convert them into a system of first-order ODE. The transformation follow the methodology we discussed in Section~\ref{se_hightofirst}. We can covert Equation~\ref{eq_dblpenhigh} into Equation~\ref{eq_dblpenfirst}. However, we cannot show Equation~\ref{eq_dblpenfirst} in shape of Equation~\ref{eq_foode} because the ODE is not a linear ODE. Once the transformation complete, we can encode Equation~\ref{eq_dblpenfirst}, and ready pass it to the Drasil Code Generator.

\begin{flalign} \label{eq_dblpenhigh}
& \theta_{1}'' = \frac{-g(2m_{1}+m_{2})\sin \theta_{1}-m_{2}g\sin (\theta_{1}-2\theta_{2})-2\sin (\theta_{1}-\theta_{2})m_{2}({\theta_{2}'}^2L_{2}+{\theta_{1}'}^2L_{1}\cos (\theta_{1}-\theta_{2}))}{L_{1}(2m_{1}+m_{2}-m_{2}\cos (2\theta_{1}-2\theta_{2}))} \\ \nonumber
& \theta_{2}'' = \frac{2\sin (\theta_{1}-\theta_{2})({\theta_{1}'}^2L_{1}(m_{1}+m_{2})+g(m_{1}+m_{2})\cos \theta_{1} + {\theta_{2}'}^2L_{2}m_{2}\cos (\theta_{1}-\theta_{2}))}{L_{2}(2m_{1}+m_{2}-m_{2}\cos (2\theta_{1}-2\theta_{2}))}
\end{flalign}

\begin{flalign} \label{eq_dblpenfirst}
  & \theta_{1}' = \omega_{1} \\ \nonumber
  & \theta_{2}' = \omega_{2} \\ \nonumber
  & \omega_{1}' = \frac{-g(2m_{1}+m_{2})\sin \theta_{1}-m_{2}g\sin (\theta_{1}-2\theta_{2})-2\sin (\theta_{1}-\theta_{2})m_{2}({\omega{2}}^2L_{2}+{\omega{1}}^2L_{1}\cos (\theta_{1}-\theta_{2}))}{L_{1}(2m_{1}+m_{2}-m_{2}\cos (2\theta_{1}-2\theta_{2}))} \\ \nonumber
  & \omega_{2}' = \frac{2\sin (\theta_{1}-\theta_{2})({\omega_{1}}^2L_{1}(m_{1}+m_{2})+g(m_{1}+m_{2})\cos \theta_{1} + {\omega_{2}}^2L_{2}m_{2}\cos (\theta_{1}-\theta_{2}))}{L_{2}(2m_{1}+m_{2}-m_{2}\cos (2\theta_{1}-2\theta_{2}))}
\end{flalign}

The following is an example how we encode Equation~\ref{eq_dblpenfirst} in the Drasil. 
\begin{listing}[ht]
\begin{haskell1}
dblPenODEInfo :: ODEInfo
dblPenODEInfo = odeInfo
...
[3*π/7, 0, 3*π/4, 0]
[ ω₁,
  -g(2m₁ + m₂)sin θ₁ - m₂gsin (θ₁ - 2θ₂) - 2sin (θ₁ - θ₂)m₂(ω₂²L₂ + ω₁²L₁cos (θ₁ - θ₂)) / L₁(2m₁ + m₂ -m₂cos (2θ₁ - 2θ₂)),
  ω₂,
  2sin (θ₁ - θ₂)(ω₁²L₁(m₁ + m₂ ) + g(m₁ + m₂ )cos θ₁ + ω₂²L₂m₂cos (θ₁ - θ₂ )) / L₂(2m₁ + m₂ -m₂cos (2θ₁ - 2θ₂))
]
...
\end{haskell1}
\end{listing}

Once the \verb|dblPenODEInfo| is ready, we will pass it to the Drasil Code generator. We generate the program solve double pendulum in four languages. The detail of each artifacts captured ODE information will show in Appendix A (link)

The limitation with manually created \verb|ODEInfo| is that there we will write the ODE twice. In this case, we encode both Equation~\ref{eq_dblpenhigh} and Equation~\ref{eq_dblpenfirst} in Drasil. They both demonstrate the phenomena of double pendulum, and exist in an isomorphic ODE type. In the next section, we would like to discuss how to automate the transformation from a higher-order ODE to a system of first-order ODE.

\section{Generate Explicit Equations}
Manually creating explicit equations is not ideal because propagates duplicate information. We want to design the Drasil framework to be as fully automatic as possible, so we want to remove human interference. Therefore, an ideal solution is to encode the ODE in a data structure. Then, we extract information from this structure and generate a form which selected external libraries can utilize. Creating the \verb|DifferentialModel| data structure satisfy the need of this idea. We can restructure an ODE base on the information from \verb|DifferentialModel|. The scope of this research only cover generating explicit explicit for a single higher-order ODE. In the future, we are looking for generating explicit equations for a system of higher-order ODE.

Once we encode the ODE in \verb|DifferentialModel|, we want to restructure its equivalent system of first-order ODE in shape of Equation~\ref{eq_foode}. For the convenience of implementation, we shuffle the data around in Equation~\ref{eq_foodeexample}. We reversed the order of \textbf{X}, and it start with $x_{n}$ and end with $x_{1}$. The coefficient matrix \textbf{A} has change as well, but \textbf{X'} and \textbf{c} remain unchanged.


% However, the Double Pendulum case study unable to utilize any function in \verb|DifferentialModel| because it was designed for a linear ODE. Further research is required.

\begin{equation} \label{eq_foodeexamplecolor}
	\begin{bmatrix}
		\highlight{yellow}{x_{1}'} \\
    \highlight{yellow}{\dots} \\
    \highlight{yellow}{x_{n-2}'} \\
    \highlight{yellow}{x_{n-1}'} \\
    \highlight{yellow}{x_{n}'}
	\end{bmatrix}
    = 
  \begin{bmatrix}
		\highlight{orange}{0}, & \highlight{orange}{0}, & \highlight{orange}{\dots}, & \highlight{orange}{1}, & \highlight{orange}{0} \\
    \highlight{orange}{\dots} \\
    \highlight{orange}{0}, & \highlight{orange}{1}, & \highlight{orange}{\dots}, & \highlight{orange}{0}, & \highlight{orange}{0} \\
    \highlight{orange}{1}, & \highlight{orange}{0}, & \highlight{orange}{\dots}, & \highlight{orange}{0}, & \highlight{orange}{0} \\
    \highlight{cyan}{b_{n-1}(t)}, & \highlight{cyan}{b_{n-2}(t)}, & \highlight{cyan}{\dots}, & \highlight{cyan}{b_{1}(t)}, & \highlight{cyan}{b_{0}(t)}
	\end{bmatrix}
    \cdot
  \begin{bmatrix}
    \highlight{yellow}{x_{n}} \\
    \highlight{yellow}{\dots} \\
    \highlight{yellow}{x_{3}} \\
		\highlight{yellow}{x_{2}} \\
    \highlight{yellow}{x_{1}}
	\end{bmatrix}
    + 
  \begin{bmatrix}
    \highlight{lightgray}{0} \\
    \highlight{lightgray}{0} \\
    \highlight{lightgray}{\dots} \\
    \highlight{lightgray}{0} \\
    \highlight{red}{h(t)}
	\end{bmatrix}
\end{equation}

Since the Equation~\ref{eq_foodeexamplecolor} is an expansion of Equation~\ref{eq_foode}, we will use symbols in both equations to explain how to generate Equation~\ref{eq_foodeexamplecolor}. We highlighted \textbf{X'} and \textbf{X} in yellow color in Equation~\ref{eq_foodeexamplecolor}. The number of elements in \textbf{X'} and \textbf{X} depend on how many new dependent variables introduces. If the higher-order ODE is second-order, there will be two new dependent variables be introduced and the length of \textbf{X'} and \textbf{X} will be 2. If the higher-order ODE is nth-order, there will be n new dependent variables be introduced and the length of \textbf{X'} and \textbf{X} will be n. For \textbf{X'}, known it is n-th order ODE, we parameterize $x'$ from 1 to n. For \textbf{X}, known it is n-th order ODE, we parameterize $x$ from n to 1.

In Chapter, we create \verb|DifferentialModel| to represent the ODE in matrix form. 

Fix-me DE model need to remove the highest order. Once we isolation the highest order on the left side, there will be no highest order in right side
Fix-me Code implementation has highest order on the top, data will shuffle around and looks different than report


We highlighted the n$\times$n coefficient matrix \textbf{A} in orange and blue color in Equation~\ref{eq_foodeexamplecolor}. The orange part is the identity matrix. Since the lowest higher-order ODE is a second-order, so the smallest identity matrix is [1, 0]. Equation~\ref{eq_foodeexamplecolortwo} shows a completed transformation for a second-order ODE.
\begin{equation} \label{eq_foodeexamplecolortwo}
	\begin{bmatrix}
		{x_{1}'} \\
    {x_{2}'} 
	\end{bmatrix}
    = 
  \begin{bmatrix}
		{1}, & {0} \\
    {b_{1}(t)}, & {b_{0}(t)}
	\end{bmatrix}
    \cdot
  \begin{bmatrix}
		{x_{2}} \\
    {x_{1}} 
	\end{bmatrix}
    + 
  \begin{bmatrix}
    {0} \\
    {h(t)}
	\end{bmatrix}
\end{equation}

If it is a fourth-order ODE, the \textbf{A} will be a n$\times$n matrix, and the dimension of the identity matrix will be n-1$\times$n. Equation~\ref{eq_foodeexamplecolorfour} shows a completed transformation for a fourth-order ODE.
\begin{equation} \label{eq_foodeexamplecolorfour}
	\begin{bmatrix}
		{x_{1}'} \\
    {x_{2}'} \\
    {x_{3}'} \\
    {x_{4}'}
	\end{bmatrix}
    = 
  \begin{bmatrix}
		{0}, & {0}, & {1}, & {0} \\
    {0}, & {1}, & {0}, & {0} \\
    {1}, & {0}, & {0}, & {0} \\
    {b_{3}(t)}, & {b_{2}(t)}, & {b_{1}(t)}, & {b_{0}(t)}
	\end{bmatrix}
    \cdot
  \begin{bmatrix}
		{x_{4}} \\
    {x_{3}} \\
    {x_{2}} \\
    {x_{1}}
	\end{bmatrix}
    + 
  \begin{bmatrix}
    {0} \\
    {0} \\
    {0} \\
    {h(t)}
	\end{bmatrix}
\end{equation}

The identity matrix starts at (n-1)th row with $[1, 0, \dots]$. If there is a second identity row, we add $[0, 1, \dots]$ before the start row and so on. We observe there is an pattern for the identity matrix, so we can generate it. In Code~\ref{code_createidentity}, \verb|constIdentityRowVect| and \verb|addIdentityValue| are responsible for creating identity rows. We first create a row contain a list of 0. Then, we replace one of 0 to 1. The \verb|addIdentityCoeffs| run through a recursion to add the row at the beginning of the \verb|[[Expr]]|.

\begin{listing}[ht]
\begin{haskell1}
-- | Add Identity Matrix to Coefficients
-- | len is the length of the identity row,
-- | index is the location of identity value (start with 0)
addIdentityCoeffs :: [[Expr]] -> Int -> Int -> [[Expr]]
addIdentityCoeffs es len index
  | len == index + 1 = es
  | otherwise = addIdentityCoeffs (constIdentityRowVect len index : es) len (index + 1)

-- | Construct an identity row vector.
constIdentityRowVect :: Int -> Int -> [Expr]
constIdentityRowVect len index = addIdentityValue index $ replicate len $ exactDbl 0

-- | Recreate the identity row vector with identity value 
addIdentityValue :: Int -> [Expr] -> [Expr]
addIdentityValue n es = fst splits ++ [exactDbl 1] ++ tail (snd splits)
  where splits = splitAt n es
\end{haskell1}
\captionof{listing}{Source code for creating identity matrix(highlighted in orange)}
\label{code_createidentity}
\end{listing}

We highlighted the constant vector \textbf{c} in gray and red color in Equation~\ref{eq_foodeexamplecolor}. The vector \textbf{c} has the length of n. The last element of the constant vector \textbf{c} will be h(t), and anything above h(t) will be zeros. In Code~\ref{code_createconstant}, in \verb|addIdentityConsts|, given the expression of h(t) and the order number of the ODE, we add (n-1) 0s above the h(t). 

\begin{listing}[ht]
\begin{haskell1}
-- | Add Identity Matrix to Constants
-- | len is the size of new constant vector
addIdentityConsts :: [Expr] -> Int -> [Expr]
addIdentityConsts expr len = replicate (len - 1) (exactDbl 0) ++ expr
\end{haskell1}
\captionof{listing}{Source code for creating constant matrix \textbf{c}}
\label{code_createconstant}
\end{listing}

For the blue and red parts in Equation~\ref{eq_foodeexamplecolor}, they can be determined by Equation~\ref{eq_linearDE}. The \verb|DifferentialModel| preserve the relationship for Equation~\ref{eq_linearDE}, but it did not isolate the highest order to the left hand side. To isolate the highest order, we have to move shuffle terms between left hand side and right hand side. 
\begin{equation}
	a_n(t) \cdot y^n(t) + a_{n-1}(t) \cdot y^{n-1}(t) + \dots + a_1(t) \cdot y'(t) + a_0(t) \cdot y(t) = h(t) \nonumber
\end{equation}
Firstly, we move every terms from left to right, except the highest order. 
\begin{equation}
	a_n(t) \cdot y^n(t)  = -a_{n-1}(t) \cdot y^{n-1}(t) + \dots + -a_1(t) \cdot y'(t) + -a_0(t) \cdot y(t) + h(t) \nonumber
\end{equation}
Secondly, we cancel out the coefficient, $a_n(t)$.
\begin{equation}
	y^n(t)  = \frac{-a_{n-1}(t) \cdot y^{n-1}(t) + \dots + -a_1(t) \cdot y'(t) + -a_0(t) \cdot y(t) + h(t)}{a_n(t)} \nonumber
\end{equation}
Lastly, this can be write in a matrix form
\begin{equation} 
  \begin{bmatrix}
		y^n(t)
	\end{bmatrix}
  = 
	\begin{bmatrix}
		-\frac{a_{n-1}(t)}{a_n(t)}, \dots, & -\frac{a_{1}(t)}{a_n(t)} & -\frac{a_{0}(t)}{a_n(t)}
	\end{bmatrix}
	\cdot
	\begin{bmatrix}
		y^{n-1}(t) \\
		\dots \\
    y'(t) \\
		y(t)  
	\end{bmatrix}
	+
	\begin{bmatrix}
		\frac{h(t)}{a_n(t)}
	\end{bmatrix}
  \nonumber
\end{equation}
Base on $x_{n}'$ = $y_{n}$ and relationship in Equation~\ref{eq_newvars}, we can get 
\begin{equation}
  \begin{bmatrix}
		x_{n}'
	\end{bmatrix}
  = 
	\begin{bmatrix}
		-\frac{a_{n-1}(t)}{a_n(t)}, \dots, & -\frac{a_{1}(t)}{a_n(t)} & -\frac{a_{0}(t)}{a_n(t)}
	\end{bmatrix}
	\cdot
	\begin{bmatrix}
		x_{n} \\
		\dots \\
    x_{2} \\
		x_{1}  
	\end{bmatrix}
	+
	\begin{bmatrix}
		\frac{h(t)}{a_n(t)}
	\end{bmatrix}
  \nonumber
\end{equation}
Replacing variables in Equation~\ref{eq_foodeexamplecolor}, we can get a new matrix.
\begin{equation}\label{eq_foodeexampledetail}
	\begin{bmatrix}
		\highlight{yellow}{x_{1}'} \\
    \highlight{yellow}{\dots} \\
    \highlight{yellow}{x_{n-2}'} \\
    \highlight{yellow}{x_{n-1}'} \\
    \highlight{yellow}{x_{n}'}
	\end{bmatrix}
    = 
  \begin{bmatrix}
		\highlight{orange}{0}, & \highlight{orange}{0}, & \highlight{orange}{\dots}, & \highlight{orange}{1}, & \highlight{orange}{0} \\
    \highlight{orange}{\dots} \\
    \highlight{orange}{0}, & \highlight{orange}{1}, & \highlight{orange}{\dots}, & \highlight{orange}{0}, & \highlight{orange}{0} \\
    \highlight{orange}{1}, & \highlight{orange}{0}, & \highlight{orange}{\dots}, & \highlight{orange}{0}, & \highlight{orange}{0} \\
    \highlight{cyan}{-\frac{a_{n-1}(t)}{a_n(t)}}, & \highlight{cyan}{-\frac{a_{n-2}(t)}{a_n(t)}}, & \highlight{cyan}{\dots}, & \highlight{cyan}{-\frac{a_{1}(t)}{a_n(t)}}, & \highlight{cyan}{-\frac{a_{0}(t)}{a_n(t)}}
	\end{bmatrix}
    \cdot
  \begin{bmatrix}
    \highlight{yellow}{x_{n}} \\
    \highlight{yellow}{\dots} \\
    \highlight{yellow}{x_{3}} \\
		\highlight{yellow}{x_{2}} \\
    \highlight{yellow}{x_{1}}
	\end{bmatrix}
    + 
  \begin{bmatrix}
    \highlight{lightgray}{0} \\
    \highlight{lightgray}{0} \\
    \highlight{lightgray}{\dots} \\
    \highlight{lightgray}{0} \\
    \highlight{red}{\frac{h(t)}{a_n(t)}}
	\end{bmatrix}
\end{equation}

Here is the implementation for creating Equation~\ref{eq_foodeexampledetail} in Drasil. In Code~\ref{code_isohighode}, we remove the highest order, because we want to isolate the highest order to the left hand side.
\begin{listing}[ht]
\begin{haskell1}
-- | Delete the highest order
transUnknowns :: [Unknown] -> [Unknown]
transUnknowns = tail
\end{haskell1}
\captionof{listing}{Source code for isolating the highest order}
\label{code_isohighode}
\end{listing}

In Code~\ref{code_cancelcoe}, the \verb|transCoefficients| cancel out the coefficient, $a_n(t)$, in blue highlighted. The \verb|divideCosntants| cancel out the coefficient in red highlighted.
\begin{listing}[ht]
\begin{haskell1}
-- | Cancel the leading coefficient of the highest order in the coefficient matrix
transCoefficients :: [Expr] -> [Expr]
transCoefficients es
  | head es == exactDbl 1 = mapNeg $ tail es
  | otherwise = mapNeg $ tail $ map ($/ head es) es
    where mapNeg = map (\x -> if x == exactDbl 0 then exactDbl 0 else neg x)

-- | divide the leading coefficient of the highest order in constant
divideCosntants :: Expr -> Expr -> Expr
divideCosntants a b
  | b == exactDbl 0 = error "Divisor can't be zero"
  | b == exactDbl 1 = a
  | otherwise       = a $/ b
\end{haskell1}
\captionof{listing}{Source code for canceling the coefficient from the highest order}
\label{code_cancelcoe}
\end{listing}

In Code~\ref{code_odesolverformat}, the \verb|ODESolverFormat| contians information of the system of first-order ODE. The \verb|makeAODESolverFormat| is the smart constructor to create an \verb|ODESolverFormat| by giving a \verb|DifferentialModel|.
\begin{listing}[ht]
\begin{haskell1}
-- Acceptable format for ODE solvers
-- X' = AX + c
-- coeffVects is A - coefficient matrix with identity matrix
-- unknownVect is X - unknown column vector after reduce the highest order
-- constantVect is c - constant column vector with identity matrix, 
-- X' is a column vector of first-order unknowns
data ODESolverFormat = X'{
  coeffVects :: [[Expr]],
  unknownVect :: [Integer],
  constantVect :: [Expr]
}

-- | Construct an ODESolverFormat for solving the ODE.
makeAODESolverFormat :: DifferentialModel -> ODESolverFormat
makeAODESolverFormat dm = X' transEs transUnks transConsts
  where transUnks = transUnknowns $ dm ^. unknowns
        transEs = addIdentityCoeffs [transCoefficients $ head (dm ^. coefficients)] (length transUnks) 0
        transConsts = addIdentityConsts [head (dm ^. dmConstants) `divideCosntants` head (head (dm ^. coefficients))] (length transUnks)
\end{haskell1}
\captionof{listing}{Source code for generating Equation~\ref{eq_foode}, \textbf{X'} = \textbf{AX} + \textbf{c}}
\label{code_odesolverformat}
\end{listing}

In Chapter 3, we mentioned we will solve the ODE as an IVP. In Code~\ref{code_ivpinfo}, we create \verb|InitialValueProblem| to stroe IVP related information.
\begin{listing}[ht]
\begin{haskell1}
-- Information for solving an initial value problem
data InitialValueProblem = IVP{
  initTime :: Expr, -- inital time
  finalTime :: Expr, -- end time
  initValues :: [Expr] -- intial values
}
\end{haskell1}
\captionof{listing}{Source code for IVP infomation}
\label{code_ivpinfo}
\end{listing}

Lastly, in Code~\ref{code_generateodeinfo}, we create a new smart construct to automatically genrate \verb|ODEInfo| by taking \verb|DifferentialModel| from the genreal data pool.
\begin{listing}[ht]
\begin{haskell1}
odeInfo' :: [CodeVarChunk] -> ODEOptions -> DifferentialModel -> InitialValueProblem -> ODEInfo
odeInfo' ovs opt dm ivp = ODEInfo 
  (quantvar $ _indepVar dm) 
  (quantvar $ _depVar dm) 
  ovs 
  (expr $ initTime ivp)
  (expr $ finalTime ivp)
  (map expr $ initValues ivp)
  (createFinalExpr dm)
  opt

createFinalExpr :: DifferentialModel -> [CodeExpr]
createFinalExpr dm = map expr $ formEquations (coeffVects ode) (unknownVect ode) (constantVect ode) (_depVar dm)
  where ode = makeAODESolverFormat dm

formEquations :: [[Expr]] -> [Unknown] -> [Expr] -> ConstrConcept-> [Expr]
formEquations [] _ _ _ = []
formEquations _ [] _ _ = []
formEquations _ _ [] _ = []
formEquations (ex:exs) unks (y:ys) depVa =
  (if y == exactDbl 0 then finalExpr else finalExpr `addRe` y) : formEquations exs unks ys depVa
  where indexUnks = map (idx (sy depVa) . int) unks -- create X
        filteredExprs = filter (\x -> fst x /= exactDbl 0) (zip ex indexUnks) -- remove zero coefficients
        termExprs = map (uncurry mulRe) filteredExprs -- multiple coefficient with depend variables
        finalExpr = foldl1 addRe termExprs -- add terms together
\end{haskell1}
\captionof{listing}{Source code for generating ODEInfo}
\label{code_generateodeinfo}
\end{listing}
