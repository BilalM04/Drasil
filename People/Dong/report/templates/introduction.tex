\chapter{Introduction}
Drasil is a framework that generates software, including code, documentation, software requirement specification, user manual, axillary files, and so on. We call those artifacts ``software artifacts''. By now, the Drasil framework targets generating software to overcome scientific problems. Recently, the Drasil team has been interested in expanding its knowledge to solve a high-order ordinary differential equation (ODE). It would not be difficult to directly add ODE knowledge into the Drasil framework because this requires Drasil to have codified knowledge in ODE, which Drasil currently doesn't have. Thus, we believe a compromised way to solve a high-order ODE is to generate a program interface that connects with its ODE external libraries. There are three main reasons why we want to do that.

1. Scientists and researchers frequently use ODE as a research model in scientific problems, and this model describes the nature phenomenons. Building a research model in software is relatively common, and the software that the Drasil framework generates can solve scientific problems. Thus, expanding the Drasil framework's potential to solve all ODE would solve many scientific problems. Currently, the Drasil can only solve first-order ODEs.

2. Many external libraries are hard to write and embody much knowledge, so the Drasil team wants to re-use them instead of reproducing them. Among many external libraries, libraries that solve ODEs are probably the most important ones. 

3. Another reason is that the Drasil team is interested in how the Drasil framework interacts with external libraries. Once the team understands how to interact between the Drasil framework and external libraries, they will start to add more external libraries. In this way, it would unlock the potential to allow the Drasil framework to solve more scientific problems than before. 

However, the Drasil framework neither captures ODE knowledge nor solves high-order ordinary differential equations. The previous researcher researched to solve a first-order ODE, but it only covers a small area of the knowledge of ordinary differential equations. Adding high-order linear ODEs into the Drasil framework will expand the area where it has never reached before. Therefore, my research will incorporate high-order linear ODEs in a complex knowledge-based and generative environment that can link to externally provided libraries.

To solve a high-order linear ODE, we have to represent ODEs in the Drasil database. On the one hand, users can input an ODE as naturally as writing an ODE in mathematical expressions, such as the example~\ref{eq_odeexmaple}. On the other hand, they can display the ODE in the style of conventional mathematical expressions. The data representation will preserve the relationship between each element in the equation. Then, we will analyze the commonality and variability of selected four external libraries. This analysis will lead us to know how external libraries solve ODEs, what their capabilities are, what options they have, and what interfaces look like. Last, we need to bridge the gap between the Drasil ODE data representation and external libraries. The Drasil ODE data representation can not directly communicate with external libraries. Each library has its standard in terms of solving ODEs. The existing gap requires a transformation from the Drasil ODE data representation to a generic data form before solving ODE in each programming language. Finally, users can run software artifacts to get the numerical solution of the ODE.

Before conducting my research, the Drasil framework can solve explicit equations and numerically solve a first-order ODE. After my research, the Drasil framework will have full capability to solve a high-order linear ODE numerically. Cases study of NoPCM and PDController will utilize a newly created model to generate programs to solve a high-order linear ODE in four different programming languages. In addition, we will explore the possibility of solving a system of ODE numerically. We will introduce a new case study, the double pendulum, which contains an example that solves a system of high-order non-linear ODE.

Chapter 1 will cover how to represent the data of linear ODE in Drasil. Then, in Chapter 2, we will analyze external libraries. In Chapter 3, we will explore how to connect the Drasil ODE data representation with external libraries. Last, we will discuss a user's choice to solve ODE differently in the Drasil framework.