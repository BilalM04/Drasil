\chapter{Your Appendix}
\label{appendix_a}

This appendix provides detailed explanations of various parts of DifferentialModel.

\section{Constructors of DifferentialModel}
\label{const_de}

% \begin{listing}[ht]
\begin{haskell1}
-- $K_d$ is qdDerivGain
-- $y_t$ is opProcessVariable
-- $K_p$ is qdPropGain
-- $r_t$ is qdSetPointTD
imPDRC :: DifferentialModel
imPDRC = makeASingleDE
	time
	opProcessVariable
	lhs
	rhs
	"imPDRC"
	(nounPhraseSP "Computation of the Process Variable as a function of time")
	EmptyS
	where 
	lhs = [exactDbl 1 `addRe` sy qdDerivGain $* (opProcessVariable $^^ 1)]
	$+ (exactDbl 1 $* (opProcessVariable $^^ 2))
	$+ (exactDbl 20 `addRe` sy qdPropGain $* (opProcessVariable $^^ 0))
	rhs = sy qdSetPointTD `mulRe` sy qdPropGain
\end{haskell1}
\captionof{listing}{Using input language for the example~\ref{eq_odeexmaple} in DifferentialModel}
% \label{code_scexinputl}
% \end{listing}

% \begin{listing}[ht]
\begin{haskell1}
imPDRC :: DifferentialModel
imPDRC = makeASystemDE
	time
	opProcessVariable
	coeffs = [[exactDbl 1, exactDbl 1 `addRe` sy qdDerivGain, exactDbl 20 `addRe` sy qdPropGain]]
	unknowns = [2, 1, 0]
	constants = [sy qdSetPointTD `mulRe` sy qdPropGain]
	"imPDRC"
	(nounPhraseSP "Computation of the Process Variable as a function of time")
	EmptyS
\end{haskell1}
\captionof{listing}{Explicitly set values for the example~\ref{eq_odeexmaple} in DifferentialModel}
% \label{code_scexmatrix}
% \end{listing}

\section{Numerical Solution Implementation}
\label{numsol}

% \begin{listing}[ht]
% \begin{python1}
% def func_y_t(K_d, K_p, r_t, t_sim, t_step):
%     def f(t, y_t):
%         return [y_t[1], -(1.0 + K_d) * y_t[1] + -(20.0 + K_p) * y_t[0] + r_t * K_p]
    
%     r = scipy.integrate.ode(f)
%     r.set_integrator("dopri5", atol=Constants.Constants.AbsTol, rtol=Constants.Constants.RelTol)
%     r.set_initial_value([0.0, 0.0], 0.0)
%     y_t = [[0.0, 0.0][0]]
%     while r.successful() and r.t < t_sim:
%         r.integrate(r.t + t_step)
%         y_t.append(r.y[0])
    
%     return y_t
% \end{python1}
% \captionof{listing}{Source code of solving PDController in Scipy}
% \label{code_pythonscipy}
% \end{listing}

% \begin{listing}[ht]
\begin{java1}
public static ArrayList<Double> func_y_t(double K_d, double K_p, double r_t, double t_sim, double t_step) {
	ArrayList<Double> y_t;
	ODEStepHandler stepHandler = new ODEStepHandler();
	ODE ode = new ODE(K_p, K_d, r_t);
	double[] curr_vals = {0.0, 0.0};

	FirstOrderIntegrator it = new DormandPrince54Integrator(t_step, t_step, Constants.AbsTol, Constants.RelTol);
	it.addStepHandler(stepHandler);
	it.integrate(ode, 0.0, curr_vals, t_sim, curr_vals);
	y_t = stepHandler.y_t;

	return y_t;
}
\end{java1}
\captionof{listing}{A linear system of first-order representation in ACM}
% \label{code_javaacm}
% \end{listing}

% \begin{listing}
\begin{cplusplus1}
vector<double> func_y_t(double K_d, double K_p, double r_t, double t_sim, double t_step) {
	vector<double> y_t;
	ODE ode = ODE(K_p, K_d, r_t);
	vector<double> currVals{0.0, 0.0};
	Populate pop = Populate(y_t);
		
	boost::numeric::odeint::runge_kutta_dopri5<vector<double>> rk = boost::numeric::odeint::runge_kutta_dopri5<vector<double>>();
	auto stepper = boost::numeric::odeint::make_controlled(Constants::AbsTol, Constants::RelTol, rk);
	boost::numeric::odeint::integrate_const(stepper, ode, currVals, 0.0, t_sim, t_step, pop);
	
	return y_t;
}	
\end{cplusplus1}
\captionof{listing}{A linear system of first-order representation in ODEINT}
% \label{code_cplusplusodeint}
% \end{listing}

% \begin{listing}[ht]
% \begin{csharp1}
% public static List<double> func_y_t(double K_d, double K_p, double r_t, double t_sim, double t_step) {
% 	List<double> y_t;
% 	Func<double, Vector, Vector> f = (t, y_t_vec) => {
% 		return new Vector(y_t_vec[1], -(1.0 + K_d) * y_t_vec[1] + -(20.0 + K_p) * y_t_vec[0] + r_t * K_p);
% 	};
% 	Options opts = new Options();
% 	opts.AbsoluteTolerance = Constants.AbsTol;
% 	opts.RelativeTolerance = Constants.RelTol;
	
% 	Vector initv = new Vector(new double[] {0.0, 0.0});
% 	IEnumerable<SolPoint> sol = Ode.RK547M(0.0, initv, f, opts);
% 	IEnumerable<SolPoint> points = sol.SolveFromToStep(0.0, t_sim, t_step);
% 	y_t = new List<double> {};
% 	foreach (SolPoint sp in points) {
% 		y_t.Add(sp.X[0]);
% 	}
	
% 	return y_t;
% }
% \end{csharp1}
% \captionof{listing}{Source code of solving PDController in OSLO}
% \label{code_csharposlo}
% \end{listing}


\section{Algorithm in External Libraries}
\label{alg_externallib}

\begin{table}[ht]
\begin{tabular}{ p{0.2\textwidth} p{0.7\textwidth} }
	\textbf{Name} & \textbf{Description} \\
	\toprule
	\verb|zvode| & Complex-valued Variable-coefficient Ordinary Differential Equation solver, with fixed-leading-coefficient implementation. It provides implicit Adams method (for non-stiff problems) and a method based on backward differentiation formulas (BDF) (for stiff problems).\\ \hline
	\verb|lsoda| & Real-valued Variable-coefficient Ordinary Differential Equation solver, with fixed-leading-coefficient implementation. It provides automatic method switching between implicit Adams method (for non-stiff problems) and a method based on backward differentiation formulas (BDF) (for stiff problems).\\ \hline
	\verb|dopri5| & This is an explicit runge-kutta method of order (4)5 due to Dormand \& Prince (with stepsize control and dense output).\\ \hline
	\verb|dop853| & This is an explicit runge-kutta method of order 8(5,3) due to Dormand \& Prince (with stepsize control and dense output).\\
	\bottomrule	
\end{tabular}	
\caption{Algorithm Options in Scipy - Python~\citep{scipyfun}}	
\label{tab_algscipy}
\end{table}

\begin{table}[ht]
\begin{tabular}{ p{0.2\textwidth} p{0.7\textwidth} }
	\textbf{Name} & \textbf{Description} \\
	\toprule
	\verb|RK547M| & This method is most appropriate for solving non-stiff ODE systems. It is based on classical Runge-Kutta formulae with modifications for automatic error and step size control.\\ \hline
	\verb|GearBDF| & It is an implementation of the Gear back differentiation method, a multi-step implicit method for stiff ODE systems solving.\\
	\bottomrule	
\end{tabular}	
\caption{Algorithm Options in OSLO - C\#~\citep{oslofun}}	
\label{tab_algodeint}
\end{table}

\begin{table}[ht]
\begin{tabular}{ p{0.13\textwidth} | p{0.27\textwidth} p{0.57\textwidth} }
	\textbf{Step Size} & \textbf{Name} & \textbf{Description} \\
	\toprule
	Fixed Step & \verb|Euler| & This class implements a simple Euler integrator for Ordinary Differential Equations.\\ \hline
	& \verb|Midpoint| & This class implements a second order Runge-Kutta integrator for Ordinary Differential Equations.\\ \hline
	& \verb|Classical RungeKutta| & This class implements the classical fourth order Runge-Kutta integrator for Ordinary Differential Equations (it is the most often used Runge-Kutta method).\\ \hline
	& \verb|Gill| & This class implements the Gill fourth order Runge-Kutta integrator for Ordinary Differential Equations.\\ \hline
	& \verb|Luther| & This class implements the Luther sixth order Runge-Kutta integrator for Ordinary Differential Equations.\\ \hline
	Adaptive Stepsize & \verb|Higham and Hall| & This class implements the 5(4) Higham and Hall integrator for Ordinary Differential Equations.\\ \hline
	& \verb|DormandPrince 5(4)| & This class implements the 5(4) Dormand-Prince integrator for Ordinary Differential Equations.\\ \hline
	& \verb|DormandPrince 8(5,3)| & This class implements the 8(5,3) Dormand-Prince integrator for Ordinary Differential Equations.\\ \hline
	& \verb|Gragg-Bulirsch-Stoer| & This class implements a Gragg-Bulirsch-Stoer integrator for Ordinary Differential Equations.\\ \hline
	& \verb|Adams-Bashforth| & This class implements explicit Adams-Bashforth integrators for Ordinary Differential Equations.\\ \hline
	& \verb|Adams-Moulton| & This class implements implicit Adams-Moulton integrators for Ordinary Differential Equations.\\
	\bottomrule	
\end{tabular}	
\caption{Algorithm Options in Apache Commons Maths - Java~\citep{apachefun}}	
\label{tab_algacm}
\end{table}

\begin{table}[ht]
\begin{tabular}{ p{0.4\textwidth} p{0.5\textwidth} }
	\textbf{Name} & \textbf{Description} \\
	\toprule
	\verb|euler| & Explicit Euler: Very simple, only for demonstrating purpose\\ \hline
	\verb|runge_kutta4| & Runge-Kutta 4: The classical Runge Kutta scheme, good general scheme without error control.\\ \hline
	\verb|runge_kutta_cash_karp54| & Cash-Karp: Good general scheme with error estimation.\\ \hline
	\verb|runge_kutta_dopri5| & Dormand-Prince 5: Standard method with error control and dense output.\\ \hline
	\verb|runge_kutta_fehlberg78| & Fehlberg 78: Good high order method with error estimation.\\ \hline

	\verb|adams_bashforth_moulton| & Adams-Bashforth-Moulton: Multi-step method with high performance.\\ \hline
	\verb|controlled_runge_kutta| & Controlled Error Stepper: Error control for the Runge-Kutta steppers.\\ \hline
	\verb|dense_output_runge_kutta| & Dense Output Stepper: Dense output for the Runge-Kutta steppers.\\ \hline
	\verb|bulirsch_stoer| & Bulirsch-Stoer: Stepper with step size, order control and dense output. Very good if high precision is required..\\ \hline
	\verb|implicit_euler| & Implicit Euler: Basic implicit routine.\\ \hline
	\verb|rosenbrock4| & Rosenbrock 4: Solver for stiff systems with error control and dense output.\\ \hline
	\verb|symplectic_euler| & Symplectic Euler: Basic symplectic solver for separable Hamiltonian system.\\ \hline
	\verb|symplectic_rkn_sb3a_mclachlan| & Symplectic RKN McLachlan: Symplectic solver for separable Hamiltonian system with order 6.\\
	\bottomrule	
\end{tabular}	
\caption{Algorithm Options in ODEINT - C++~\citep{odeintfun}}	
\label{tab_algodeint}
\end{table}
