\chapter{Conclusion}
An ODE is a type, and it exists in many forms. Previously to this research, the Drasil Framework had no flexible and reusable structure for capturing ODE information. The Drasil teams have to manually extract useful information from the original ODE to instruct the Drasil Code Generator to generate code. This approach propagates duplicated information and loses traceability. The newly created structure, \verb|DifferentialModel|, stores linear ODE information based on the conventional matrix concept. It provides the flexibility to transform a linear ODE from one form to another mathematically equivalent form. Once we capture the knowledge of ODE in the new structure, we can reuse it for other purposes, such as producing the numerical solution and displaying the ODE.

Along with \verb|DifferentialModel|, four selected external libraries are responsible for producing the numerical solution for a system of first-order ODEs. Drasil users can get a numerical solution by choosing an algorithm. Currently, although the Calculations module outputs a finite stream of real numbers, $\mathbb{R}^m$, there are other design options. The C\# OSLO provides an option to output an infinite stream of real numbers, $\mathbb{R}^{\infty}$. It has richer data than $\mathbb{R}^m$. Also, outputting the ODE as a function that can return the value of the dependent variable for any value of the independent variable could help generate libraries in Drasil. We did not complete implementing the new specifications for this, but the analysis provides a starting point for future research.

\verb|DifferentialModel| provides reusable ODE information, and external libraries provide mathematical knowledge for solving the ODE. Before we bridge the gap between \verb|DifferentialModel| and external libraries, we enable solving any higher-order ODEs with manually written equations via external libraries. The Double Pendulum case study demonstrates the Drasil Framework can generate code that solves a system of higher-order nonlinear ODEs. With all implementations, we are ready to bridge the gap by automating the process of extracting useful information from \verb|DifferentialModel| and then forming an \verb|ODEInfo|. While we are solving a single higher-order linear ODE, we generate the \verb|ODEInfo| instead of manually creating it. The automation removes the duplicated information and potentially increases traceability.

This research accomplishes three main goals. Firstly, we capture the knowledge of linear ODE in a flexible and reusable structure. Secondly, we expand the Drasil capability to solve any higher-order ODEs with manually written equations. The last one is removing the duplicated information caused by the implementation of solving ODEs.
