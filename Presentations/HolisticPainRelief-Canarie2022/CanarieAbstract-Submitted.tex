Developing sustainable research software is painful.  Pain points include: not
enough time for documentation and testing; insufficient experience with
development or technology; frequent changes.  One treatment is literate
programming: developers write programs with human understandability as the first
goal. Another pain treatment is code generation, i.e., programs that write code.
In our holistic approach to pain management, we combine literate programming and
code generation via a pervasively generative approach.  It is about way more
than code: we also generate documentation, test scripts, test results, build
files, reports, and other resources. The holistic approach addresses all the
pain points.