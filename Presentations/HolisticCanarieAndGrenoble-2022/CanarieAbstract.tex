A Holistic Approach to Pain Relief for Research Software Developers

Developing sustainable and reproducible research software is painful. Some
sources of pain for developers include the following: lack of time for
documentation and testing; lack of software development, or technology,
experience; and the frequency of change.  One treatment for these pain points is
literate programming, where developers write programs with the first goal of
being understandable to human readers, rather than being artificially
constrained by the ordering and formatting required by conventional computer
code.  A sample literate tool is Org with Babel, which allows mixing code
(potentially in different languages), documentation, test results, and, output
(including graphs as necessary).  Developers maintain the code, documentation,
and testing together. They can immediately see the consequences of their
changes.  Others can understand and reproduce their work.

Another pain treatment tool is code generation, where we write code that writes
code --- in the same way that a compiler generates machine code, but at a higher
level of abstraction. Tools for code generation include some functions in Matlab
and Maple, Spiral for digital signal processing, and FEniCS (Finite Element and
Computational Software).  Code generation allows potentially complex code to
rapidly be written, and when a change occurs, rapidly rewritten.  

In the holistic approach to pain management, we combine literate programming and
code generation through a deeply and pervasively generative approach.  In the
holistic approach, we do not only generate code, but also documentation, test
scripts, test results, build files, reports, and other resources. We build a
knowledge base of models for physics, computing, mathematics, documentation, and
certification and then write explicit ``recipes'' that weave together this
knowledge to generate the desired artifacts.

The holistic approach addresses multiple pain points.  For instance, once the
infrastructure is in place, we can decrease the amount of development time by
automation.  Since a holistic approach allows scientists to focus on their
science, rather than software, the lack of software development experience will
be less of an issue. The holistic approach can capture computing knowledge to
mitigate the technology related pain points.  Frequent change is not a concern
for the holistic approach since developers write the recipes used for generation
at a high level making them relatively easy to change.