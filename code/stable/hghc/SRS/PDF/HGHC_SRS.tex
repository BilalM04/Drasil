\documentclass[12pt]{article}
\usepackage{fontspec}
\usepackage{fullpage}
\usepackage{hyperref}
\hypersetup{bookmarks=true,colorlinks=true,linkcolor=red,citecolor=blue,filecolor=magenta,urlcolor=cyan}
\usepackage{amsmath}
\usepackage{amssymb}
\usepackage{mathtools}
\usepackage{unicode-math}
\usepackage{tabu}
\usepackage{longtable}
\usepackage{booktabs}
\usepackage{caption}
\usepackage{enumitem}
\setmathfont{Latin Modern Math}
\newcommand{\gt}{\ensuremath >}
\newcommand{\lt}{\ensuremath <}
\global\tabulinesep=1mm
\newlist{symbDescription}{description}{1}
\setlist[symbDescription]{noitemsep, topsep=0pt, parsep=0pt, partopsep=0pt}
\title{Software Requirements Specification for HGHC}
\author{W. Spencer Smith}
\begin{document}
\maketitle
\tableofcontents
\newpage
\section{Reference Material}
\label{Sec:RefMat}
This section records information for easy reference.

\subsection{Table of Units}
\label{Sec:ToU}
The unit system used throughout is SI (Système International d'Unités). In addition to the basic units, several derived units are also used. For each unit, the \hyperref[Table:ToU]{Table of Units} lists the symbol, a description and the SI name.

\begin{longtable}{l l l}
\toprule
\textbf{Symbol} & \textbf{Description} & \textbf{SI Name}
\\
\midrule
\endhead
${{}^{\circ}\text{C}}$ & temperature & centigrade
\\
${\text{m}}$ & length & metre
\\
${\text{W}}$ & power & watt
\\
\bottomrule
\caption{Table of Units}
\label{Table:ToU}
\end{longtable}
\subsection{Table of Symbols}
\label{Sec:ToS}
The symbols used in this document are summarized in the \hyperref[Table:ToS]{Table of Symbols} along with their units. The choice of symbols was made to be consistent with the nuclear physics literature and with that used in the FP manual.

\begin{longtabu}{l X[l] l}
\toprule
\textbf{Symbol} & \textbf{Description} & \textbf{Units}
\\
\midrule
\endhead
${h_{\text{b}}}$ & Initial coolant film conductance & --
\\
${h_{\text{c}}}$ & Convective heat transfer coefficient between clad and coolant & $\frac{\text{W}}{\text{m}^{2}{}^{\circ}\text{C}}$
\\
${h_{\text{g}}}$ & Effective heat transfer coefficient between clad and fuel surface & $\frac{\text{W}}{\text{m}^{2}{}^{\circ}\text{C}}$
\\
${h_{\text{p}}}$ & Initial gap film conductance & --
\\
${k_{\text{c}}}$ & Clad conductivity & --
\\
${τ_{\text{c}}}$ & Clad thickness & --
\\
\bottomrule
\caption{Table of Symbols}
\label{Table:ToS}
\end{longtabu}
\section{Specific System Description}
\label{Sec:SpecSystDesc}
This section first presents the problem description, which gives a high-level view of the problem to be solved. This is followed by the solution characteristics specification, which presents the assumptions, theories, and definitions that are used.

\subsection{Solution Characteristics Specification}
\label{Sec:SolCharSpec}
The instance models that govern HGHC are presented in the \hyperref[Sec:IMs]{Instance Model Section}. The information to understand the meaning of the instance models and their derivation is also presented, so that the instance models can be verified.

\subsubsection{Theoretical Models}
\label{Sec:TMs}
There are no theoretical models.

\subsubsection{General Definitions}
\label{Sec:GDs}
There are no general definitions.

\subsubsection{Data Definitions}
\label{Sec:DDs}
This section collects and defines all the data needed to build the instance models.

\vspace{\baselineskip}
\noindent
\begin{minipage}{\textwidth}
\begin{tabular}{>{\raggedright}p{0.13\textwidth}>{\raggedright\arraybackslash}p{0.82\textwidth}}
\toprule \textbf{Refname} & \textbf{DD:htTransCladFuel}
\phantomsection 
\label{DD:htTransCladFuel}
\\ \midrule
Label & Effective heat transfer coefficient between clad and fuel surface
        
\\ \midrule
Symbol & ${h_{\text{g}}}$
         
\\ \midrule
Units & $\frac{\text{W}}{\text{m}^{2}{}^{\circ}\text{C}}$
        
\\ \midrule
Equation & \begin{displaymath}
           {h_{\text{g}}}=\frac{2 {k_{\text{c}}} {h_{\text{p}}}}{2 {k_{\text{c}}}+{τ_{\text{c}}} {h_{\text{p}}}}
           \end{displaymath}
\\ \midrule
Description & \begin{symbDescription}
              \item{${h_{\text{g}}}$ is the effective heat transfer coefficient between clad and fuel surface ($\frac{\text{W}}{\text{m}^{2}{}^{\circ}\text{C}}$)}
              \item{${k_{\text{c}}}$ is the clad conductivity (Unitless)}
              \item{${h_{\text{p}}}$ is the initial gap film conductance (Unitless)}
              \item{${τ_{\text{c}}}$ is the clad thickness (Unitless)}
              \end{symbDescription}
\\ \bottomrule
\end{tabular}
\end{minipage}
\vspace{\baselineskip}
\noindent
\begin{minipage}{\textwidth}
\begin{tabular}{>{\raggedright}p{0.13\textwidth}>{\raggedright\arraybackslash}p{0.82\textwidth}}
\toprule \textbf{Refname} & \textbf{DD:htTransCladCool}
\phantomsection 
\label{DD:htTransCladCool}
\\ \midrule
Label & Convective heat transfer coefficient between clad and coolant
        
\\ \midrule
Symbol & ${h_{\text{c}}}$
         
\\ \midrule
Units & $\frac{\text{W}}{\text{m}^{2}{}^{\circ}\text{C}}$
        
\\ \midrule
Equation & \begin{displaymath}
           {h_{\text{c}}}=\frac{2 {k_{\text{c}}} {h_{\text{b}}}}{2 {k_{\text{c}}}+{τ_{\text{c}}} {h_{\text{b}}}}
           \end{displaymath}
\\ \midrule
Description & \begin{symbDescription}
              \item{${h_{\text{c}}}$ is the convective heat transfer coefficient between clad and coolant ($\frac{\text{W}}{\text{m}^{2}{}^{\circ}\text{C}}$)}
              \item{${k_{\text{c}}}$ is the clad conductivity (Unitless)}
              \item{${h_{\text{b}}}$ is the initial coolant film conductance (Unitless)}
              \item{${τ_{\text{c}}}$ is the clad thickness (Unitless)}
              \end{symbDescription}
\\ \bottomrule
\end{tabular}
\end{minipage}
\subsubsection{Instance Models}
\label{Sec:IMs}
There are no instance models.

\end{document}
